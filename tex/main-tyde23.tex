\documentclass[sigplan,review,screen]{acmart}
\pdfoutput=1
\usepackage{agda}
\usepackage{stmaryrd}
\usepackage{listings}

%% There is a limit of 25 pages for a full paper or Functional Pearl
%% and 12 pages for an Experience Report; in either case, the
%% bibliography and an optional clearly marked appendix will not be
%% counted against these limits. Submissions that exceed the page
%% limits or, for other reasons, do not meet the requirements for
%% formatting, will be summarily rejected. 


%% Rights management information.  This information is sent to you
%% when you complete the rights form.  These commands have SAMPLE
%% values in them; it is your responsibility as an author to replace
%% the commands and values with those provided to you when you
%% complete the rights form.
\setcopyright{acmcopyright}
\copyrightyear{2023}
\acmYear{2023}
\acmDOI{XXXXXXX.XXXXXXX}


%%
%% These commands are for a JOURNAL article.
% \acmJournal{JACM}
% \acmVolume{37}
% \acmNumber{4}
% \acmArticle{111}
% \acmMonth{8}

%%
%% Submission ID.
%% Use this when submitting an article to a sponsored event. You'll
%% receive a unique submission ID from the organizers
%% of the event, and this ID should be used as the parameter to this command.
%%\acmSubmissionID{123-A56-BU3}

%%
%% For managing citations, it is recommended to use bibliography
%% files in BibTeX format.
%%
%% You can then either use BibTeX with the ACM-Reference-Format style,
%% or BibLaTeX with the acmnumeric or acmauthoryear sytles, that include
%% support for advanced citation of software artefact from the
%% biblatex-software package, also separately available on CTAN.
%%
%% Look at the sample-*-biblatex.tex files for templates showcasing
%% the biblatex styles.
%%

%%
%% The majority of ACM publications use numbered citations and
%% references.  The command \citestyle{authoryear} switches to the
%% "author year" style.
%%
%% If you are preparing content for an event
%% sponsored by ACM SIGGRAPH, you must use the "author year" style of
%% citations and references.
%% Uncommenting
%% the next command will enable that style.
%%\citestyle{acmauthoryear}


%%
%% end of the preamble, start of the body of the document source.
\usepackage{dsfont}
\usepackage{newunicodechar}
\newunicodechar{λ}{\ensuremath{\mathnormal\lambda}}
\newunicodechar{σ}{\ensuremath{\mathnormal\sigma}}
\newunicodechar{τ}{\ensuremath{\mathnormal\tau}}
\newunicodechar{π}{\ensuremath{\mathnormal\pi}}
\newunicodechar{ℕ}{\ensuremath{\mathbb{N}}}
\newunicodechar{∷}{\ensuremath{::}}
\newunicodechar{≡}{\ensuremath{\equiv}}
\newunicodechar{≅}{\ensuremath{\cong}}
\newunicodechar{∀}{\ensuremath{\forall}}
\newunicodechar{ᴸ}{\ensuremath{^L}}
\newunicodechar{ᴿ}{\ensuremath{^R}}
\newunicodechar{ʳ}{\ensuremath{^r}}
\newunicodechar{ⱽ}{\ensuremath{^V}}
\newunicodechar{⟧}{\ensuremath{\rrbracket}}
\newunicodechar{⟦}{\ensuremath{\llbracket}}
\newunicodechar{⊤}{\ensuremath{\top}}
\newunicodechar{⊥}{\ensuremath{\bot}}
\newunicodechar{₁}{\ensuremath{_1}}
\newunicodechar{₂}{\ensuremath{_2}}
\newunicodechar{₃}{\ensuremath{_3}}
\newunicodechar{₄}{\ensuremath{_4}}
\newunicodechar{₅}{\ensuremath{_5}}
\newunicodechar{∈}{\ensuremath{\in}}
\newunicodechar{₀}{\ensuremath{_0}}
\newunicodechar{′}{\ensuremath{'}}
\newunicodechar{ˢ}{\ensuremath{^S}}
\newunicodechar{ᴬ}{\ensuremath{^A}}
\newunicodechar{∘}{\ensuremath{\circ}}
\newunicodechar{𝟙}{\ensuremath{\mathds{1}}}  
\newunicodechar{𝟘}{\ensuremath{\mathds{O}}}
% \newunicodechar{𝟙}{\ensuremath{\mathbb{I}}}  
% \newunicodechar{𝟘}{\ensuremath{\mathbb{O}}}
\newunicodechar{ᴾ}{\ensuremath{^P}}
\newunicodechar{ᵀ}{\ensuremath{^T}}
\newunicodechar{⊎}{\ensuremath{\uplus}}
\newunicodechar{ι}{\ensuremath{\iota}}
\newunicodechar{⇐}{\ensuremath{\Leftarrow}}
\newunicodechar{⇒}{\ensuremath{\Rightarrow}}
\newunicodechar{∎}{\ensuremath{\mathnormal\blacksquare}}
\newunicodechar{➙}{\ensuremath{\to^P}}
\newunicodechar{Δ}{\ensuremath{\Delta}}
\newunicodechar{∅}{\ensuremath{\emptyset}}
\newunicodechar{⁺}{\ensuremath{^+}}
\newunicodechar{𝕏}{\ensuremath{\mathbb{X}}}
%\newunicodechar{·}{\ensuremath{\cdot}} %seems to be defined already!
\newunicodechar{∙}{\ensuremath{\bullet}}
\newunicodechar{⁇}{\ensuremath{?}}
\newunicodechar{‼}{\ensuremath{!}}
\newunicodechar{⊕}{\ensuremath{\oplus}}
\newunicodechar{ℤ}{\ensuremath{\mathbb{Z}}}
\newunicodechar{μ}{\ensuremath{\mu}}
\newunicodechar{∃}{\ensuremath{\exists}}
\newunicodechar{⨟}{\ensuremath{\fatsemi}}
\newunicodechar{Σ}{\ensuremath{\Sigma}}
\newunicodechar{ᵣ}{\ensuremath{_r}}
\newunicodechar{ᵢ}{\ensuremath{_i}}
\newunicodechar{≢}{\ensuremath{\nequiv}}
\newunicodechar{≟}{\ensuremath{\stackrel{{\tiny?}}{=}}}
\newunicodechar{≤}{\ensuremath{\le}}
\newunicodechar{ᵇ}{\ensuremath{^b}}
\newunicodechar{𝓣}{\ensuremath{\mathcal{T}}}
\newunicodechar{𝓔}{\ensuremath{\mathcal{E}}}
\newunicodechar{Γ}{\ensuremath{\Gamma}}
\newunicodechar{γ}{\ensuremath{\gamma}}
\newunicodechar{⊔}{\ensuremath{\sqcup}}
\newunicodechar{α}{\ensuremath{\alpha}}
\newunicodechar{η}{\ensuremath{\eta}}
\newunicodechar{ω}{\ensuremath{\omega}}
\newunicodechar{◁}{\ensuremath{\lhd}}
%PT: seems to be already defined
%\newcommand{\lambdabar}{{\mkern0.75mu\mathchar '26\mkern -9.75mu\lambda}}
\newunicodechar{ƛ}{\ensuremath{\lambdabar}}
\newunicodechar{Λ}{\ensuremath{\mathnormal\Lambda}}
\newunicodechar{ρ}{\ensuremath{\rho}}
\newunicodechar{𝓖}{\ensuremath{\mathcal{G}}}
\newunicodechar{ℓ}{\ensuremath{\ell}}
\newunicodechar{♯}{\ensuremath{\sharp}}
\newunicodechar{⇓}{\ensuremath{\Downarrow}}
\newunicodechar{𝓥}{\ensuremath{\mathcal{V}}}
\newunicodechar{∧}{\ensuremath{\wedge}}
\newunicodechar{ₛ}{\ensuremath{_s}}
\newunicodechar{χ}{\ensuremath{\chi}}
\newunicodechar{⊨}{\ensuremath{\models}}
\newunicodechar{⦂}{\ensuremath{\mathbf{:}}}
\newunicodechar{ς}{\ensuremath{\varsigma}}
\newunicodechar{₃}{\ensuremath{_3}}
\newunicodechar{■}{ensuremath{\qed}}
\newcommand\Aamp{\AgdaFunction{\ensuremath{\&}}}
\newcommand\Att{\AgdaInductiveConstructor{tt}}
\newcommand\Ado{\AgdaKeyword{do}}
\newcommand\AZ{\AgdaDatatype{ℤ}}
\newcommand\Asuc{\AgdaInductiveConstructor{suc}}
\newcommand\Azero{\AgdaInductiveConstructor{zero}}
\newcommand\Anat{\AgdaInductiveConstructor{nat}}
\newcommand\Aint{\AgdaInductiveConstructor{int}}
\newcommand\Abool{\AgdaInductiveConstructor{bool}}
\newcommand\Atend{\AgdaInductiveConstructor{end}}
\newcommand\Atsend[2]{\AgdaInductiveConstructor{‼{\textcolor{black}{\ensuremath{#1}}}∙{\textcolor{black}{\ensuremath{#2}}}}}
\newcommand\Atrecv[2]{\AgdaInductiveConstructor{⁇{\textcolor{black}{\ensuremath{#1}}}∙{\textcolor{black}{\ensuremath{#2}}}}}
\newcommand\Atcfsend[1]{\AgdaInductiveConstructor{‼{\textcolor{black}{\ensuremath{#1}}}}}
\newcommand\Atcfrecv[1]{\AgdaInductiveConstructor{⁇{\textcolor{black}{\ensuremath{#1}}}}}
\newcommand\Atcfcomp[2]{\AgdaInductiveConstructor{{\textcolor{black}{\ensuremath{#1}}}⨟{\textcolor{black}{\ensuremath{#2}}}}}
\newcommand\Atcfskip{\AgdaInductiveConstructor{skip}}
\newcommand\ACSKIP{\AgdaInductiveConstructor{SKIP}}
\newcommand\ACEND{\AgdaInductiveConstructor{END}}
\newcommand\ACCLOSE{\AgdaInductiveConstructor{CLOSE}}
\newcommand\ACfork{\AgdaInductiveConstructor{fork}}
\newcommand\ACconnect{\AgdaInductiveConstructor{connect}}
\newcommand\ACterminate{\AgdaInductiveConstructor{terminate}}
\newcommand\ACdelegateIN{\AgdaInductiveConstructor{delegateIN}}
\newcommand\ACdelegateOUT{\AgdaInductiveConstructor{delegateOUT}}
\newcommand\ACtransmit{\AgdaInductiveConstructor{transmit}}
\newcommand\ACbranch{\AgdaInductiveConstructor{branch}}
\newcommand\ACclose{\AgdaInductiveConstructor{close}}
\newcommand\ACSEND{\AgdaInductiveConstructor{SEND}}
\newcommand\ACRECV{\AgdaInductiveConstructor{RECV}}
\newcommand\ACSELECT{\AgdaInductiveConstructor{SELECT}}
\newcommand\ACCHOICE{\AgdaInductiveConstructor{CHOICE}}
\newcommand\AFin{\AgdaDatatype{Fin}}
\newcommand\AXCommand{\AgdaDatatype{XCmd}}
\newcommand\ACommand{\AgdaDatatype{Cmd}}
\newcommand\ACommandStack{\AgdaDatatype{CmdStack}}
\newcommand\ASession{\AgdaDatatype{Session}}
\newcommand\ASplit{\AgdaDatatype{Split}}
\newcommand\AMSession{\AgdaDatatype{MSession}}
\newcommand\ASet{\AgdaDatatype{Set}}
\newcommand\ASetOne{\AgdaDatatype{Set$_1$}}
\newcommand\ASeto{\AgdaDatatype{Setω}}
\newcommand\Abinaryp{\AgdaFunction{binaryp}}
\newcommand\Aunaryp{\AgdaFunction{unaryp}}
\newcommand\ACheck{\AgdaFunction{Check}}
\newcommand\ACausality{\AgdaFunction{Causality}}
\newcommand\ACheckDual{\AgdaFunction{CheckDual0}}
\newcommand\Aexecutor{\AgdaFunction{exec}}
\newcommand\Aexec{\AgdaFunction{exec}}
\newcommand\AIO{\AgdaFunction{IO}}
\newcommand\Amu{\AgdaInductiveConstructor{\ensuremath{\mu}}}
\newcommand\AMU{\AgdaInductiveConstructor{LOOP}}
\newcommand\AUNROLL{\AgdaInductiveConstructor{UNROLL}}
\newcommand\ACONTINUE{\AgdaInductiveConstructor{CONTINUE}}
\newcommand\Aback{\AgdaInductiveConstructor{\ensuremath{`}}}
\newcommand\Amanyunaryp{\AgdaFunction{many-unaryp}}
\newcommand\Arestart{\AgdaFunction{restart}}
\newcommand\ASerialize{\AgdaRecord{Serialize}}
\newcommand\ARawMonad{\AgdaRecord{RawMonad}}
\newcommand\AReaderT{\AgdaRecord{ReaderT}}
\newcommand\AStateT{\AgdaRecord{StateT}}
\newcommand\Aput{\AgdaFunction{put}}
\newcommand\Amodify{\AgdaFunction{modify}}
\newcommand\Aget{\AgdaFunction{get}}
\newcommand\Aproject{\AgdaFunction{project}}
\newcommand\AlocateSplit{\AgdaFunction{locate-split}}
\newcommand\Aadjust{\AgdaFunction{adjust}}
\newcommand\Aid{\AgdaFunction{id}}
\newcommand\AprimSend{\AgdaFunction{primSend}}
\newcommand\AprimRecv{\AgdaFunction{primRecv}}
\newcommand\Aleafp{\AgdaFunction{leafp}}
\newcommand\Abranchp{\AgdaFunction{branchp}}
\newcommand\Atreep{\AgdaFunction{treep}}
\newcommand\AIntTree{\AgdaDatatype{IntTree}}
\newcommand\AIntTreeF{\AgdaFunction{IntTreeF}}
\newcommand\ACLeaf{\AgdaInductiveConstructor{Leaf}}
\newcommand\ACBranch{\AgdaInductiveConstructor{Branch}}
\newcommand\AtoN{\AgdaFunction{toℕ}}
\newcommand\Asplit{\AgdaFunction{split}}
\newcommand\AisValue{\AgdaDatatype{isValue}}
\newcommand\AValue{\AgdaDatatype{Value}}
\newcommand\Ajoin{\AgdaFunction{join}}
\newcommand\ACExpr{\AgdaFunction{CExpr}}
\newcommand\Acont{\AgdaFunction{cont}}
\newcommand\AlevelEnv{\AgdaFunction{levelEnv}}
\newcommand\Adollar{\AgdaOperator{\AgdaFunction{\AgdaUnderscore{}\$\AgdaUnderscore{}}}}
\newcommand\AVSem{\AgdaOperator{\AgdaFunction{𝓥⟦\AgdaUnderscore{}⟧}}}
\newcommand\AESem{\AgdaOperator{\AgdaFunction{𝓔⟦\AgdaUnderscore{}⟧}}}
\newcommand\AGSem{\AgdaOperator{\AgdaDatatype{𝓖⟦\AgdaUnderscore{}⟧}}}
\newcommand\ATSem{\AgdaOperator{\AgdaDatatype{𝓣⟦\AgdaUnderscore{}⟧}}}
\newcommand\ADEnv{\AgdaDatatype{DEnv}}
\newcommand\AREL{\AgdaFunction{REL}}
\newcommand\ARelEnv{\AgdaFunction{RelEnv}}
\newcommand{\AV}{\AgdaFunction{𝓥}}
\newcommand{\Asubst}{\AgdaFunction{subst}}
\newcommand{\Arefl}{\AgdaInductiveConstructor{refl}}
\newcommand{\AValueDown}{\AgdaFunction{Value-\ensuremath{\Downarrow}}}
\newcommand{\AGLookup}{\AgdaFunction{𝓖-lookup}}
\newcommand{\ACsubClosed}{\AgdaFunction{Csub-closed}}

%%% Local Variables:
%%% mode: latex
%%% TeX-master: "main-icfp24"
%%% End:


%% listings
\lstset{
  language=haskell,
  basicstyle=\ttfamily,
  morekeywords={send,recv,close,send_on_,receive,End,Int,Chan},
  captionpos=b,
  literate=
  {-o}{${\multimap}$}1
  {ex}{$\exists$}1
  {all}{$\forall$}1
  {\\x}{$\times$}1
  %{(x)}{$\otimes$}1
  {fresh}{$\lozenge$}1
  {~~}{$\approx$}1
}

\begin{document}

%%
%% The "title" command has an optional parameter,
%% allowing the author to define a "short title" to be used in page headers.
\title{Towards Tagless Interpretation of Stratified System F}
\subtitle{Extended Abstract}

%%
%% The "author" command and its associated commands are used to define
%% the authors and their affiliations.
%% Of note is the shared affiliation of the first two authors, and the
%% "authornote" and "authornotemark" commands
%% used to denote shared contribution to the research.
\author{Peter Thiemann}
% \authornote{Both authors contributed equally to this research.}
\email{thiemann@acm.org}
\orcid{0000-0002-9000-1239}
\affiliation{%
  \institution{University of Freiburg}
  % \streetaddress{P.O. Box 1212}
  % \city{Dublin}
  % \state{Ohio}
  \country{Germany}
  % \postcode{43017-6221}
}

\author{Marius Weidner}
% \authornotemark[1]
\email{weidner@cs.uni-freiburg.de}
\affiliation{%
  \institution{University of Freiburg}
  % \streetaddress{P.O. Box 1212}
  % \city{Dublin}
  % \state{Ohio}
  \country{Germany}
  % \postcode{43017-6221}
}

%%
%% By default, the full list of authors will be used in the page
%% headers. Often, this list is too long, and will overlap
%% other information printed in the page headers. This command allows
%% the author to define a more concise list
%% of authors' names for this purpose.
% \renewcommand{\shortauthors}{Trovato et al.}

%%
%% The abstract is a short summary of the work to be presented in the
%% article.
\begin{abstract}
We explore the definition of an intrinsically typed interpreter for
stratified System F in Agda.
\end{abstract}

%%
%% The code below is generated by the tool at http://dl.acm.org/ccs.cfm.
%% Please copy and paste the code instead of the example below.
%%
% \begin{CCSXML}
% <ccs2012>
%  <concept>
%   <concept_id>10010520.10010553.10010562</concept_id>
%   <concept_desc>Computer systems organization~Embedded systems</concept_desc>
%   <concept_significance>500</concept_significance>
%  </concept>
%  <concept>
%   <concept_id>10010520.10010575.10010755</concept_id>
%   <concept_desc>Computer systems organization~Redundancy</concept_desc>
%   <concept_significance>300</concept_significance>
%  </concept>
%  <concept>
%   <concept_id>10010520.10010553.10010554</concept_id>
%   <concept_desc>Computer systems organization~Robotics</concept_desc>
%   <concept_significance>100</concept_significance>
%  </concept>
%  <concept>
%   <concept_id>10003033.10003083.10003095</concept_id>
%   <concept_desc>Networks~Network reliability</concept_desc>
%   <concept_significance>100</concept_significance>
%  </concept>
% </ccs2012>
% \end{CCSXML}

% \ccsdesc[500]{Computer systems organization~Embedded systems}
% \ccsdesc[300]{Computer systems organization~Redundancy}
% \ccsdesc{Computer systems organization~Robotics}
% \ccsdesc[100]{Networks~Network reliability}

%%
%% Keywords. The author(s) should pick words that accurately describe
%% the work being presented. Separate the keywords with commas.
\keywords{Agda, stratified System F, extensionality}

% \received{20 February 2007}
% \received[revised]{12 March 2009}
% \received[accepted]{5 June 2009}

%%
%% This command processes the author and affiliation and title
%% information and builds the first part of the formatted document.
\maketitle

\section{Introduction}
\label{sec:introduction}

Defining semantics is one of the key activities of a programming
languages researcher. We learn that there are different styles of
dynamics (small-step, big-step, denotational, just to name the most
frequently used one), each with different trade-offs. When it comes to
implementing or mechanizing semantics, there are further options to
choose from, in particular if we are also interested in statics.

One important choice is whether we want to express the statics
extrinsically or intrinsically, that is, do we want to start with
untyped syntax and then define the statics as an afterthought, or do
we integrate types with the syntax.

If we opt for intrinsically typed syntax, we get quite some properties
for free. For instance, a small-step semantics for intrinsically typed
syntax satisfies type preservation by construction. For another
instance, consider implementing a denotational semantics in the form
of an interpreter. With untyped syntax, the interpreter has to use
type tags and perform tag checks at run time. But with intrinsically
typed syntax we can write tagless interpreters which elide type tags
entirely.

For concreteness, we show the well-known example of a tagless
interpreter for the simply-typed lambda calculus implemented in Agda in
Figure~\ref{fig:stlc}. We define the syntax as an inductive data type along with
a compositional mapping to the semantic domain, spanned by Agda's
natural number type and the function space. 
\begin{figure}[tp]
  \begin{code}%
\>[0]\AgdaKeyword{module}\AgdaSpace{}%
\AgdaModule{STLC}\AgdaSpace{}%
\AgdaKeyword{where}\<%
\\
%
\\[\AgdaEmptyExtraSkip]%
\>[0]\AgdaKeyword{open}\AgdaSpace{}%
\AgdaKeyword{import}\AgdaSpace{}%
\AgdaModule{Data.Nat}%
\>[22]\AgdaKeyword{using}\AgdaSpace{}%
\AgdaSymbol{(}\AgdaDatatype{ℕ}\AgdaSymbol{;}\AgdaSpace{}%
\AgdaInductiveConstructor{zero}\AgdaSymbol{;}\AgdaSpace{}%
\AgdaInductiveConstructor{suc}\AgdaSymbol{)}\<%
\\
\>[0]\AgdaKeyword{open}\AgdaSpace{}%
\AgdaKeyword{import}\AgdaSpace{}%
\AgdaModule{Data.List}\AgdaSpace{}%
\AgdaKeyword{using}\AgdaSpace{}%
\AgdaSymbol{(}\AgdaDatatype{List}\AgdaSymbol{;}\AgdaSpace{}%
\AgdaInductiveConstructor{[]}\AgdaSymbol{;}\AgdaSpace{}%
\AgdaOperator{\AgdaInductiveConstructor{\AgdaUnderscore{}∷\AgdaUnderscore{}}}\AgdaSymbol{)}\<%
\\
%
\\[\AgdaEmptyExtraSkip]%
\>[0]\AgdaKeyword{data}\AgdaSpace{}%
\AgdaDatatype{Type}\AgdaSpace{}%
\AgdaSymbol{:}\AgdaSpace{}%
\AgdaPrimitive{Set}\AgdaSpace{}%
\AgdaKeyword{where}\<%
\\
\>[0][@{}l@{\AgdaIndent{0}}]%
\>[2]\AgdaInductiveConstructor{nat}\AgdaSpace{}%
\AgdaSymbol{:}\AgdaSpace{}%
\AgdaDatatype{Type}\<%
\\
%
\>[2]\AgdaOperator{\AgdaInductiveConstructor{\AgdaUnderscore{}⇒\AgdaUnderscore{}}}\AgdaSpace{}%
\AgdaSymbol{:}\AgdaSpace{}%
\AgdaDatatype{Type}\AgdaSpace{}%
\AgdaSymbol{→}\AgdaSpace{}%
\AgdaDatatype{Type}\AgdaSpace{}%
\AgdaSymbol{→}\AgdaSpace{}%
\AgdaDatatype{Type}\<%
\\
%
\\[\AgdaEmptyExtraSkip]%
\>[0]\AgdaOperator{\AgdaFunction{𝓣⟦\AgdaUnderscore{}⟧}}\AgdaSpace{}%
\AgdaSymbol{:}\AgdaSpace{}%
\AgdaDatatype{Type}\AgdaSpace{}%
\AgdaSymbol{→}\AgdaSpace{}%
\AgdaPrimitive{Set}\<%
\\
\>[0]\AgdaOperator{\AgdaFunction{𝓣⟦}}\AgdaSpace{}%
\AgdaInductiveConstructor{nat}\AgdaSpace{}%
\AgdaOperator{\AgdaFunction{⟧}}\AgdaSpace{}%
\AgdaSymbol{=}\AgdaSpace{}%
\AgdaDatatype{ℕ}\<%
\\
\>[0]\AgdaOperator{\AgdaFunction{𝓣⟦}}\AgdaSpace{}%
\AgdaBound{S}\AgdaSpace{}%
\AgdaOperator{\AgdaInductiveConstructor{⇒}}\AgdaSpace{}%
\AgdaBound{T}\AgdaSpace{}%
\AgdaOperator{\AgdaFunction{⟧}}\AgdaSpace{}%
\AgdaSymbol{=}\AgdaSpace{}%
\AgdaOperator{\AgdaFunction{𝓣⟦}}\AgdaSpace{}%
\AgdaBound{S}\AgdaSpace{}%
\AgdaOperator{\AgdaFunction{⟧}}\AgdaSpace{}%
\AgdaSymbol{→}\AgdaSpace{}%
\AgdaOperator{\AgdaFunction{𝓣⟦}}\AgdaSpace{}%
\AgdaBound{T}\AgdaSpace{}%
\AgdaOperator{\AgdaFunction{⟧}}\<%
\\
%
\\[\AgdaEmptyExtraSkip]%
\>[0]\AgdaFunction{Env}\AgdaSpace{}%
\AgdaSymbol{=}\AgdaSpace{}%
\AgdaDatatype{List}\AgdaSpace{}%
\AgdaDatatype{Type}\<%
\\
%
\\[\AgdaEmptyExtraSkip]%
\>[0]\AgdaKeyword{data}\AgdaSpace{}%
\AgdaOperator{\AgdaDatatype{\AgdaUnderscore{}∈\AgdaUnderscore{}}}\AgdaSpace{}%
\AgdaSymbol{:}\AgdaSpace{}%
\AgdaDatatype{Type}\AgdaSpace{}%
\AgdaSymbol{→}\AgdaSpace{}%
\AgdaFunction{Env}\AgdaSpace{}%
\AgdaSymbol{→}\AgdaSpace{}%
\AgdaPrimitive{Set}\AgdaSpace{}%
\AgdaKeyword{where}\<%
\\
\>[0][@{}l@{\AgdaIndent{0}}]%
\>[2]\AgdaInductiveConstructor{here}%
\>[8]\AgdaSymbol{:}\AgdaSpace{}%
\AgdaSymbol{∀}\AgdaSpace{}%
\AgdaSymbol{\{}\AgdaBound{T}\AgdaSpace{}%
\AgdaBound{Γ}\AgdaSymbol{\}}\AgdaSpace{}%
\AgdaSymbol{→}\AgdaSpace{}%
\AgdaBound{T}\AgdaSpace{}%
\AgdaOperator{\AgdaDatatype{∈}}\AgdaSpace{}%
\AgdaSymbol{(}\AgdaBound{T}\AgdaSpace{}%
\AgdaOperator{\AgdaInductiveConstructor{∷}}\AgdaSpace{}%
\AgdaBound{Γ}\AgdaSymbol{)}\<%
\\
%
\>[2]\AgdaInductiveConstructor{there}\AgdaSpace{}%
\AgdaSymbol{:}\AgdaSpace{}%
\AgdaSymbol{∀}\AgdaSpace{}%
\AgdaSymbol{\{}\AgdaBound{S}\AgdaSpace{}%
\AgdaBound{T}\AgdaSpace{}%
\AgdaBound{Γ}\AgdaSymbol{\}}\AgdaSpace{}%
\AgdaSymbol{→}\AgdaSpace{}%
\AgdaBound{S}\AgdaSpace{}%
\AgdaOperator{\AgdaDatatype{∈}}\AgdaSpace{}%
\AgdaBound{Γ}\AgdaSpace{}%
\AgdaSymbol{→}\AgdaSpace{}%
\AgdaBound{S}\AgdaSpace{}%
\AgdaOperator{\AgdaDatatype{∈}}\AgdaSpace{}%
\AgdaSymbol{(}\AgdaBound{T}\AgdaSpace{}%
\AgdaOperator{\AgdaInductiveConstructor{∷}}\AgdaSpace{}%
\AgdaBound{Γ}\AgdaSymbol{)}\<%
\\
%
\\[\AgdaEmptyExtraSkip]%
\>[0]\AgdaKeyword{data}\AgdaSpace{}%
\AgdaDatatype{Expr}\AgdaSpace{}%
\AgdaSymbol{(}\AgdaBound{Γ}\AgdaSpace{}%
\AgdaSymbol{:}\AgdaSpace{}%
\AgdaFunction{Env}\AgdaSymbol{)}\AgdaSpace{}%
\AgdaSymbol{:}\AgdaSpace{}%
\AgdaDatatype{Type}\AgdaSpace{}%
\AgdaSymbol{→}\AgdaSpace{}%
\AgdaPrimitive{Set}\AgdaSpace{}%
\AgdaKeyword{where}\<%
\\
\>[0][@{}l@{\AgdaIndent{0}}]%
\>[2]\AgdaInductiveConstructor{con}\AgdaSpace{}%
\AgdaSymbol{:}\AgdaSpace{}%
\AgdaDatatype{ℕ}\AgdaSpace{}%
\AgdaSymbol{→}\AgdaSpace{}%
\AgdaDatatype{Expr}\AgdaSpace{}%
\AgdaBound{Γ}\AgdaSpace{}%
\AgdaInductiveConstructor{nat}\<%
\\
%
\>[2]\AgdaInductiveConstructor{var}\AgdaSpace{}%
\AgdaSymbol{:}\AgdaSpace{}%
\AgdaSymbol{∀}\AgdaSpace{}%
\AgdaSymbol{\{}\AgdaBound{T}\AgdaSymbol{\}}\AgdaSpace{}%
\AgdaSymbol{→}\AgdaSpace{}%
\AgdaBound{T}\AgdaSpace{}%
\AgdaOperator{\AgdaDatatype{∈}}\AgdaSpace{}%
\AgdaBound{Γ}\AgdaSpace{}%
\AgdaSymbol{→}\AgdaSpace{}%
\AgdaDatatype{Expr}\AgdaSpace{}%
\AgdaBound{Γ}\AgdaSpace{}%
\AgdaBound{T}\<%
\\
%
\>[2]\AgdaInductiveConstructor{lam}\AgdaSpace{}%
\AgdaSymbol{:}\AgdaSpace{}%
\AgdaSymbol{∀}\AgdaSpace{}%
\AgdaSymbol{\{}\AgdaBound{S}\AgdaSpace{}%
\AgdaBound{T}\AgdaSymbol{\}}\AgdaSpace{}%
\AgdaSymbol{→}\AgdaSpace{}%
\AgdaDatatype{Expr}\AgdaSpace{}%
\AgdaSymbol{(}\AgdaBound{S}\AgdaSpace{}%
\AgdaOperator{\AgdaInductiveConstructor{∷}}\AgdaSpace{}%
\AgdaBound{Γ}\AgdaSymbol{)}\AgdaSpace{}%
\AgdaBound{T}\AgdaSpace{}%
\AgdaSymbol{→}\AgdaSpace{}%
\AgdaDatatype{Expr}\AgdaSpace{}%
\AgdaBound{Γ}\AgdaSpace{}%
\AgdaSymbol{(}\AgdaBound{S}\AgdaSpace{}%
\AgdaOperator{\AgdaInductiveConstructor{⇒}}\AgdaSpace{}%
\AgdaBound{T}\AgdaSymbol{)}\<%
\\
%
\>[2]\AgdaInductiveConstructor{app}\AgdaSpace{}%
\AgdaSymbol{:}\AgdaSpace{}%
\AgdaSymbol{∀}\AgdaSpace{}%
\AgdaSymbol{\{}\AgdaBound{S}\AgdaSpace{}%
\AgdaBound{T}\AgdaSymbol{\}}\AgdaSpace{}%
\AgdaSymbol{→}\AgdaSpace{}%
\AgdaDatatype{Expr}\AgdaSpace{}%
\AgdaBound{Γ}\AgdaSpace{}%
\AgdaSymbol{(}\AgdaBound{S}\AgdaSpace{}%
\AgdaOperator{\AgdaInductiveConstructor{⇒}}\AgdaSpace{}%
\AgdaBound{T}\AgdaSymbol{)}\AgdaSpace{}%
\AgdaSymbol{→}\AgdaSpace{}%
\AgdaDatatype{Expr}\AgdaSpace{}%
\AgdaBound{Γ}\AgdaSpace{}%
\AgdaBound{S}\AgdaSpace{}%
\AgdaSymbol{→}\AgdaSpace{}%
\AgdaDatatype{Expr}\AgdaSpace{}%
\AgdaBound{Γ}\AgdaSpace{}%
\AgdaBound{T}\<%
\\
%
\\[\AgdaEmptyExtraSkip]%
\>[0]\AgdaKeyword{data}\AgdaSpace{}%
\AgdaOperator{\AgdaDatatype{𝓖⟦\AgdaUnderscore{}⟧}}\AgdaSpace{}%
\AgdaSymbol{:}\AgdaSpace{}%
\AgdaFunction{Env}\AgdaSpace{}%
\AgdaSymbol{→}\AgdaSpace{}%
\AgdaPrimitive{Set}\AgdaSpace{}%
\AgdaKeyword{where}\<%
\\
\>[0][@{}l@{\AgdaIndent{0}}]%
\>[2]\AgdaInductiveConstructor{[]}%
\>[6]\AgdaSymbol{:}\AgdaSpace{}%
\AgdaOperator{\AgdaDatatype{𝓖⟦}}\AgdaSpace{}%
\AgdaInductiveConstructor{[]}\AgdaSpace{}%
\AgdaOperator{\AgdaDatatype{⟧}}\<%
\\
%
\>[2]\AgdaOperator{\AgdaInductiveConstructor{\AgdaUnderscore{}∷\AgdaUnderscore{}}}\AgdaSpace{}%
\AgdaSymbol{:}\AgdaSpace{}%
\AgdaSymbol{∀}\AgdaSpace{}%
\AgdaSymbol{\{}\AgdaBound{T}\AgdaSpace{}%
\AgdaBound{Γ}\AgdaSymbol{\}}\AgdaSpace{}%
\AgdaSymbol{→}\AgdaSpace{}%
\AgdaOperator{\AgdaFunction{𝓣⟦}}\AgdaSpace{}%
\AgdaBound{T}\AgdaSpace{}%
\AgdaOperator{\AgdaFunction{⟧}}\AgdaSpace{}%
\AgdaSymbol{→}\AgdaSpace{}%
\AgdaOperator{\AgdaDatatype{𝓖⟦}}\AgdaSpace{}%
\AgdaBound{Γ}\AgdaSpace{}%
\AgdaOperator{\AgdaDatatype{⟧}}\AgdaSpace{}%
\AgdaSymbol{→}\AgdaSpace{}%
\AgdaOperator{\AgdaDatatype{𝓖⟦}}\AgdaSpace{}%
\AgdaBound{T}\AgdaSpace{}%
\AgdaOperator{\AgdaInductiveConstructor{∷}}\AgdaSpace{}%
\AgdaBound{Γ}\AgdaSpace{}%
\AgdaOperator{\AgdaDatatype{⟧}}\<%
\\
%
\\[\AgdaEmptyExtraSkip]%
\>[0]\AgdaFunction{lookup}\AgdaSpace{}%
\AgdaSymbol{:}\AgdaSpace{}%
\AgdaSymbol{∀}\AgdaSpace{}%
\AgdaSymbol{\{}\AgdaBound{T}\AgdaSpace{}%
\AgdaBound{Γ}\AgdaSymbol{\}}\AgdaSpace{}%
\AgdaSymbol{→}\AgdaSpace{}%
\AgdaBound{T}\AgdaSpace{}%
\AgdaOperator{\AgdaDatatype{∈}}\AgdaSpace{}%
\AgdaBound{Γ}\AgdaSpace{}%
\AgdaSymbol{→}\AgdaSpace{}%
\AgdaOperator{\AgdaDatatype{𝓖⟦}}\AgdaSpace{}%
\AgdaBound{Γ}\AgdaSpace{}%
\AgdaOperator{\AgdaDatatype{⟧}}\AgdaSpace{}%
\AgdaSymbol{→}\AgdaSpace{}%
\AgdaOperator{\AgdaFunction{𝓣⟦}}\AgdaSpace{}%
\AgdaBound{T}\AgdaSpace{}%
\AgdaOperator{\AgdaFunction{⟧}}\<%
\\
\>[0]\AgdaFunction{lookup}\AgdaSpace{}%
\AgdaInductiveConstructor{here}\AgdaSpace{}%
\AgdaSymbol{(}\AgdaBound{x}\AgdaSpace{}%
\AgdaOperator{\AgdaInductiveConstructor{∷}}\AgdaSpace{}%
\AgdaSymbol{\AgdaUnderscore{})}\AgdaSpace{}%
\AgdaSymbol{=}\AgdaSpace{}%
\AgdaBound{x}\<%
\\
\>[0]\AgdaFunction{lookup}\AgdaSpace{}%
\AgdaSymbol{(}\AgdaInductiveConstructor{there}\AgdaSpace{}%
\AgdaBound{x}\AgdaSymbol{)}\AgdaSpace{}%
\AgdaSymbol{(\AgdaUnderscore{}}\AgdaSpace{}%
\AgdaOperator{\AgdaInductiveConstructor{∷}}\AgdaSpace{}%
\AgdaBound{γ}\AgdaSymbol{)}\AgdaSpace{}%
\AgdaSymbol{=}\AgdaSpace{}%
\AgdaFunction{lookup}\AgdaSpace{}%
\AgdaBound{x}\AgdaSpace{}%
\AgdaBound{γ}\<%
\\
%
\\[\AgdaEmptyExtraSkip]%
\>[0]\AgdaOperator{\AgdaFunction{𝓔⟦\AgdaUnderscore{}⟧}}\AgdaSpace{}%
\AgdaSymbol{:}\AgdaSpace{}%
\AgdaSymbol{∀}\AgdaSpace{}%
\AgdaSymbol{\{}\AgdaBound{Γ}\AgdaSpace{}%
\AgdaBound{T}\AgdaSymbol{\}}\AgdaSpace{}%
\AgdaSymbol{→}\AgdaSpace{}%
\AgdaDatatype{Expr}\AgdaSpace{}%
\AgdaBound{Γ}\AgdaSpace{}%
\AgdaBound{T}\AgdaSpace{}%
\AgdaSymbol{→}\AgdaSpace{}%
\AgdaOperator{\AgdaDatatype{𝓖⟦}}\AgdaSpace{}%
\AgdaBound{Γ}\AgdaSpace{}%
\AgdaOperator{\AgdaDatatype{⟧}}\AgdaSpace{}%
\AgdaSymbol{→}\AgdaSpace{}%
\AgdaOperator{\AgdaFunction{𝓣⟦}}\AgdaSpace{}%
\AgdaBound{T}\AgdaSpace{}%
\AgdaOperator{\AgdaFunction{⟧}}\<%
\\
\>[0]\AgdaOperator{\AgdaFunction{𝓔⟦}}\AgdaSpace{}%
\AgdaInductiveConstructor{con}\AgdaSpace{}%
\AgdaBound{n}\AgdaSpace{}%
\AgdaOperator{\AgdaFunction{⟧}}\AgdaSpace{}%
\AgdaBound{γ}\AgdaSpace{}%
\AgdaSymbol{=}\AgdaSpace{}%
\AgdaBound{n}\<%
\\
\>[0]\AgdaOperator{\AgdaFunction{𝓔⟦}}\AgdaSpace{}%
\AgdaInductiveConstructor{var}\AgdaSpace{}%
\AgdaBound{x}\AgdaSpace{}%
\AgdaOperator{\AgdaFunction{⟧}}\AgdaSpace{}%
\AgdaBound{γ}\AgdaSpace{}%
\AgdaSymbol{=}\AgdaSpace{}%
\AgdaFunction{lookup}\AgdaSpace{}%
\AgdaBound{x}\AgdaSpace{}%
\AgdaBound{γ}\<%
\\
\>[0]\AgdaOperator{\AgdaFunction{𝓔⟦}}\AgdaSpace{}%
\AgdaInductiveConstructor{lam}\AgdaSpace{}%
\AgdaBound{e}\AgdaSpace{}%
\AgdaOperator{\AgdaFunction{⟧}}\AgdaSpace{}%
\AgdaBound{γ}\AgdaSpace{}%
\AgdaSymbol{=}\AgdaSpace{}%
\AgdaSymbol{λ}\AgdaSpace{}%
\AgdaBound{v}\AgdaSpace{}%
\AgdaSymbol{→}\AgdaSpace{}%
\AgdaOperator{\AgdaFunction{𝓔⟦}}\AgdaSpace{}%
\AgdaBound{e}\AgdaSpace{}%
\AgdaOperator{\AgdaFunction{⟧}}\AgdaSpace{}%
\AgdaSymbol{(}\AgdaBound{v}\AgdaSpace{}%
\AgdaOperator{\AgdaInductiveConstructor{∷}}\AgdaSpace{}%
\AgdaBound{γ}\AgdaSymbol{)}\<%
\\
\>[0]\AgdaOperator{\AgdaFunction{𝓔⟦}}\AgdaSpace{}%
\AgdaInductiveConstructor{app}\AgdaSpace{}%
\AgdaBound{e₁}\AgdaSpace{}%
\AgdaBound{e₂}\AgdaSpace{}%
\AgdaOperator{\AgdaFunction{⟧}}\AgdaSpace{}%
\AgdaBound{γ}\AgdaSpace{}%
\AgdaSymbol{=}\AgdaSpace{}%
\AgdaOperator{\AgdaFunction{𝓔⟦}}\AgdaSpace{}%
\AgdaBound{e₁}\AgdaSpace{}%
\AgdaOperator{\AgdaFunction{⟧}}\AgdaSpace{}%
\AgdaBound{γ}\AgdaSpace{}%
\AgdaSymbol{(}\AgdaOperator{\AgdaFunction{𝓔⟦}}\AgdaSpace{}%
\AgdaBound{e₂}\AgdaSpace{}%
\AgdaOperator{\AgdaFunction{⟧}}\AgdaSpace{}%
\AgdaBound{γ}\AgdaSymbol{)}\<%
\end{code}
\begin{code}[hide]%
\>[0]\<%
\\
\>[0]\AgdaFunction{\AgdaUnderscore{}}\AgdaSpace{}%
\AgdaSymbol{:}\AgdaSpace{}%
\AgdaDatatype{Expr}\AgdaSpace{}%
\AgdaInductiveConstructor{[]}\AgdaSpace{}%
\AgdaSymbol{(}\AgdaInductiveConstructor{nat}\AgdaSpace{}%
\AgdaOperator{\AgdaInductiveConstructor{⇒}}\AgdaSpace{}%
\AgdaInductiveConstructor{nat}\AgdaSymbol{)}\<%
\\
\>[0]\AgdaSymbol{\AgdaUnderscore{}}\AgdaSpace{}%
\AgdaSymbol{=}\AgdaSpace{}%
\AgdaInductiveConstructor{lam}\AgdaSpace{}%
\AgdaSymbol{(}\AgdaInductiveConstructor{con}\AgdaSpace{}%
\AgdaInductiveConstructor{zero}\AgdaSymbol{)}\<%
\\
%
\\[\AgdaEmptyExtraSkip]%
\>[0]\AgdaFunction{\AgdaUnderscore{}}\AgdaSpace{}%
\AgdaSymbol{:}\AgdaSpace{}%
\AgdaDatatype{Expr}\AgdaSpace{}%
\AgdaInductiveConstructor{[]}\AgdaSpace{}%
\AgdaSymbol{(}\AgdaInductiveConstructor{nat}\AgdaSpace{}%
\AgdaOperator{\AgdaInductiveConstructor{⇒}}\AgdaSpace{}%
\AgdaInductiveConstructor{nat}\AgdaSymbol{)}\<%
\\
\>[0]\AgdaSymbol{\AgdaUnderscore{}}\AgdaSpace{}%
\AgdaSymbol{=}\AgdaSpace{}%
\AgdaInductiveConstructor{lam}\AgdaSpace{}%
\AgdaSymbol{(}\AgdaInductiveConstructor{var}\AgdaSpace{}%
\AgdaInductiveConstructor{here}\AgdaSymbol{)}\<%
\end{code}

  \caption{Simply typed lambda calculus, denotationally}
  \label{fig:stlc}
\end{figure}
We define intrinsically typed syntax of expressions as an inductive datatype
parameterized over a typing environment and indexed on the return
type. For variables, we use de Bruin indices into the typing
environment.

The semantics of a typing environment is a run-time environment in the
form of a heterogenous list of suitably typed values. With all that,
we can define the semantics of an expression ... as a function from
the semantics of a typing environment ... to the semantics of the type
... Clearly this definition also serves as a tagless interpreter for
the simply-typed lambda calculus, which means that type preservation
is also built into its definition. Moreover, as Agda accepts this
definition as terminating, we know that evaluation of every
simply-typed lambda term terminates; a non-trivial semantic property
of the simply-typed lambda calculus.

Agda-encodings of intrinsically-typed interpreters have been explored
quite a lot, but rarely in the context of polymorphic source
languages. One possible reason is that the archetypical polymorphic
lambda calculus, System~F, cannot be embedded in Agda because of its
impredicativity. This begs the question if we can develop a tagless
interpreter for a predicative version of System~F in Agda.

We answer this question affirmatively for Leivant's stratified version
of the polymorphic lambda calculus \cite{DBLP:journals/iandc/Leivant91}. The key idea of his
calculus is to stratify the set of polymorphic types in levels such
that universal quantification only ranges over strictly smaller
levels. This restriction literally embodies predicativity and, as we
will discover, the stratification corresponds directly to Agda's
universe stratification.

{[Apologies for the scarcity of references in this draft. We're aware
  of much more work using intrinsically typed interpreters, but we're
  running out of time. Any pointers are appreciated.]}

% As the construction is more involved than the one for simply-typed
% lambda calculus, we only give the key definitions and discuss some of
% the problems arising on the way to tagless interpretation. 

\begin{code}[hide]%
\>[0]\AgdaKeyword{module}\AgdaSpace{}%
\AgdaModule{Tagless-final}\AgdaSpace{}%
\AgdaKeyword{where}\<%
\\
%
\\[\AgdaEmptyExtraSkip]%
\>[0]\AgdaKeyword{open}\AgdaSpace{}%
\AgdaKeyword{import}\AgdaSpace{}%
\AgdaModule{Level}\<%
\\
\>[0]\AgdaKeyword{open}\AgdaSpace{}%
\AgdaKeyword{import}\AgdaSpace{}%
\AgdaModule{Data.Fin}\AgdaSpace{}%
\AgdaKeyword{using}\AgdaSpace{}%
\AgdaSymbol{(}\AgdaDatatype{Fin}\AgdaSymbol{)}\AgdaSpace{}%
\AgdaKeyword{renaming}\AgdaSpace{}%
\AgdaSymbol{(}\AgdaInductiveConstructor{zero}\AgdaSpace{}%
\AgdaSymbol{to}\AgdaSpace{}%
\AgdaInductiveConstructor{fzero}\AgdaSymbol{;}\AgdaSpace{}%
\AgdaInductiveConstructor{suc}\AgdaSpace{}%
\AgdaSymbol{to}\AgdaSpace{}%
\AgdaInductiveConstructor{fsuc}\AgdaSymbol{)}\<%
\\
\>[0]\AgdaKeyword{open}\AgdaSpace{}%
\AgdaKeyword{import}\AgdaSpace{}%
\AgdaModule{Data.Nat}\AgdaSpace{}%
\AgdaKeyword{using}\AgdaSpace{}%
\AgdaSymbol{(}\AgdaDatatype{ℕ}\AgdaSymbol{)}\<%
\\
\>[0]\AgdaKeyword{open}\AgdaSpace{}%
\AgdaKeyword{import}\AgdaSpace{}%
\AgdaModule{Data.List}\AgdaSpace{}%
\AgdaKeyword{using}\AgdaSpace{}%
\AgdaSymbol{(}\AgdaDatatype{List}\AgdaSymbol{;}\AgdaSpace{}%
\AgdaInductiveConstructor{[]}\AgdaSymbol{;}\AgdaSpace{}%
\AgdaOperator{\AgdaInductiveConstructor{\AgdaUnderscore{}∷\AgdaUnderscore{}}}\AgdaSymbol{;}\AgdaSpace{}%
\AgdaOperator{\AgdaFunction{\AgdaUnderscore{}++\AgdaUnderscore{}}}\AgdaSymbol{;}\AgdaSpace{}%
\AgdaFunction{length}\AgdaSymbol{;}\AgdaSpace{}%
\AgdaFunction{lookup}\AgdaSymbol{;}\AgdaSpace{}%
\AgdaFunction{tabulate}\AgdaSymbol{)}\<%
\\
\>[0]\AgdaKeyword{open}\AgdaSpace{}%
\AgdaKeyword{import}\AgdaSpace{}%
\AgdaModule{Data.Unit}\<%
\\
\>[0]\AgdaKeyword{open}\AgdaSpace{}%
\AgdaKeyword{import}\AgdaSpace{}%
\AgdaModule{Function}\AgdaSpace{}%
\AgdaKeyword{using}\AgdaSpace{}%
\AgdaSymbol{(}\AgdaOperator{\AgdaFunction{\AgdaUnderscore{}∘\AgdaUnderscore{}}}\AgdaSymbol{;}\AgdaSpace{}%
\AgdaFunction{id}\AgdaSymbol{)}\<%
\\
\>[0]\AgdaKeyword{open}\AgdaSpace{}%
\AgdaKeyword{import}\AgdaSpace{}%
\AgdaModule{Relation.Binary.PropositionalEquality}\<%
\\
\>[0][@{}l@{\AgdaIndent{0}}]%
\>[2]\AgdaKeyword{using}\AgdaSpace{}%
\AgdaSymbol{(}\AgdaOperator{\AgdaDatatype{\AgdaUnderscore{}≡\AgdaUnderscore{}}}\AgdaSymbol{;}\AgdaSpace{}%
\AgdaOperator{\AgdaFunction{\AgdaUnderscore{}≢\AgdaUnderscore{}}}\AgdaSymbol{;}\AgdaSpace{}%
\AgdaInductiveConstructor{refl}\AgdaSymbol{;}\AgdaSpace{}%
\AgdaFunction{sym}\AgdaSymbol{;}\AgdaSpace{}%
\AgdaFunction{trans}\AgdaSymbol{;}\AgdaSpace{}%
\AgdaFunction{cong}\AgdaSymbol{;}\AgdaSpace{}%
\AgdaFunction{cong₂}\AgdaSymbol{;}\AgdaSpace{}%
\AgdaFunction{subst}\AgdaSymbol{;}\AgdaSpace{}%
\AgdaFunction{resp₂}\AgdaSymbol{)}\<%
\\
%
\\[\AgdaEmptyExtraSkip]%
\>[0]\AgdaComment{----------------------------------------------------------------------}\<%
\\
\>[0]\AgdaKeyword{postulate}\<%
\end{code}
\newcommand\TFDependentExt{%
\begin{code}%
\>[0][@{}l@{\AgdaIndent{1}}]%
\>[2]\AgdaPostulate{∀-extensionality}\AgdaSpace{}%
\AgdaSymbol{:}\<%
\\
\>[2][@{}l@{\AgdaIndent{0}}]%
\>[4]\AgdaSymbol{∀}\AgdaSpace{}%
\AgdaSymbol{\{}\AgdaBound{a}\AgdaSpace{}%
\AgdaBound{b}\AgdaSymbol{\}\{}\AgdaBound{A}\AgdaSpace{}%
\AgdaSymbol{:}\AgdaSpace{}%
\AgdaPrimitive{Set}\AgdaSpace{}%
\AgdaBound{a}\AgdaSymbol{\}\{}\AgdaBound{F}\AgdaSpace{}%
\AgdaBound{G}\AgdaSpace{}%
\AgdaSymbol{:}\AgdaSpace{}%
\AgdaSymbol{(}\AgdaBound{α}\AgdaSpace{}%
\AgdaSymbol{:}\AgdaSpace{}%
\AgdaBound{A}\AgdaSymbol{)}\AgdaSpace{}%
\AgdaSymbol{→}\AgdaSpace{}%
\AgdaPrimitive{Set}\AgdaSpace{}%
\AgdaBound{b}\AgdaSymbol{\}}\<%
\\
%
\>[4]\AgdaSymbol{→}\AgdaSpace{}%
\AgdaSymbol{(∀}\AgdaSpace{}%
\AgdaSymbol{(}\AgdaBound{α}\AgdaSpace{}%
\AgdaSymbol{:}\AgdaSpace{}%
\AgdaBound{A}\AgdaSymbol{)}\AgdaSpace{}%
\AgdaSymbol{→}\AgdaSpace{}%
\AgdaBound{F}\AgdaSpace{}%
\AgdaBound{α}\AgdaSpace{}%
\AgdaOperator{\AgdaDatatype{≡}}\AgdaSpace{}%
\AgdaBound{G}\AgdaSpace{}%
\AgdaBound{α}\AgdaSymbol{)}\<%
\\
%
\>[4]\AgdaSymbol{→}\AgdaSpace{}%
\AgdaSymbol{((}\AgdaBound{α}\AgdaSpace{}%
\AgdaSymbol{:}\AgdaSpace{}%
\AgdaBound{A}\AgdaSymbol{)}\AgdaSpace{}%
\AgdaSymbol{→}\AgdaSpace{}%
\AgdaBound{F}\AgdaSpace{}%
\AgdaBound{α}\AgdaSymbol{)}\AgdaSpace{}%
\AgdaOperator{\AgdaDatatype{≡}}\AgdaSpace{}%
\AgdaSymbol{((}\AgdaBound{α}\AgdaSpace{}%
\AgdaSymbol{:}\AgdaSpace{}%
\AgdaBound{A}\AgdaSymbol{)}\AgdaSpace{}%
\AgdaSymbol{→}\AgdaSpace{}%
\AgdaBound{G}\AgdaSpace{}%
\AgdaBound{α}\AgdaSymbol{)}\<%
\end{code}}
\begin{code}[hide]%
\>[0]\AgdaComment{--\ equality\ involving\ Setω}\<%
\\
%
\\[\AgdaEmptyExtraSkip]%
\>[0]\AgdaKeyword{data}\AgdaSpace{}%
\AgdaOperator{\AgdaDatatype{\AgdaUnderscore{}≡ω\AgdaUnderscore{}}}\AgdaSpace{}%
\AgdaSymbol{\{}\AgdaBound{A}\AgdaSpace{}%
\AgdaSymbol{:}\AgdaSpace{}%
\AgdaPrimitive{Setω}\AgdaSymbol{\}}\AgdaSpace{}%
\AgdaSymbol{(}\AgdaBound{x}\AgdaSpace{}%
\AgdaSymbol{:}\AgdaSpace{}%
\AgdaBound{A}\AgdaSymbol{)}\AgdaSpace{}%
\AgdaSymbol{:}\AgdaSpace{}%
\AgdaBound{A}\AgdaSpace{}%
\AgdaSymbol{→}\AgdaSpace{}%
\AgdaPrimitive{Setω}\AgdaSpace{}%
\AgdaKeyword{where}\<%
\\
\>[0][@{}l@{\AgdaIndent{0}}]%
\>[2]\AgdaInductiveConstructor{refl}\AgdaSpace{}%
\AgdaSymbol{:}\AgdaSpace{}%
\AgdaBound{x}\AgdaSpace{}%
\AgdaOperator{\AgdaDatatype{≡ω}}\AgdaSpace{}%
\AgdaBound{x}\<%
\\
%
\\[\AgdaEmptyExtraSkip]%
\>[0]\AgdaFunction{congωl}\AgdaSpace{}%
\AgdaSymbol{:}\AgdaSpace{}%
\AgdaSymbol{∀}\AgdaSpace{}%
\AgdaSymbol{\{}\AgdaBound{b}\AgdaSymbol{\}}\AgdaSpace{}%
\AgdaSymbol{\{}\AgdaBound{A}\AgdaSpace{}%
\AgdaSymbol{:}\AgdaSpace{}%
\AgdaPrimitive{Setω}\AgdaSymbol{\}}\AgdaSpace{}%
\AgdaSymbol{\{}\AgdaBound{B}\AgdaSpace{}%
\AgdaSymbol{:}\AgdaSpace{}%
\AgdaPrimitive{Set}\AgdaSpace{}%
\AgdaBound{b}\AgdaSymbol{\}}\AgdaSpace{}%
\AgdaSymbol{(}\AgdaBound{f}\AgdaSpace{}%
\AgdaSymbol{:}\AgdaSpace{}%
\AgdaBound{A}\AgdaSpace{}%
\AgdaSymbol{→}\AgdaSpace{}%
\AgdaBound{B}\AgdaSymbol{)}\AgdaSpace{}%
\AgdaSymbol{\{}\AgdaBound{x}\AgdaSpace{}%
\AgdaBound{y}\AgdaSpace{}%
\AgdaSymbol{:}\AgdaSpace{}%
\AgdaBound{A}\AgdaSymbol{\}}\AgdaSpace{}%
\AgdaSymbol{→}\AgdaSpace{}%
\AgdaBound{x}\AgdaSpace{}%
\AgdaOperator{\AgdaDatatype{≡ω}}\AgdaSpace{}%
\AgdaBound{y}\AgdaSpace{}%
\AgdaSymbol{→}\AgdaSpace{}%
\AgdaBound{f}\AgdaSpace{}%
\AgdaBound{x}\AgdaSpace{}%
\AgdaOperator{\AgdaDatatype{≡}}\AgdaSpace{}%
\AgdaBound{f}\AgdaSpace{}%
\AgdaBound{y}\<%
\\
\>[0]\AgdaFunction{congωl}\AgdaSpace{}%
\AgdaBound{f}\AgdaSpace{}%
\AgdaInductiveConstructor{refl}\AgdaSpace{}%
\AgdaSymbol{=}\AgdaSpace{}%
\AgdaInductiveConstructor{refl}\<%
\\
%
\\[\AgdaEmptyExtraSkip]%
\>[0]\AgdaFunction{conglω}\AgdaSpace{}%
\AgdaSymbol{:}\AgdaSpace{}%
\AgdaSymbol{∀}\AgdaSpace{}%
\AgdaSymbol{\{}\AgdaBound{a}\AgdaSymbol{\}}\AgdaSpace{}%
\AgdaSymbol{\{}\AgdaBound{A}\AgdaSpace{}%
\AgdaSymbol{:}\AgdaSpace{}%
\AgdaPrimitive{Set}\AgdaSpace{}%
\AgdaBound{a}\AgdaSymbol{\}}\AgdaSpace{}%
\AgdaSymbol{\{}\AgdaBound{B}\AgdaSpace{}%
\AgdaSymbol{:}\AgdaSpace{}%
\AgdaPrimitive{Setω}\AgdaSymbol{\}}\AgdaSpace{}%
\AgdaSymbol{(}\AgdaBound{f}\AgdaSpace{}%
\AgdaSymbol{:}\AgdaSpace{}%
\AgdaBound{A}\AgdaSpace{}%
\AgdaSymbol{→}\AgdaSpace{}%
\AgdaBound{B}\AgdaSymbol{)}\AgdaSpace{}%
\AgdaSymbol{\{}\AgdaBound{x}\AgdaSpace{}%
\AgdaBound{y}\AgdaSpace{}%
\AgdaSymbol{:}\AgdaSpace{}%
\AgdaBound{A}\AgdaSymbol{\}}\AgdaSpace{}%
\AgdaSymbol{→}\AgdaSpace{}%
\AgdaBound{x}\AgdaSpace{}%
\AgdaOperator{\AgdaDatatype{≡}}\AgdaSpace{}%
\AgdaBound{y}\AgdaSpace{}%
\AgdaSymbol{→}\AgdaSpace{}%
\AgdaBound{f}\AgdaSpace{}%
\AgdaBound{x}\AgdaSpace{}%
\AgdaOperator{\AgdaDatatype{≡ω}}\AgdaSpace{}%
\AgdaBound{f}\AgdaSpace{}%
\AgdaBound{y}\<%
\\
\>[0]\AgdaFunction{conglω}\AgdaSpace{}%
\AgdaBound{f}\AgdaSpace{}%
\AgdaInductiveConstructor{refl}\AgdaSpace{}%
\AgdaSymbol{=}\AgdaSpace{}%
\AgdaInductiveConstructor{refl}\<%
\\
%
\\[\AgdaEmptyExtraSkip]%
\>[0]\AgdaFunction{congωω}\AgdaSpace{}%
\AgdaSymbol{:}\AgdaSpace{}%
\AgdaSymbol{∀}\AgdaSpace{}%
\AgdaSymbol{\{}\AgdaBound{A}\AgdaSpace{}%
\AgdaSymbol{:}\AgdaSpace{}%
\AgdaPrimitive{Setω}\AgdaSymbol{\}}\AgdaSpace{}%
\AgdaSymbol{\{}\AgdaBound{B}\AgdaSpace{}%
\AgdaSymbol{:}\AgdaSpace{}%
\AgdaPrimitive{Setω}\AgdaSymbol{\}}\AgdaSpace{}%
\AgdaSymbol{(}\AgdaBound{f}\AgdaSpace{}%
\AgdaSymbol{:}\AgdaSpace{}%
\AgdaBound{A}\AgdaSpace{}%
\AgdaSymbol{→}\AgdaSpace{}%
\AgdaBound{B}\AgdaSymbol{)}\AgdaSpace{}%
\AgdaSymbol{\{}\AgdaBound{x}\AgdaSpace{}%
\AgdaBound{y}\AgdaSpace{}%
\AgdaSymbol{:}\AgdaSpace{}%
\AgdaBound{A}\AgdaSymbol{\}}\AgdaSpace{}%
\AgdaSymbol{→}\AgdaSpace{}%
\AgdaBound{x}\AgdaSpace{}%
\AgdaOperator{\AgdaDatatype{≡ω}}\AgdaSpace{}%
\AgdaBound{y}\AgdaSpace{}%
\AgdaSymbol{→}\AgdaSpace{}%
\AgdaBound{f}\AgdaSpace{}%
\AgdaBound{x}\AgdaSpace{}%
\AgdaOperator{\AgdaDatatype{≡ω}}\AgdaSpace{}%
\AgdaBound{f}\AgdaSpace{}%
\AgdaBound{y}\<%
\\
\>[0]\AgdaFunction{congωω}\AgdaSpace{}%
\AgdaBound{f}\AgdaSpace{}%
\AgdaInductiveConstructor{refl}\AgdaSpace{}%
\AgdaSymbol{=}\AgdaSpace{}%
\AgdaInductiveConstructor{refl}\<%
\\
%
\\[\AgdaEmptyExtraSkip]%
\>[0]\AgdaFunction{transω}\AgdaSpace{}%
\AgdaSymbol{:}\AgdaSpace{}%
\AgdaSymbol{∀}\AgdaSpace{}%
\AgdaSymbol{\{}\AgdaBound{A}\AgdaSpace{}%
\AgdaSymbol{:}\AgdaSpace{}%
\AgdaPrimitive{Setω}\AgdaSymbol{\}}\AgdaSpace{}%
\AgdaSymbol{\{}\AgdaBound{x}\AgdaSpace{}%
\AgdaBound{y}\AgdaSpace{}%
\AgdaBound{z}\AgdaSpace{}%
\AgdaSymbol{:}\AgdaSpace{}%
\AgdaBound{A}\AgdaSymbol{\}}\AgdaSpace{}%
\AgdaSymbol{→}\AgdaSpace{}%
\AgdaBound{x}\AgdaSpace{}%
\AgdaOperator{\AgdaDatatype{≡ω}}\AgdaSpace{}%
\AgdaBound{y}\AgdaSpace{}%
\AgdaSymbol{→}\AgdaSpace{}%
\AgdaBound{y}\AgdaSpace{}%
\AgdaOperator{\AgdaDatatype{≡ω}}\AgdaSpace{}%
\AgdaBound{z}\AgdaSpace{}%
\AgdaSymbol{→}\AgdaSpace{}%
\AgdaBound{x}\AgdaSpace{}%
\AgdaOperator{\AgdaDatatype{≡ω}}\AgdaSpace{}%
\AgdaBound{z}\<%
\\
\>[0]\AgdaFunction{transω}\AgdaSpace{}%
\AgdaInductiveConstructor{refl}\AgdaSpace{}%
\AgdaInductiveConstructor{refl}\AgdaSpace{}%
\AgdaSymbol{=}\AgdaSpace{}%
\AgdaInductiveConstructor{refl}\<%
\\
%
\\[\AgdaEmptyExtraSkip]%
\>[0]\AgdaComment{----------------------------------------------------------------------}\<%
\\
%
\\[\AgdaEmptyExtraSkip]%
\>[0]\AgdaComment{--\ variables\ in\ a\ list}\<%
\\
%
\\[\AgdaEmptyExtraSkip]%
\>[0]\AgdaKeyword{data}\AgdaSpace{}%
\AgdaOperator{\AgdaDatatype{\AgdaUnderscore{}∈\AgdaUnderscore{}}}\AgdaSpace{}%
\AgdaSymbol{\{}\AgdaBound{A}\AgdaSpace{}%
\AgdaSymbol{:}\AgdaSpace{}%
\AgdaPrimitive{Set}\AgdaSymbol{\}}\AgdaSpace{}%
\AgdaSymbol{(}\AgdaBound{a}\AgdaSpace{}%
\AgdaSymbol{:}\AgdaSpace{}%
\AgdaBound{A}\AgdaSymbol{)}\AgdaSpace{}%
\AgdaSymbol{:}\AgdaSpace{}%
\AgdaDatatype{List}\AgdaSpace{}%
\AgdaBound{A}\AgdaSpace{}%
\AgdaSymbol{→}\AgdaSpace{}%
\AgdaPrimitive{Set}\AgdaSpace{}%
\AgdaKeyword{where}\<%
\\
\>[0][@{}l@{\AgdaIndent{0}}]%
\>[2]\AgdaInductiveConstructor{here}%
\>[8]\AgdaSymbol{:}\AgdaSpace{}%
\AgdaSymbol{∀\{}\AgdaBound{Δ}\AgdaSymbol{\}}\AgdaSpace{}%
\AgdaSymbol{→}\AgdaSpace{}%
\AgdaBound{a}\AgdaSpace{}%
\AgdaOperator{\AgdaDatatype{∈}}\AgdaSpace{}%
\AgdaSymbol{(}\AgdaBound{a}\AgdaSpace{}%
\AgdaOperator{\AgdaInductiveConstructor{∷}}\AgdaSpace{}%
\AgdaBound{Δ}\AgdaSymbol{)}\<%
\\
%
\>[2]\AgdaInductiveConstructor{there}\AgdaSpace{}%
\AgdaSymbol{:}\AgdaSpace{}%
\AgdaSymbol{∀\{}\AgdaBound{a′}\AgdaSpace{}%
\AgdaBound{Δ}\AgdaSymbol{\}}\AgdaSpace{}%
\AgdaSymbol{→}\AgdaSpace{}%
\AgdaBound{a}\AgdaSpace{}%
\AgdaOperator{\AgdaDatatype{∈}}\AgdaSpace{}%
\AgdaBound{Δ}\AgdaSpace{}%
\AgdaSymbol{→}\AgdaSpace{}%
\AgdaBound{a}\AgdaSpace{}%
\AgdaOperator{\AgdaDatatype{∈}}\AgdaSpace{}%
\AgdaSymbol{(}\AgdaBound{a′}\AgdaSpace{}%
\AgdaOperator{\AgdaInductiveConstructor{∷}}\AgdaSpace{}%
\AgdaBound{Δ}\AgdaSymbol{)}\<%
\\
%
\\[\AgdaEmptyExtraSkip]%
\>[0]\AgdaComment{--\ level\ environments}\<%
\\
%
\\[\AgdaEmptyExtraSkip]%
\>[0]\AgdaKeyword{variable}\AgdaSpace{}%
\AgdaGeneralizable{Δ}\AgdaSpace{}%
\AgdaGeneralizable{Δ₁}\AgdaSpace{}%
\AgdaGeneralizable{Δ₂}\AgdaSpace{}%
\AgdaSymbol{:}\AgdaSpace{}%
\AgdaDatatype{List}\AgdaSpace{}%
\AgdaPostulate{Level}\<%
\\
\>[0]\AgdaKeyword{variable}\AgdaSpace{}%
\AgdaGeneralizable{l}\AgdaSpace{}%
\AgdaGeneralizable{l′}\AgdaSpace{}%
\AgdaSymbol{:}\AgdaSpace{}%
\AgdaPostulate{Level}\<%
\\
%
\\[\AgdaEmptyExtraSkip]%
\>[0]\AgdaComment{--\ types}\<%
\end{code}
\newcommand\TFType{%
\begin{code}%
\>[0]\AgdaFunction{LEnv}\AgdaSpace{}%
\AgdaSymbol{=}\AgdaSpace{}%
\AgdaDatatype{List}\AgdaSpace{}%
\AgdaPostulate{Level}\<%
\\
\>[0]\AgdaKeyword{data}\AgdaSpace{}%
\AgdaDatatype{Type}\AgdaSpace{}%
\AgdaSymbol{(}\AgdaBound{Δ}\AgdaSpace{}%
\AgdaSymbol{:}\AgdaSpace{}%
\AgdaFunction{LEnv}\AgdaSymbol{)}\AgdaSpace{}%
\AgdaSymbol{:}\AgdaSpace{}%
\AgdaPostulate{Level}\AgdaSpace{}%
\AgdaSymbol{→}\AgdaSpace{}%
\AgdaPrimitive{Set}\AgdaSpace{}%
\AgdaKeyword{where}\<%
\\
\>[0][@{}l@{\AgdaIndent{0}}]%
\>[2]\AgdaInductiveConstructor{nat}\AgdaSpace{}%
\AgdaSymbol{:}\AgdaSpace{}%
\AgdaDatatype{Type}\AgdaSpace{}%
\AgdaBound{Δ}\AgdaSpace{}%
\AgdaPrimitive{zero}\<%
\\
%
\>[2]\AgdaOperator{\AgdaInductiveConstructor{`\AgdaUnderscore{}}}%
\>[6]\AgdaSymbol{:}\AgdaSpace{}%
\AgdaGeneralizable{l}\AgdaSpace{}%
\AgdaOperator{\AgdaDatatype{∈}}\AgdaSpace{}%
\AgdaBound{Δ}\AgdaSpace{}%
\AgdaSymbol{→}\AgdaSpace{}%
\AgdaDatatype{Type}\AgdaSpace{}%
\AgdaBound{Δ}\AgdaSpace{}%
\AgdaGeneralizable{l}\<%
\\
%
\>[2]\AgdaOperator{\AgdaInductiveConstructor{\AgdaUnderscore{}⇒\AgdaUnderscore{}}}\AgdaSpace{}%
\AgdaSymbol{:}\AgdaSpace{}%
\AgdaDatatype{Type}\AgdaSpace{}%
\AgdaBound{Δ}\AgdaSpace{}%
\AgdaGeneralizable{l}\AgdaSpace{}%
\AgdaSymbol{→}\AgdaSpace{}%
\AgdaDatatype{Type}\AgdaSpace{}%
\AgdaBound{Δ}\AgdaSpace{}%
\AgdaGeneralizable{l′}\AgdaSpace{}%
\AgdaSymbol{→}\AgdaSpace{}%
\AgdaDatatype{Type}\AgdaSpace{}%
\AgdaBound{Δ}\AgdaSpace{}%
\AgdaSymbol{(}\AgdaGeneralizable{l}\AgdaSpace{}%
\AgdaOperator{\AgdaPrimitive{⊔}}\AgdaSpace{}%
\AgdaGeneralizable{l′}\AgdaSymbol{)}\<%
\\
%
\>[2]\AgdaInductiveConstructor{`∀}%
\>[6]\AgdaSymbol{:}\AgdaSpace{}%
\AgdaSymbol{∀}\AgdaSpace{}%
\AgdaBound{l}\AgdaSpace{}%
\AgdaSymbol{→}\AgdaSpace{}%
\AgdaDatatype{Type}\AgdaSpace{}%
\AgdaSymbol{(}\AgdaBound{l}\AgdaSpace{}%
\AgdaOperator{\AgdaInductiveConstructor{∷}}\AgdaSpace{}%
\AgdaBound{Δ}\AgdaSymbol{)}\AgdaSpace{}%
\AgdaGeneralizable{l′}\AgdaSpace{}%
\AgdaSymbol{→}\AgdaSpace{}%
\AgdaDatatype{Type}\AgdaSpace{}%
\AgdaBound{Δ}\AgdaSpace{}%
\AgdaSymbol{(}\AgdaPrimitive{suc}\AgdaSpace{}%
\AgdaBound{l}\AgdaSpace{}%
\AgdaOperator{\AgdaPrimitive{⊔}}\AgdaSpace{}%
\AgdaGeneralizable{l′}\AgdaSymbol{)}\<%
\end{code}}
\begin{code}[hide]%
\>[0]\AgdaComment{--\ level\ of\ type\ according\ to\ Leivant'91}\<%
\\
\>[0]\AgdaFunction{level}\AgdaSpace{}%
\AgdaSymbol{:}\AgdaSpace{}%
\AgdaDatatype{Type}\AgdaSpace{}%
\AgdaGeneralizable{Δ}\AgdaSpace{}%
\AgdaGeneralizable{l}\AgdaSpace{}%
\AgdaSymbol{→}\AgdaSpace{}%
\AgdaPostulate{Level}\<%
\\
\>[0]\AgdaFunction{level}\AgdaSpace{}%
\AgdaSymbol{\{}\AgdaArgument{l}\AgdaSpace{}%
\AgdaSymbol{=}\AgdaSpace{}%
\AgdaBound{l}\AgdaSymbol{\}}\AgdaSpace{}%
\AgdaBound{T}\AgdaSpace{}%
\AgdaSymbol{=}\AgdaSpace{}%
\AgdaBound{l}\<%
\\
%
\\[\AgdaEmptyExtraSkip]%
\>[0]\AgdaComment{--\ semantic\ environments\ (mapping\ level\ l\ to\ an\ element\ of\ Set\ l)}\<%
\end{code}
\newcommand\TFTEnv{%
\begin{code}%
\>[0]\AgdaKeyword{data}\AgdaSpace{}%
\AgdaDatatype{DEnv}\AgdaSpace{}%
\AgdaSymbol{:}\AgdaSpace{}%
\AgdaFunction{LEnv}\AgdaSpace{}%
\AgdaSymbol{→}\AgdaSpace{}%
\AgdaPrimitive{Setω}\AgdaSpace{}%
\AgdaKeyword{where}\<%
\\
\>[0][@{}l@{\AgdaIndent{0}}]%
\>[2]\AgdaInductiveConstructor{[]}%
\>[6]\AgdaSymbol{:}\AgdaSpace{}%
\AgdaDatatype{DEnv}\AgdaSpace{}%
\AgdaInductiveConstructor{[]}\<%
\\
%
\>[2]\AgdaOperator{\AgdaInductiveConstructor{\AgdaUnderscore{}∷\AgdaUnderscore{}}}\AgdaSpace{}%
\AgdaSymbol{:}\AgdaSpace{}%
\AgdaPrimitive{Set}\AgdaSpace{}%
\AgdaGeneralizable{l}\AgdaSpace{}%
\AgdaSymbol{→}\AgdaSpace{}%
\AgdaDatatype{DEnv}\AgdaSpace{}%
\AgdaGeneralizable{Δ}\AgdaSpace{}%
\AgdaSymbol{→}\AgdaSpace{}%
\AgdaDatatype{DEnv}\AgdaSpace{}%
\AgdaSymbol{(}\AgdaGeneralizable{l}\AgdaSpace{}%
\AgdaOperator{\AgdaInductiveConstructor{∷}}\AgdaSpace{}%
\AgdaGeneralizable{Δ}\AgdaSymbol{)}\<%
\end{code}}
\begin{code}[hide]%
\>[0]\AgdaFunction{apply-env}\AgdaSpace{}%
\AgdaSymbol{:}\AgdaSpace{}%
\AgdaDatatype{DEnv}\AgdaSpace{}%
\AgdaGeneralizable{Δ}\AgdaSpace{}%
\AgdaSymbol{→}\AgdaSpace{}%
\AgdaGeneralizable{l}\AgdaSpace{}%
\AgdaOperator{\AgdaDatatype{∈}}\AgdaSpace{}%
\AgdaGeneralizable{Δ}\AgdaSpace{}%
\AgdaSymbol{→}\AgdaSpace{}%
\AgdaPrimitive{Set}\AgdaSpace{}%
\AgdaGeneralizable{l}\<%
\\
\>[0]\AgdaFunction{apply-env}\AgdaSpace{}%
\AgdaInductiveConstructor{[]}\AgdaSpace{}%
\AgdaSymbol{()}\<%
\\
\>[0]\AgdaFunction{apply-env}\AgdaSpace{}%
\AgdaSymbol{(}\AgdaBound{x}\AgdaSpace{}%
\AgdaOperator{\AgdaInductiveConstructor{∷}}\AgdaSpace{}%
\AgdaSymbol{\AgdaUnderscore{})}\AgdaSpace{}%
\AgdaInductiveConstructor{here}\AgdaSpace{}%
\AgdaSymbol{=}\AgdaSpace{}%
\AgdaBound{x}\<%
\\
\>[0]\AgdaFunction{apply-env}\AgdaSpace{}%
\AgdaSymbol{(\AgdaUnderscore{}}\AgdaSpace{}%
\AgdaOperator{\AgdaInductiveConstructor{∷}}\AgdaSpace{}%
\AgdaBound{η}\AgdaSymbol{)}\AgdaSpace{}%
\AgdaSymbol{(}\AgdaInductiveConstructor{there}\AgdaSpace{}%
\AgdaBound{x}\AgdaSymbol{)}\AgdaSpace{}%
\AgdaSymbol{=}\AgdaSpace{}%
\AgdaFunction{apply-env}\AgdaSpace{}%
\AgdaBound{η}\AgdaSpace{}%
\AgdaBound{x}\<%
\\
%
\\[\AgdaEmptyExtraSkip]%
\>[0]\AgdaComment{--\ the\ meaning\ of\ a\ stratified\ type\ in\ terms\ of\ Agda\ universes}\<%
\end{code}
\newcommand\TFTSem{%
\begin{code}%
\>[0]\AgdaOperator{\AgdaFunction{𝓣⟦\AgdaUnderscore{}⟧}}\AgdaSpace{}%
\AgdaSymbol{:}\AgdaSpace{}%
\AgdaDatatype{Type}\AgdaSpace{}%
\AgdaGeneralizable{Δ}\AgdaSpace{}%
\AgdaGeneralizable{l}\AgdaSpace{}%
\AgdaSymbol{→}\AgdaSpace{}%
\AgdaDatatype{DEnv}\AgdaSpace{}%
\AgdaGeneralizable{Δ}\AgdaSpace{}%
\AgdaSymbol{→}\AgdaSpace{}%
\AgdaPrimitive{Set}\AgdaSpace{}%
\AgdaGeneralizable{l}\<%
\\
\>[0]\AgdaOperator{\AgdaFunction{𝓣⟦}}\AgdaSpace{}%
\AgdaInductiveConstructor{nat}\AgdaSpace{}%
\AgdaOperator{\AgdaFunction{⟧}}\AgdaSpace{}%
\AgdaBound{η}\AgdaSpace{}%
\AgdaSymbol{=}\AgdaSpace{}%
\AgdaDatatype{ℕ}\<%
\\
\>[0]\AgdaOperator{\AgdaFunction{𝓣⟦}}\AgdaSpace{}%
\AgdaOperator{\AgdaInductiveConstructor{`}}\AgdaSpace{}%
\AgdaBound{α}\AgdaSpace{}%
\AgdaOperator{\AgdaFunction{⟧}}\AgdaSpace{}%
\AgdaBound{η}\AgdaSpace{}%
\AgdaSymbol{=}\AgdaSpace{}%
\AgdaFunction{apply-env}\AgdaSpace{}%
\AgdaBound{η}\AgdaSpace{}%
\AgdaBound{α}\<%
\\
\>[0]\AgdaOperator{\AgdaFunction{𝓣⟦}}\AgdaSpace{}%
\AgdaBound{T₁}\AgdaSpace{}%
\AgdaOperator{\AgdaInductiveConstructor{⇒}}\AgdaSpace{}%
\AgdaBound{T₂}\AgdaSpace{}%
\AgdaOperator{\AgdaFunction{⟧}}\AgdaSpace{}%
\AgdaBound{η}\AgdaSpace{}%
\AgdaSymbol{=}\AgdaSpace{}%
\AgdaOperator{\AgdaFunction{𝓣⟦}}\AgdaSpace{}%
\AgdaBound{T₁}\AgdaSpace{}%
\AgdaOperator{\AgdaFunction{⟧}}\AgdaSpace{}%
\AgdaBound{η}\AgdaSpace{}%
\AgdaSymbol{→}\AgdaSpace{}%
\AgdaOperator{\AgdaFunction{𝓣⟦}}\AgdaSpace{}%
\AgdaBound{T₂}\AgdaSpace{}%
\AgdaOperator{\AgdaFunction{⟧}}\AgdaSpace{}%
\AgdaBound{η}\<%
\\
\>[0]\AgdaOperator{\AgdaFunction{𝓣⟦}}\AgdaSpace{}%
\AgdaInductiveConstructor{`∀}\AgdaSpace{}%
\AgdaBound{l}\AgdaSpace{}%
\AgdaBound{T}\AgdaSpace{}%
\AgdaOperator{\AgdaFunction{⟧}}\AgdaSpace{}%
\AgdaBound{η}\AgdaSpace{}%
\AgdaSymbol{=}\AgdaSpace{}%
\AgdaSymbol{(}\AgdaBound{⟦α⟧}\AgdaSpace{}%
\AgdaSymbol{:}\AgdaSpace{}%
\AgdaPrimitive{Set}\AgdaSpace{}%
\AgdaBound{l}\AgdaSymbol{)}\AgdaSpace{}%
\AgdaSymbol{→}\AgdaSpace{}%
\AgdaOperator{\AgdaFunction{𝓣⟦}}\AgdaSpace{}%
\AgdaBound{T}\AgdaSpace{}%
\AgdaOperator{\AgdaFunction{⟧}}\AgdaSpace{}%
\AgdaSymbol{(}\AgdaBound{⟦α⟧}\AgdaSpace{}%
\AgdaOperator{\AgdaInductiveConstructor{∷}}\AgdaSpace{}%
\AgdaBound{η}\AgdaSymbol{)}\<%
\end{code}}
\begin{code}[hide]%
\>[0]\AgdaComment{--\ renaming\ on\ types}\<%
\\
%
\\[\AgdaEmptyExtraSkip]%
\>[0]\AgdaFunction{Ren}\AgdaSpace{}%
\AgdaSymbol{:}\AgdaSpace{}%
\AgdaFunction{LEnv}\AgdaSpace{}%
\AgdaSymbol{→}\AgdaSpace{}%
\AgdaFunction{LEnv}\AgdaSpace{}%
\AgdaSymbol{→}\AgdaSpace{}%
\AgdaPrimitive{Set}\<%
\\
\>[0]\AgdaFunction{Ren}\AgdaSpace{}%
\AgdaBound{Δ₁}\AgdaSpace{}%
\AgdaBound{Δ₂}\AgdaSpace{}%
\AgdaSymbol{=}\AgdaSpace{}%
\AgdaSymbol{∀}\AgdaSpace{}%
\AgdaSymbol{\{}\AgdaBound{l}\AgdaSymbol{\}}\AgdaSpace{}%
\AgdaSymbol{→}\AgdaSpace{}%
\AgdaBound{l}\AgdaSpace{}%
\AgdaOperator{\AgdaDatatype{∈}}\AgdaSpace{}%
\AgdaBound{Δ₁}\AgdaSpace{}%
\AgdaSymbol{→}\AgdaSpace{}%
\AgdaBound{l}\AgdaSpace{}%
\AgdaOperator{\AgdaDatatype{∈}}\AgdaSpace{}%
\AgdaBound{Δ₂}\<%
\\
%
\\[\AgdaEmptyExtraSkip]%
\>[0]\AgdaFunction{wkᵣ}\AgdaSpace{}%
\AgdaSymbol{:}\AgdaSpace{}%
\AgdaFunction{Ren}\AgdaSpace{}%
\AgdaGeneralizable{Δ}\AgdaSpace{}%
\AgdaSymbol{(}\AgdaGeneralizable{l}\AgdaSpace{}%
\AgdaOperator{\AgdaInductiveConstructor{∷}}\AgdaSpace{}%
\AgdaGeneralizable{Δ}\AgdaSymbol{)}\<%
\\
\>[0]\AgdaFunction{wkᵣ}\AgdaSpace{}%
\AgdaSymbol{=}\AgdaSpace{}%
\AgdaInductiveConstructor{there}\<%
\\
%
\\[\AgdaEmptyExtraSkip]%
\>[0]\AgdaFunction{extᵣ}\AgdaSpace{}%
\AgdaSymbol{:}\AgdaSpace{}%
\AgdaFunction{Ren}\AgdaSpace{}%
\AgdaGeneralizable{Δ₁}\AgdaSpace{}%
\AgdaGeneralizable{Δ₂}\AgdaSpace{}%
\AgdaSymbol{→}\AgdaSpace{}%
\AgdaFunction{Ren}\AgdaSpace{}%
\AgdaSymbol{(}\AgdaGeneralizable{l}\AgdaSpace{}%
\AgdaOperator{\AgdaInductiveConstructor{∷}}\AgdaSpace{}%
\AgdaGeneralizable{Δ₁}\AgdaSymbol{)}\AgdaSpace{}%
\AgdaSymbol{(}\AgdaGeneralizable{l}\AgdaSpace{}%
\AgdaOperator{\AgdaInductiveConstructor{∷}}\AgdaSpace{}%
\AgdaGeneralizable{Δ₂}\AgdaSymbol{)}\<%
\\
\>[0]\AgdaFunction{extᵣ}\AgdaSpace{}%
\AgdaBound{ρ}\AgdaSpace{}%
\AgdaInductiveConstructor{here}\AgdaSpace{}%
\AgdaSymbol{=}\AgdaSpace{}%
\AgdaInductiveConstructor{here}\<%
\\
\>[0]\AgdaFunction{extᵣ}\AgdaSpace{}%
\AgdaBound{ρ}\AgdaSpace{}%
\AgdaSymbol{(}\AgdaInductiveConstructor{there}\AgdaSpace{}%
\AgdaBound{x}\AgdaSymbol{)}\AgdaSpace{}%
\AgdaSymbol{=}\AgdaSpace{}%
\AgdaInductiveConstructor{there}\AgdaSpace{}%
\AgdaSymbol{(}\AgdaBound{ρ}\AgdaSpace{}%
\AgdaBound{x}\AgdaSymbol{)}\<%
\\
%
\\[\AgdaEmptyExtraSkip]%
\>[0]\AgdaFunction{renT}\AgdaSpace{}%
\AgdaSymbol{:}\AgdaSpace{}%
\AgdaFunction{Ren}\AgdaSpace{}%
\AgdaGeneralizable{Δ₁}\AgdaSpace{}%
\AgdaGeneralizable{Δ₂}\AgdaSpace{}%
\AgdaSymbol{→}\AgdaSpace{}%
\AgdaSymbol{(}\AgdaDatatype{Type}\AgdaSpace{}%
\AgdaGeneralizable{Δ₁}\AgdaSpace{}%
\AgdaGeneralizable{l}\AgdaSpace{}%
\AgdaSymbol{→}\AgdaSpace{}%
\AgdaDatatype{Type}\AgdaSpace{}%
\AgdaGeneralizable{Δ₂}\AgdaSpace{}%
\AgdaGeneralizable{l}\AgdaSymbol{)}\<%
\\
\>[0]\AgdaFunction{renT}\AgdaSpace{}%
\AgdaBound{ρ}\AgdaSpace{}%
\AgdaSymbol{(}\AgdaOperator{\AgdaInductiveConstructor{`}}\AgdaSpace{}%
\AgdaBound{x}\AgdaSymbol{)}\AgdaSpace{}%
\AgdaSymbol{=}\AgdaSpace{}%
\AgdaOperator{\AgdaInductiveConstructor{`}}\AgdaSpace{}%
\AgdaBound{ρ}\AgdaSpace{}%
\AgdaBound{x}\<%
\\
\>[0]\AgdaFunction{renT}\AgdaSpace{}%
\AgdaBound{ρ}\AgdaSpace{}%
\AgdaSymbol{(}\AgdaBound{T₁}\AgdaSpace{}%
\AgdaOperator{\AgdaInductiveConstructor{⇒}}\AgdaSpace{}%
\AgdaBound{T₂}\AgdaSymbol{)}\AgdaSpace{}%
\AgdaSymbol{=}\AgdaSpace{}%
\AgdaFunction{renT}\AgdaSpace{}%
\AgdaBound{ρ}\AgdaSpace{}%
\AgdaBound{T₁}\AgdaSpace{}%
\AgdaOperator{\AgdaInductiveConstructor{⇒}}\AgdaSpace{}%
\AgdaFunction{renT}\AgdaSpace{}%
\AgdaBound{ρ}\AgdaSpace{}%
\AgdaBound{T₂}\<%
\\
\>[0]\AgdaFunction{renT}\AgdaSpace{}%
\AgdaBound{ρ}\AgdaSpace{}%
\AgdaSymbol{(}\AgdaInductiveConstructor{`∀}\AgdaSpace{}%
\AgdaBound{lev}\AgdaSpace{}%
\AgdaBound{T}\AgdaSymbol{)}\AgdaSpace{}%
\AgdaSymbol{=}\AgdaSpace{}%
\AgdaInductiveConstructor{`∀}\AgdaSpace{}%
\AgdaBound{lev}\AgdaSpace{}%
\AgdaSymbol{(}\AgdaFunction{renT}\AgdaSpace{}%
\AgdaSymbol{(}\AgdaFunction{extᵣ}\AgdaSpace{}%
\AgdaBound{ρ}\AgdaSymbol{)}\AgdaSpace{}%
\AgdaBound{T}\AgdaSymbol{)}\<%
\\
\>[0]\AgdaFunction{renT}\AgdaSpace{}%
\AgdaBound{ρ}\AgdaSpace{}%
\AgdaInductiveConstructor{nat}\AgdaSpace{}%
\AgdaSymbol{=}\AgdaSpace{}%
\AgdaInductiveConstructor{nat}\<%
\\
%
\\[\AgdaEmptyExtraSkip]%
\>[0]\AgdaFunction{wkT}\AgdaSpace{}%
\AgdaSymbol{:}\AgdaSpace{}%
\AgdaDatatype{Type}\AgdaSpace{}%
\AgdaGeneralizable{Δ}\AgdaSpace{}%
\AgdaGeneralizable{l′}\AgdaSpace{}%
\AgdaSymbol{→}\AgdaSpace{}%
\AgdaDatatype{Type}\AgdaSpace{}%
\AgdaSymbol{(}\AgdaGeneralizable{l}\AgdaSpace{}%
\AgdaOperator{\AgdaInductiveConstructor{∷}}\AgdaSpace{}%
\AgdaGeneralizable{Δ}\AgdaSymbol{)}\AgdaSpace{}%
\AgdaGeneralizable{l′}\<%
\\
\>[0]\AgdaFunction{wkT}\AgdaSpace{}%
\AgdaSymbol{=}\AgdaSpace{}%
\AgdaFunction{renT}\AgdaSpace{}%
\AgdaFunction{wkᵣ}\<%
\\
%
\\[\AgdaEmptyExtraSkip]%
\>[0]\AgdaComment{--\ the\ action\ of\ renaming\ on\ semantic\ environments}\<%
\\
%
\\[\AgdaEmptyExtraSkip]%
\>[0]\AgdaFunction{Ren*}\AgdaSpace{}%
\AgdaSymbol{:}\AgdaSpace{}%
\AgdaSymbol{(}\AgdaBound{ρ}\AgdaSpace{}%
\AgdaSymbol{:}\AgdaSpace{}%
\AgdaFunction{Ren}\AgdaSpace{}%
\AgdaGeneralizable{Δ₁}\AgdaSpace{}%
\AgdaGeneralizable{Δ₂}\AgdaSymbol{)}\AgdaSpace{}%
\AgdaSymbol{→}\AgdaSpace{}%
\AgdaSymbol{(}\AgdaBound{η₁}\AgdaSpace{}%
\AgdaSymbol{:}\AgdaSpace{}%
\AgdaDatatype{DEnv}\AgdaSpace{}%
\AgdaGeneralizable{Δ₁}\AgdaSymbol{)}\AgdaSpace{}%
\AgdaSymbol{→}\AgdaSpace{}%
\AgdaSymbol{(}\AgdaBound{η₂}\AgdaSpace{}%
\AgdaSymbol{:}\AgdaSpace{}%
\AgdaDatatype{DEnv}\AgdaSpace{}%
\AgdaGeneralizable{Δ₂}\AgdaSymbol{)}\AgdaSpace{}%
\AgdaSymbol{→}\AgdaSpace{}%
\AgdaPrimitive{Setω}\<%
\\
\>[0]\AgdaFunction{Ren*}\AgdaSpace{}%
\AgdaSymbol{\{}\AgdaBound{Δ₁}\AgdaSymbol{\}}\AgdaSpace{}%
\AgdaBound{ρ}\AgdaSpace{}%
\AgdaBound{η₁}\AgdaSpace{}%
\AgdaBound{η₂}\AgdaSpace{}%
\AgdaSymbol{=}\AgdaSpace{}%
\AgdaSymbol{∀}\AgdaSpace{}%
\AgdaSymbol{\{}\AgdaBound{l}\AgdaSpace{}%
\AgdaSymbol{:}\AgdaSpace{}%
\AgdaPostulate{Level}\AgdaSymbol{\}}\AgdaSpace{}%
\AgdaSymbol{→}\AgdaSpace{}%
\AgdaSymbol{(}\AgdaBound{x}\AgdaSpace{}%
\AgdaSymbol{:}\AgdaSpace{}%
\AgdaBound{l}\AgdaSpace{}%
\AgdaOperator{\AgdaDatatype{∈}}\AgdaSpace{}%
\AgdaBound{Δ₁}\AgdaSymbol{)}\AgdaSpace{}%
\AgdaSymbol{→}\AgdaSpace{}%
\AgdaFunction{apply-env}\AgdaSpace{}%
\AgdaBound{η₂}\AgdaSpace{}%
\AgdaSymbol{(}\AgdaBound{ρ}\AgdaSpace{}%
\AgdaBound{x}\AgdaSymbol{)}\AgdaSpace{}%
\AgdaOperator{\AgdaDatatype{≡}}\AgdaSpace{}%
\AgdaFunction{apply-env}\AgdaSpace{}%
\AgdaBound{η₁}\AgdaSpace{}%
\AgdaBound{x}\<%
\\
%
\\[\AgdaEmptyExtraSkip]%
\>[0]\AgdaFunction{wkᵣ∈Ren*}\AgdaSpace{}%
\AgdaSymbol{:}\AgdaSpace{}%
\AgdaSymbol{∀}\AgdaSpace{}%
\AgdaSymbol{(}\AgdaBound{η}\AgdaSpace{}%
\AgdaSymbol{:}\AgdaSpace{}%
\AgdaDatatype{DEnv}\AgdaSpace{}%
\AgdaGeneralizable{Δ}\AgdaSymbol{)}\AgdaSpace{}%
\AgdaSymbol{(}\AgdaBound{⟦α⟧}\AgdaSpace{}%
\AgdaSymbol{:}\AgdaSpace{}%
\AgdaPrimitive{Set}\AgdaSpace{}%
\AgdaGeneralizable{l}\AgdaSymbol{)}\<%
\\
\>[0][@{}l@{\AgdaIndent{0}}]%
\>[2]\AgdaSymbol{→}\AgdaSpace{}%
\AgdaFunction{Ren*}\AgdaSpace{}%
\AgdaSymbol{(}\AgdaFunction{wkᵣ}\AgdaSymbol{\{}\AgdaGeneralizable{Δ}\AgdaSymbol{\}\{}\AgdaGeneralizable{l}\AgdaSymbol{\})}\AgdaSpace{}%
\AgdaBound{η}\AgdaSpace{}%
\AgdaSymbol{(}\AgdaBound{⟦α⟧}\AgdaSpace{}%
\AgdaOperator{\AgdaInductiveConstructor{∷}}\AgdaSpace{}%
\AgdaBound{η}\AgdaSymbol{)}\<%
\\
\>[0]\AgdaFunction{wkᵣ∈Ren*}\AgdaSpace{}%
\AgdaBound{η}\AgdaSpace{}%
\AgdaBound{⟦α⟧}\AgdaSpace{}%
\AgdaBound{x}\AgdaSpace{}%
\AgdaSymbol{=}\AgdaSpace{}%
\AgdaInductiveConstructor{refl}\<%
\\
%
\\[\AgdaEmptyExtraSkip]%
\>[0]\AgdaFunction{ren*-id}\AgdaSpace{}%
\AgdaSymbol{:}\AgdaSpace{}%
\AgdaSymbol{(}\AgdaBound{η}\AgdaSpace{}%
\AgdaSymbol{:}\AgdaSpace{}%
\AgdaDatatype{DEnv}\AgdaSpace{}%
\AgdaGeneralizable{Δ}\AgdaSymbol{)}\AgdaSpace{}%
\AgdaSymbol{→}\AgdaSpace{}%
\AgdaFunction{Ren*}\AgdaSpace{}%
\AgdaFunction{id}\AgdaSpace{}%
\AgdaBound{η}\AgdaSpace{}%
\AgdaBound{η}\<%
\\
\>[0]\AgdaFunction{ren*-id}\AgdaSpace{}%
\AgdaBound{η}\AgdaSpace{}%
\AgdaBound{x}\AgdaSpace{}%
\AgdaSymbol{=}\AgdaSpace{}%
\AgdaInductiveConstructor{refl}\<%
\\
%
\\[\AgdaEmptyExtraSkip]%
\>[0]\AgdaFunction{ren*-pop}\AgdaSpace{}%
\AgdaSymbol{:}\AgdaSpace{}%
\AgdaSymbol{(}\AgdaBound{ρ}\AgdaSpace{}%
\AgdaSymbol{:}\AgdaSpace{}%
\AgdaFunction{Ren}\AgdaSpace{}%
\AgdaSymbol{(}\AgdaGeneralizable{l}\AgdaSpace{}%
\AgdaOperator{\AgdaInductiveConstructor{∷}}\AgdaSpace{}%
\AgdaGeneralizable{Δ₁}\AgdaSymbol{)}\AgdaSpace{}%
\AgdaGeneralizable{Δ₂}\AgdaSymbol{)}\AgdaSpace{}%
\AgdaSymbol{(}\AgdaBound{α}\AgdaSpace{}%
\AgdaSymbol{:}\AgdaSpace{}%
\AgdaPrimitive{Set}\AgdaSpace{}%
\AgdaGeneralizable{l}\AgdaSymbol{)}\AgdaSpace{}%
\AgdaSymbol{(}\AgdaBound{η₁}\AgdaSpace{}%
\AgdaSymbol{:}\AgdaSpace{}%
\AgdaDatatype{DEnv}\AgdaSpace{}%
\AgdaGeneralizable{Δ₁}\AgdaSymbol{)}\AgdaSpace{}%
\AgdaSymbol{(}\AgdaBound{η₂}\AgdaSpace{}%
\AgdaSymbol{:}\AgdaSpace{}%
\AgdaDatatype{DEnv}\AgdaSpace{}%
\AgdaGeneralizable{Δ₂}\AgdaSymbol{)}\<%
\\
\>[0][@{}l@{\AgdaIndent{0}}]%
\>[2]\AgdaSymbol{→}\AgdaSpace{}%
\AgdaFunction{Ren*}\AgdaSpace{}%
\AgdaBound{ρ}\AgdaSpace{}%
\AgdaSymbol{(}\AgdaBound{α}\AgdaSpace{}%
\AgdaOperator{\AgdaInductiveConstructor{∷}}\AgdaSpace{}%
\AgdaBound{η₁}\AgdaSymbol{)}\AgdaSpace{}%
\AgdaBound{η₂}\<%
\\
%
\>[2]\AgdaSymbol{→}\AgdaSpace{}%
\AgdaFunction{Ren*}\AgdaSpace{}%
\AgdaSymbol{(}\AgdaBound{ρ}\AgdaSpace{}%
\AgdaOperator{\AgdaFunction{∘}}\AgdaSpace{}%
\AgdaInductiveConstructor{there}\AgdaSymbol{)}\AgdaSpace{}%
\AgdaBound{η₁}\AgdaSpace{}%
\AgdaBound{η₂}\<%
\\
\>[0]\AgdaFunction{ren*-pop}\AgdaSpace{}%
\AgdaBound{ρ}\AgdaSpace{}%
\AgdaBound{α}\AgdaSpace{}%
\AgdaBound{η₁}\AgdaSpace{}%
\AgdaBound{η₂}\AgdaSpace{}%
\AgdaBound{ren*}\AgdaSpace{}%
\AgdaBound{x}\AgdaSpace{}%
\AgdaSymbol{=}\AgdaSpace{}%
\AgdaBound{ren*}\AgdaSpace{}%
\AgdaSymbol{(}\AgdaInductiveConstructor{there}\AgdaSpace{}%
\AgdaBound{x}\AgdaSymbol{)}\<%
\\
%
\\[\AgdaEmptyExtraSkip]%
\>[0]\AgdaFunction{ren*-ext}\AgdaSpace{}%
\AgdaSymbol{:}\AgdaSpace{}%
\AgdaSymbol{∀}\AgdaSpace{}%
\AgdaSymbol{\{}\AgdaBound{ρ}\AgdaSpace{}%
\AgdaSymbol{:}\AgdaSpace{}%
\AgdaFunction{Ren}\AgdaSpace{}%
\AgdaGeneralizable{Δ₁}\AgdaSpace{}%
\AgdaGeneralizable{Δ₂}\AgdaSymbol{\}\{}\AgdaBound{η₁}\AgdaSpace{}%
\AgdaSymbol{:}\AgdaSpace{}%
\AgdaDatatype{DEnv}\AgdaSpace{}%
\AgdaGeneralizable{Δ₁}\AgdaSymbol{\}\{}\AgdaBound{η₂}\AgdaSpace{}%
\AgdaSymbol{:}\AgdaSpace{}%
\AgdaDatatype{DEnv}\AgdaSpace{}%
\AgdaGeneralizable{Δ₂}\AgdaSymbol{\}}\AgdaSpace{}%
\AgdaSymbol{(}\AgdaBound{⟦α⟧}\AgdaSpace{}%
\AgdaSymbol{:}\AgdaSpace{}%
\AgdaPrimitive{Set}\AgdaSpace{}%
\AgdaGeneralizable{l}\AgdaSymbol{)}\<%
\\
\>[0][@{}l@{\AgdaIndent{0}}]%
\>[2]\AgdaSymbol{→}\AgdaSpace{}%
\AgdaFunction{Ren*}\AgdaSpace{}%
\AgdaBound{ρ}\AgdaSpace{}%
\AgdaBound{η₁}\AgdaSpace{}%
\AgdaBound{η₂}\<%
\\
%
\>[2]\AgdaSymbol{→}\AgdaSpace{}%
\AgdaFunction{Ren*}\AgdaSpace{}%
\AgdaSymbol{(}\AgdaFunction{extᵣ}\AgdaSpace{}%
\AgdaBound{ρ}\AgdaSymbol{)}\AgdaSpace{}%
\AgdaSymbol{(}\AgdaBound{⟦α⟧}\AgdaSpace{}%
\AgdaOperator{\AgdaInductiveConstructor{∷}}\AgdaSpace{}%
\AgdaBound{η₁}\AgdaSymbol{)}\AgdaSpace{}%
\AgdaSymbol{(}\AgdaBound{⟦α⟧}\AgdaSpace{}%
\AgdaOperator{\AgdaInductiveConstructor{∷}}\AgdaSpace{}%
\AgdaBound{η₂}\AgdaSymbol{)}\<%
\\
\>[0]\AgdaFunction{ren*-ext}\AgdaSpace{}%
\AgdaBound{⟦α⟧}\AgdaSpace{}%
\AgdaBound{ren*}\AgdaSpace{}%
\AgdaInductiveConstructor{here}\AgdaSpace{}%
\AgdaSymbol{=}\AgdaSpace{}%
\AgdaInductiveConstructor{refl}\<%
\\
\>[0]\AgdaFunction{ren*-ext}\AgdaSpace{}%
\AgdaBound{⟦α⟧}\AgdaSpace{}%
\AgdaBound{ren*}\AgdaSpace{}%
\AgdaSymbol{(}\AgdaInductiveConstructor{there}\AgdaSpace{}%
\AgdaBound{x}\AgdaSymbol{)}\AgdaSpace{}%
\AgdaSymbol{=}\AgdaSpace{}%
\AgdaBound{ren*}\AgdaSpace{}%
\AgdaBound{x}\<%
\end{code}
\newcommand\TFRenPreserverSemanticsType{%
\begin{code}%
\>[0]\AgdaFunction{ren*-preserves-semantics}\AgdaSpace{}%
\AgdaSymbol{:}\<%
\\
\>[0][@{}l@{\AgdaIndent{0}}]%
\>[2]\AgdaSymbol{∀}\AgdaSpace{}%
\AgdaSymbol{\{}\AgdaBound{ρ}\AgdaSpace{}%
\AgdaSymbol{:}\AgdaSpace{}%
\AgdaFunction{Ren}\AgdaSpace{}%
\AgdaGeneralizable{Δ₁}\AgdaSpace{}%
\AgdaGeneralizable{Δ₂}\AgdaSymbol{\}\{}\AgdaBound{η₁}\AgdaSpace{}%
\AgdaSymbol{:}\AgdaSpace{}%
\AgdaDatatype{DEnv}\AgdaSpace{}%
\AgdaGeneralizable{Δ₁}\AgdaSymbol{\}\{}\AgdaBound{η₂}\AgdaSpace{}%
\AgdaSymbol{:}\AgdaSpace{}%
\AgdaDatatype{DEnv}\AgdaSpace{}%
\AgdaGeneralizable{Δ₂}\AgdaSymbol{\}}\<%
\\
%
\>[2]\AgdaSymbol{→}\AgdaSpace{}%
\AgdaSymbol{(}\AgdaBound{ren*}\AgdaSpace{}%
\AgdaSymbol{:}\AgdaSpace{}%
\AgdaFunction{Ren*}\AgdaSpace{}%
\AgdaBound{ρ}\AgdaSpace{}%
\AgdaBound{η₁}\AgdaSpace{}%
\AgdaBound{η₂}\AgdaSymbol{)}\AgdaSpace{}%
\AgdaSymbol{→}\AgdaSpace{}%
\AgdaSymbol{(}\AgdaBound{T}\AgdaSpace{}%
\AgdaSymbol{:}\AgdaSpace{}%
\AgdaDatatype{Type}\AgdaSpace{}%
\AgdaGeneralizable{Δ₁}\AgdaSpace{}%
\AgdaGeneralizable{l}\AgdaSymbol{)}\<%
\\
%
\>[2]\AgdaSymbol{→}\AgdaSpace{}%
\AgdaOperator{\AgdaFunction{𝓣⟦}}\AgdaSpace{}%
\AgdaFunction{renT}\AgdaSpace{}%
\AgdaBound{ρ}\AgdaSpace{}%
\AgdaBound{T}\AgdaSpace{}%
\AgdaOperator{\AgdaFunction{⟧}}\AgdaSpace{}%
\AgdaBound{η₂}\AgdaSpace{}%
\AgdaOperator{\AgdaDatatype{≡}}\AgdaSpace{}%
\AgdaOperator{\AgdaFunction{𝓣⟦}}\AgdaSpace{}%
\AgdaBound{T}\AgdaSpace{}%
\AgdaOperator{\AgdaFunction{⟧}}\AgdaSpace{}%
\AgdaBound{η₁}\<%
\end{code}}
\begin{code}[hide]%
\>[0]\AgdaFunction{ren*-preserves-semantics}\AgdaSpace{}%
\AgdaBound{ren*}\AgdaSpace{}%
\AgdaInductiveConstructor{nat}\AgdaSpace{}%
\AgdaSymbol{=}\AgdaSpace{}%
\AgdaInductiveConstructor{refl}\<%
\\
\>[0]\AgdaFunction{ren*-preserves-semantics}\AgdaSpace{}%
\AgdaBound{ren*}\AgdaSpace{}%
\AgdaSymbol{(}\AgdaOperator{\AgdaInductiveConstructor{`}}\AgdaSpace{}%
\AgdaBound{x}\AgdaSymbol{)}\AgdaSpace{}%
\AgdaSymbol{=}\AgdaSpace{}%
\AgdaBound{ren*}\AgdaSpace{}%
\AgdaBound{x}\<%
\\
\>[0]\AgdaFunction{ren*-preserves-semantics}\AgdaSpace{}%
\AgdaSymbol{\{}\AgdaArgument{ρ}\AgdaSpace{}%
\AgdaSymbol{=}\AgdaSpace{}%
\AgdaBound{ρ}\AgdaSymbol{\}\{}\AgdaBound{η₁}\AgdaSymbol{\}\{}\AgdaBound{η₂}\AgdaSymbol{\}}\AgdaSpace{}%
\AgdaBound{ren*}\AgdaSpace{}%
\AgdaSymbol{(}\AgdaBound{T₁}\AgdaSpace{}%
\AgdaOperator{\AgdaInductiveConstructor{⇒}}\AgdaSpace{}%
\AgdaBound{T₂}\AgdaSymbol{)}\<%
\\
\>[0][@{}l@{\AgdaIndent{0}}]%
\>[2]\AgdaKeyword{rewrite}\AgdaSpace{}%
\AgdaFunction{ren*-preserves-semantics}\AgdaSpace{}%
\AgdaSymbol{\{}\AgdaArgument{ρ}\AgdaSpace{}%
\AgdaSymbol{=}\AgdaSpace{}%
\AgdaBound{ρ}\AgdaSymbol{\}\{}\AgdaBound{η₁}\AgdaSymbol{\}\{}\AgdaBound{η₂}\AgdaSymbol{\}}\AgdaSpace{}%
\AgdaBound{ren*}\AgdaSpace{}%
\AgdaBound{T₁}\<%
\\
%
\>[2]\AgdaSymbol{|}\AgdaSpace{}%
\AgdaFunction{ren*-preserves-semantics}\AgdaSpace{}%
\AgdaSymbol{\{}\AgdaArgument{ρ}\AgdaSpace{}%
\AgdaSymbol{=}\AgdaSpace{}%
\AgdaBound{ρ}\AgdaSymbol{\}\{}\AgdaBound{η₁}\AgdaSymbol{\}\{}\AgdaBound{η₂}\AgdaSymbol{\}}\AgdaSpace{}%
\AgdaBound{ren*}\AgdaSpace{}%
\AgdaBound{T₂}\<%
\\
%
\>[2]\AgdaSymbol{=}\AgdaSpace{}%
\AgdaInductiveConstructor{refl}\<%
\\
\>[0]\AgdaFunction{ren*-preserves-semantics}\AgdaSpace{}%
\AgdaSymbol{\{}\AgdaArgument{ρ}\AgdaSpace{}%
\AgdaSymbol{=}\AgdaSpace{}%
\AgdaBound{ρ}\AgdaSymbol{\}\{}\AgdaBound{η₁}\AgdaSymbol{\}\{}\AgdaBound{η₂}\AgdaSymbol{\}}\AgdaSpace{}%
\AgdaBound{ren*}\AgdaSpace{}%
\AgdaSymbol{(}\AgdaInductiveConstructor{`∀}\AgdaSpace{}%
\AgdaBound{l}\AgdaSpace{}%
\AgdaBound{T}\AgdaSymbol{)}\AgdaSpace{}%
\AgdaSymbol{=}\<%
\\
\>[0][@{}l@{\AgdaIndent{0}}]%
\>[2]\AgdaPostulate{∀-extensionality}\AgdaSpace{}%
\AgdaSymbol{(λ}\AgdaSpace{}%
\AgdaBound{α}\AgdaSpace{}%
\AgdaSymbol{→}\<%
\\
\>[2][@{}l@{\AgdaIndent{0}}]%
\>[4]\AgdaFunction{ren*-preserves-semantics}\AgdaSymbol{\{}\AgdaArgument{ρ}\AgdaSpace{}%
\AgdaSymbol{=}\AgdaSpace{}%
\AgdaFunction{extᵣ}\AgdaSpace{}%
\AgdaBound{ρ}\AgdaSymbol{\}\{}\AgdaBound{α}\AgdaSpace{}%
\AgdaOperator{\AgdaInductiveConstructor{∷}}\AgdaSpace{}%
\AgdaBound{η₁}\AgdaSymbol{\}\{}\AgdaBound{α}\AgdaSpace{}%
\AgdaOperator{\AgdaInductiveConstructor{∷}}\AgdaSpace{}%
\AgdaBound{η₂}\AgdaSymbol{\}}\AgdaSpace{}%
\AgdaSymbol{(}\AgdaFunction{ren*-ext}\AgdaSymbol{\{}\AgdaArgument{ρ}\AgdaSpace{}%
\AgdaSymbol{=}\AgdaSpace{}%
\AgdaBound{ρ}\AgdaSymbol{\}}\AgdaSpace{}%
\AgdaBound{α}\AgdaSpace{}%
\AgdaBound{ren*}\AgdaSymbol{)}\AgdaSpace{}%
\AgdaBound{T}\AgdaSymbol{)}\<%
\\
%
\\[\AgdaEmptyExtraSkip]%
\>[0]\AgdaKeyword{postulate}\<%
\\
\>[0][@{}l@{\AgdaIndent{0}}]%
\>[2]\AgdaPostulate{problematic-goal}\AgdaSpace{}%
\AgdaSymbol{:}\<%
\\
\>[2][@{}l@{\AgdaIndent{0}}]%
\>[4]\AgdaSymbol{∀}\AgdaSpace{}%
\AgdaSymbol{\{}\AgdaBound{l′}\AgdaSymbol{\}\{}\AgdaBound{Δ₁}\AgdaSpace{}%
\AgdaBound{Δ₂}\AgdaSpace{}%
\AgdaSymbol{:}\AgdaSpace{}%
\AgdaFunction{LEnv}\AgdaSymbol{\}\{}\AgdaBound{ρ}\AgdaSpace{}%
\AgdaSymbol{:}\AgdaSpace{}%
\AgdaFunction{Ren}\AgdaSpace{}%
\AgdaBound{Δ₁}\AgdaSpace{}%
\AgdaBound{Δ₂}\AgdaSymbol{\}\{}\AgdaBound{η₁}\AgdaSpace{}%
\AgdaSymbol{:}\AgdaSpace{}%
\AgdaDatatype{DEnv}\AgdaSpace{}%
\AgdaBound{Δ₁}\AgdaSymbol{\}\{}\AgdaBound{η₂}\AgdaSpace{}%
\AgdaSymbol{:}\AgdaSpace{}%
\AgdaDatatype{DEnv}\AgdaSpace{}%
\AgdaBound{Δ₂}\AgdaSymbol{\}}\<%
\\
%
\>[4]\AgdaSymbol{→}\AgdaSpace{}%
\AgdaSymbol{(}\AgdaBound{ren*}\AgdaSpace{}%
\AgdaSymbol{:}\AgdaSpace{}%
\AgdaFunction{Ren*}\AgdaSpace{}%
\AgdaBound{ρ}\AgdaSpace{}%
\AgdaBound{η₁}\AgdaSpace{}%
\AgdaBound{η₂}\AgdaSymbol{)}\AgdaSpace{}%
\AgdaSymbol{→}\<%
\end{code}
\newcommand\TFProblematicGoal{%
\begin{code}%
%
\>[4]\AgdaSymbol{(}\AgdaBound{T}\AgdaSpace{}%
\AgdaSymbol{:}\AgdaSpace{}%
\AgdaDatatype{Type}\AgdaSpace{}%
\AgdaSymbol{(}\AgdaGeneralizable{l}\AgdaSpace{}%
\AgdaOperator{\AgdaInductiveConstructor{∷}}\AgdaSpace{}%
\AgdaBound{Δ₁}\AgdaSymbol{)}\AgdaSpace{}%
\AgdaBound{l′}\AgdaSymbol{)}\AgdaSpace{}%
\AgdaSymbol{→}\<%
\\
%
\>[4]\AgdaSymbol{((}\AgdaBound{⟦α⟧}\AgdaSpace{}%
\AgdaSymbol{:}\AgdaSpace{}%
\AgdaPrimitive{Set}\AgdaSpace{}%
\AgdaGeneralizable{l}\AgdaSymbol{)}\AgdaSpace{}%
\AgdaSymbol{→}\AgdaSpace{}%
\AgdaOperator{\AgdaFunction{𝓣⟦}}\AgdaSpace{}%
\AgdaFunction{renT}\AgdaSpace{}%
\AgdaSymbol{(}\AgdaFunction{extᵣ}\AgdaSpace{}%
\AgdaBound{ρ}\AgdaSymbol{)}\AgdaSpace{}%
\AgdaBound{T}\AgdaSpace{}%
\AgdaOperator{\AgdaFunction{⟧}}\AgdaSpace{}%
\AgdaSymbol{(}\AgdaBound{⟦α⟧}\AgdaSpace{}%
\AgdaOperator{\AgdaInductiveConstructor{∷}}\AgdaSpace{}%
\AgdaBound{η₂}\AgdaSymbol{))}\AgdaSpace{}%
\AgdaOperator{\AgdaDatatype{≡}}\<%
\\
\>[4][@{}l@{\AgdaIndent{0}}]%
\>[7]\AgdaSymbol{((}\AgdaBound{⟦α⟧}\AgdaSpace{}%
\AgdaSymbol{:}\AgdaSpace{}%
\AgdaPrimitive{Set}\AgdaSpace{}%
\AgdaGeneralizable{l}\AgdaSymbol{)}\AgdaSpace{}%
\AgdaSymbol{→}\AgdaSpace{}%
\AgdaOperator{\AgdaFunction{𝓣⟦}}\AgdaSpace{}%
\AgdaBound{T}\AgdaSpace{}%
\AgdaOperator{\AgdaFunction{⟧}}\AgdaSpace{}%
\AgdaSymbol{(}\AgdaBound{⟦α⟧}\AgdaSpace{}%
\AgdaOperator{\AgdaInductiveConstructor{∷}}\AgdaSpace{}%
\AgdaBound{η₁}\AgdaSymbol{))}\<%
\end{code}}
\begin{code}[hide]%
\>[0]\AgdaComment{--\ substitution\ on\ types}\<%
\\
%
\\[\AgdaEmptyExtraSkip]%
\>[0]\AgdaKeyword{data}\AgdaSpace{}%
\AgdaDatatype{Sub}\AgdaSpace{}%
\AgdaSymbol{:}\AgdaSpace{}%
\AgdaFunction{LEnv}\AgdaSpace{}%
\AgdaSymbol{→}\AgdaSpace{}%
\AgdaFunction{LEnv}\AgdaSpace{}%
\AgdaSymbol{→}\AgdaSpace{}%
\AgdaPrimitive{Set}\AgdaSpace{}%
\AgdaKeyword{where}\<%
\\
\>[0][@{}l@{\AgdaIndent{0}}]%
\>[2]\AgdaInductiveConstructor{[]}%
\>[6]\AgdaSymbol{:}\AgdaSpace{}%
\AgdaDatatype{Sub}\AgdaSpace{}%
\AgdaInductiveConstructor{[]}\AgdaSpace{}%
\AgdaGeneralizable{Δ₂}\<%
\\
%
\>[2]\AgdaOperator{\AgdaInductiveConstructor{\AgdaUnderscore{}∷\AgdaUnderscore{}}}\AgdaSpace{}%
\AgdaSymbol{:}\AgdaSpace{}%
\AgdaDatatype{Type}\AgdaSpace{}%
\AgdaGeneralizable{Δ₂}\AgdaSpace{}%
\AgdaGeneralizable{l}\AgdaSpace{}%
\AgdaSymbol{→}\AgdaSpace{}%
\AgdaDatatype{Sub}\AgdaSpace{}%
\AgdaGeneralizable{Δ₁}\AgdaSpace{}%
\AgdaGeneralizable{Δ₂}\AgdaSpace{}%
\AgdaSymbol{→}\AgdaSpace{}%
\AgdaDatatype{Sub}\AgdaSpace{}%
\AgdaSymbol{(}\AgdaGeneralizable{l}\AgdaSpace{}%
\AgdaOperator{\AgdaInductiveConstructor{∷}}\AgdaSpace{}%
\AgdaGeneralizable{Δ₁}\AgdaSymbol{)}\AgdaSpace{}%
\AgdaGeneralizable{Δ₂}\<%
\\
%
\\[\AgdaEmptyExtraSkip]%
\>[0]\AgdaFunction{apply-sub}\AgdaSpace{}%
\AgdaSymbol{:}\AgdaSpace{}%
\AgdaDatatype{Sub}\AgdaSpace{}%
\AgdaGeneralizable{Δ₁}\AgdaSpace{}%
\AgdaGeneralizable{Δ₂}\AgdaSpace{}%
\AgdaSymbol{→}\AgdaSpace{}%
\AgdaGeneralizable{l}\AgdaSpace{}%
\AgdaOperator{\AgdaDatatype{∈}}\AgdaSpace{}%
\AgdaGeneralizable{Δ₁}\AgdaSpace{}%
\AgdaSymbol{→}\AgdaSpace{}%
\AgdaDatatype{Type}\AgdaSpace{}%
\AgdaGeneralizable{Δ₂}\AgdaSpace{}%
\AgdaGeneralizable{l}\<%
\\
\>[0]\AgdaFunction{apply-sub}\AgdaSpace{}%
\AgdaSymbol{(}\AgdaBound{T}\AgdaSpace{}%
\AgdaOperator{\AgdaInductiveConstructor{∷}}\AgdaSpace{}%
\AgdaSymbol{\AgdaUnderscore{})}\AgdaSpace{}%
\AgdaInductiveConstructor{here}\AgdaSpace{}%
\AgdaSymbol{=}\AgdaSpace{}%
\AgdaBound{T}\<%
\\
\>[0]\AgdaFunction{apply-sub}\AgdaSpace{}%
\AgdaSymbol{(\AgdaUnderscore{}}\AgdaSpace{}%
\AgdaOperator{\AgdaInductiveConstructor{∷}}\AgdaSpace{}%
\AgdaBound{σ}\AgdaSymbol{)}\AgdaSpace{}%
\AgdaSymbol{(}\AgdaInductiveConstructor{there}\AgdaSpace{}%
\AgdaBound{x}\AgdaSymbol{)}\AgdaSpace{}%
\AgdaSymbol{=}\AgdaSpace{}%
\AgdaFunction{apply-sub}\AgdaSpace{}%
\AgdaBound{σ}\AgdaSpace{}%
\AgdaBound{x}\<%
\\
%
\\[\AgdaEmptyExtraSkip]%
\>[0]\AgdaFunction{build-id}\AgdaSpace{}%
\AgdaSymbol{:}\AgdaSpace{}%
\AgdaSymbol{(}\AgdaBound{Δ₁}\AgdaSpace{}%
\AgdaSymbol{:}\AgdaSpace{}%
\AgdaFunction{LEnv}\AgdaSymbol{)}\AgdaSpace{}%
\AgdaSymbol{→}\AgdaSpace{}%
\AgdaFunction{Ren}\AgdaSpace{}%
\AgdaBound{Δ₁}\AgdaSpace{}%
\AgdaGeneralizable{Δ}\AgdaSpace{}%
\AgdaSymbol{→}\AgdaSpace{}%
\AgdaDatatype{Sub}\AgdaSpace{}%
\AgdaBound{Δ₁}\AgdaSpace{}%
\AgdaGeneralizable{Δ}\<%
\\
\>[0]\AgdaFunction{build-id}\AgdaSpace{}%
\AgdaInductiveConstructor{[]}\AgdaSpace{}%
\AgdaBound{ρ}\AgdaSpace{}%
\AgdaSymbol{=}\AgdaSpace{}%
\AgdaInductiveConstructor{[]}\<%
\\
\>[0]\AgdaFunction{build-id}\AgdaSpace{}%
\AgdaSymbol{(}\AgdaBound{l}\AgdaSpace{}%
\AgdaOperator{\AgdaInductiveConstructor{∷}}\AgdaSpace{}%
\AgdaBound{Δ₁}\AgdaSymbol{)}\AgdaSpace{}%
\AgdaBound{ρ}\AgdaSpace{}%
\AgdaSymbol{=}\AgdaSpace{}%
\AgdaSymbol{(}\AgdaOperator{\AgdaInductiveConstructor{`}}\AgdaSpace{}%
\AgdaBound{ρ}\AgdaSpace{}%
\AgdaInductiveConstructor{here}\AgdaSymbol{)}\AgdaSpace{}%
\AgdaOperator{\AgdaInductiveConstructor{∷}}\AgdaSpace{}%
\AgdaFunction{build-id}\AgdaSpace{}%
\AgdaBound{Δ₁}\AgdaSpace{}%
\AgdaSymbol{(}\AgdaBound{ρ}\AgdaSpace{}%
\AgdaOperator{\AgdaFunction{∘}}\AgdaSpace{}%
\AgdaInductiveConstructor{there}\AgdaSymbol{)}\<%
\\
%
\\[\AgdaEmptyExtraSkip]%
\>[0]\AgdaFunction{idₛ}\AgdaSpace{}%
\AgdaSymbol{:}\AgdaSpace{}%
\AgdaDatatype{Sub}\AgdaSpace{}%
\AgdaGeneralizable{Δ}\AgdaSpace{}%
\AgdaGeneralizable{Δ}\<%
\\
\>[0]\AgdaFunction{idₛ}\AgdaSpace{}%
\AgdaSymbol{\{}\AgdaBound{Δ}\AgdaSymbol{\}}\AgdaSpace{}%
\AgdaSymbol{=}\AgdaSpace{}%
\AgdaFunction{build-id}\AgdaSpace{}%
\AgdaBound{Δ}\AgdaSpace{}%
\AgdaFunction{id}\<%
\\
%
\\[\AgdaEmptyExtraSkip]%
\>[0]\AgdaFunction{wkₛ}\AgdaSpace{}%
\AgdaSymbol{:}\AgdaSpace{}%
\AgdaDatatype{Sub}\AgdaSpace{}%
\AgdaGeneralizable{Δ₁}\AgdaSpace{}%
\AgdaGeneralizable{Δ₂}\AgdaSpace{}%
\AgdaSymbol{→}\AgdaSpace{}%
\AgdaDatatype{Sub}\AgdaSpace{}%
\AgdaGeneralizable{Δ₁}\AgdaSpace{}%
\AgdaSymbol{(}\AgdaGeneralizable{l}\AgdaSpace{}%
\AgdaOperator{\AgdaInductiveConstructor{∷}}\AgdaSpace{}%
\AgdaGeneralizable{Δ₂}\AgdaSymbol{)}\<%
\\
\>[0]\AgdaFunction{wkₛ}\AgdaSpace{}%
\AgdaInductiveConstructor{[]}\AgdaSpace{}%
\AgdaSymbol{=}\AgdaSpace{}%
\AgdaInductiveConstructor{[]}\<%
\\
\>[0]\AgdaFunction{wkₛ}\AgdaSpace{}%
\AgdaSymbol{(}\AgdaBound{T}\AgdaSpace{}%
\AgdaOperator{\AgdaInductiveConstructor{∷}}\AgdaSpace{}%
\AgdaBound{σ}\AgdaSymbol{)}\AgdaSpace{}%
\AgdaSymbol{=}\AgdaSpace{}%
\AgdaFunction{wkT}\AgdaSpace{}%
\AgdaBound{T}\AgdaSpace{}%
\AgdaOperator{\AgdaInductiveConstructor{∷}}\AgdaSpace{}%
\AgdaFunction{wkₛ}\AgdaSpace{}%
\AgdaBound{σ}\<%
\\
%
\\[\AgdaEmptyExtraSkip]%
\>[0]\AgdaFunction{extₛ}\AgdaSpace{}%
\AgdaSymbol{:}\AgdaSpace{}%
\AgdaDatatype{Sub}\AgdaSpace{}%
\AgdaGeneralizable{Δ₁}\AgdaSpace{}%
\AgdaGeneralizable{Δ₂}\AgdaSpace{}%
\AgdaSymbol{→}\AgdaSpace{}%
\AgdaSymbol{∀}\AgdaSpace{}%
\AgdaSymbol{\{}\AgdaBound{l}\AgdaSymbol{\}}\AgdaSpace{}%
\AgdaSymbol{→}\AgdaSpace{}%
\AgdaDatatype{Sub}\AgdaSpace{}%
\AgdaSymbol{(}\AgdaBound{l}\AgdaSpace{}%
\AgdaOperator{\AgdaInductiveConstructor{∷}}\AgdaSpace{}%
\AgdaGeneralizable{Δ₁}\AgdaSymbol{)}\AgdaSpace{}%
\AgdaSymbol{(}\AgdaBound{l}\AgdaSpace{}%
\AgdaOperator{\AgdaInductiveConstructor{∷}}\AgdaSpace{}%
\AgdaGeneralizable{Δ₂}\AgdaSymbol{)}\<%
\\
\>[0]\AgdaFunction{extₛ}\AgdaSpace{}%
\AgdaBound{σ}\AgdaSpace{}%
\AgdaSymbol{=}\AgdaSpace{}%
\AgdaOperator{\AgdaInductiveConstructor{`}}\AgdaSpace{}%
\AgdaInductiveConstructor{here}\AgdaSpace{}%
\AgdaOperator{\AgdaInductiveConstructor{∷}}\AgdaSpace{}%
\AgdaFunction{wkₛ}\AgdaSpace{}%
\AgdaBound{σ}\<%
\\
%
\\[\AgdaEmptyExtraSkip]%
\>[0]\AgdaFunction{subT}\AgdaSpace{}%
\AgdaSymbol{:}\AgdaSpace{}%
\AgdaDatatype{Sub}\AgdaSpace{}%
\AgdaGeneralizable{Δ₁}\AgdaSpace{}%
\AgdaGeneralizable{Δ₂}\AgdaSpace{}%
\AgdaSymbol{→}\AgdaSpace{}%
\AgdaDatatype{Type}\AgdaSpace{}%
\AgdaGeneralizable{Δ₁}\AgdaSpace{}%
\AgdaGeneralizable{l}\AgdaSpace{}%
\AgdaSymbol{→}\AgdaSpace{}%
\AgdaDatatype{Type}\AgdaSpace{}%
\AgdaGeneralizable{Δ₂}\AgdaSpace{}%
\AgdaGeneralizable{l}\<%
\\
\>[0]\AgdaFunction{subT}\AgdaSpace{}%
\AgdaBound{σ}\AgdaSpace{}%
\AgdaInductiveConstructor{nat}\AgdaSpace{}%
\AgdaSymbol{=}\AgdaSpace{}%
\AgdaInductiveConstructor{nat}\<%
\\
\>[0]\AgdaFunction{subT}\AgdaSpace{}%
\AgdaBound{σ}\AgdaSpace{}%
\AgdaSymbol{(}\AgdaOperator{\AgdaInductiveConstructor{`}}\AgdaSpace{}%
\AgdaBound{α}\AgdaSymbol{)}\AgdaSpace{}%
\AgdaSymbol{=}\AgdaSpace{}%
\AgdaFunction{apply-sub}\AgdaSpace{}%
\AgdaBound{σ}\AgdaSpace{}%
\AgdaBound{α}\<%
\\
\>[0]\AgdaFunction{subT}\AgdaSpace{}%
\AgdaBound{σ}\AgdaSpace{}%
\AgdaSymbol{(}\AgdaBound{T₁}\AgdaSpace{}%
\AgdaOperator{\AgdaInductiveConstructor{⇒}}\AgdaSpace{}%
\AgdaBound{T₂}\AgdaSymbol{)}\AgdaSpace{}%
\AgdaSymbol{=}\AgdaSpace{}%
\AgdaFunction{subT}\AgdaSpace{}%
\AgdaBound{σ}\AgdaSpace{}%
\AgdaBound{T₁}\AgdaSpace{}%
\AgdaOperator{\AgdaInductiveConstructor{⇒}}\AgdaSpace{}%
\AgdaFunction{subT}\AgdaSpace{}%
\AgdaBound{σ}\AgdaSpace{}%
\AgdaBound{T₂}\<%
\\
\>[0]\AgdaFunction{subT}\AgdaSpace{}%
\AgdaBound{σ}\AgdaSpace{}%
\AgdaSymbol{(}\AgdaInductiveConstructor{`∀}\AgdaSpace{}%
\AgdaBound{l}\AgdaSpace{}%
\AgdaBound{T}\AgdaSymbol{)}\AgdaSpace{}%
\AgdaSymbol{=}\AgdaSpace{}%
\AgdaInductiveConstructor{`∀}\AgdaSpace{}%
\AgdaBound{l}\AgdaSpace{}%
\AgdaSymbol{(}\AgdaFunction{subT}\AgdaSpace{}%
\AgdaSymbol{(}\AgdaFunction{extₛ}\AgdaSpace{}%
\AgdaBound{σ}\AgdaSymbol{)}\AgdaSpace{}%
\AgdaBound{T}\AgdaSymbol{)}\<%
\\
%
\\[\AgdaEmptyExtraSkip]%
\>[0]\AgdaFunction{singleₛ}\AgdaSpace{}%
\AgdaSymbol{:}\AgdaSpace{}%
\AgdaDatatype{Sub}\AgdaSpace{}%
\AgdaGeneralizable{Δ₁}\AgdaSpace{}%
\AgdaGeneralizable{Δ₂}\AgdaSpace{}%
\AgdaSymbol{→}\AgdaSpace{}%
\AgdaSymbol{∀}\AgdaSpace{}%
\AgdaSymbol{\{}\AgdaBound{l}\AgdaSymbol{\}}\AgdaSpace{}%
\AgdaSymbol{→}\AgdaSpace{}%
\AgdaDatatype{Type}\AgdaSpace{}%
\AgdaGeneralizable{Δ₂}\AgdaSpace{}%
\AgdaBound{l}\AgdaSpace{}%
\AgdaSymbol{→}\AgdaSpace{}%
\AgdaDatatype{Sub}\AgdaSpace{}%
\AgdaSymbol{(}\AgdaBound{l}\AgdaSpace{}%
\AgdaOperator{\AgdaInductiveConstructor{∷}}\AgdaSpace{}%
\AgdaGeneralizable{Δ₁}\AgdaSymbol{)}\AgdaSpace{}%
\AgdaGeneralizable{Δ₂}\<%
\\
\>[0]\AgdaFunction{singleₛ}\AgdaSpace{}%
\AgdaBound{σ}\AgdaSpace{}%
\AgdaBound{T'}\AgdaSpace{}%
\AgdaSymbol{=}\AgdaSpace{}%
\AgdaBound{T'}\AgdaSpace{}%
\AgdaOperator{\AgdaInductiveConstructor{∷}}\AgdaSpace{}%
\AgdaBound{σ}\<%
\\
%
\\[\AgdaEmptyExtraSkip]%
\>[0]\AgdaOperator{\AgdaFunction{\AgdaUnderscore{}[\AgdaUnderscore{}]T}}\AgdaSpace{}%
\AgdaSymbol{:}\AgdaSpace{}%
\AgdaDatatype{Type}\AgdaSpace{}%
\AgdaSymbol{(}\AgdaGeneralizable{l}\AgdaSpace{}%
\AgdaOperator{\AgdaInductiveConstructor{∷}}\AgdaSpace{}%
\AgdaGeneralizable{Δ}\AgdaSymbol{)}\AgdaSpace{}%
\AgdaGeneralizable{l′}\AgdaSpace{}%
\AgdaSymbol{→}\AgdaSpace{}%
\AgdaDatatype{Type}\AgdaSpace{}%
\AgdaGeneralizable{Δ}\AgdaSpace{}%
\AgdaGeneralizable{l}\AgdaSpace{}%
\AgdaSymbol{→}\AgdaSpace{}%
\AgdaDatatype{Type}\AgdaSpace{}%
\AgdaGeneralizable{Δ}\AgdaSpace{}%
\AgdaGeneralizable{l′}\<%
\\
\>[0]\AgdaOperator{\AgdaFunction{\AgdaUnderscore{}[\AgdaUnderscore{}]T}}\AgdaSpace{}%
\AgdaBound{T}\AgdaSpace{}%
\AgdaBound{T'}\AgdaSpace{}%
\AgdaSymbol{=}\AgdaSpace{}%
\AgdaFunction{subT}\AgdaSpace{}%
\AgdaSymbol{(}\AgdaFunction{singleₛ}\AgdaSpace{}%
\AgdaFunction{idₛ}\AgdaSpace{}%
\AgdaBound{T'}\AgdaSymbol{)}\AgdaSpace{}%
\AgdaBound{T}\<%
\\
%
\\[\AgdaEmptyExtraSkip]%
\>[0]\AgdaComment{--\ type\ environments}\<%
\end{code}
\newcommand\TFTVEnv{%
\begin{code}%
\>[0]\AgdaKeyword{data}\AgdaSpace{}%
\AgdaDatatype{TEnv}\AgdaSpace{}%
\AgdaSymbol{:}\AgdaSpace{}%
\AgdaFunction{LEnv}\AgdaSpace{}%
\AgdaSymbol{→}\AgdaSpace{}%
\AgdaPrimitive{Set}\AgdaSpace{}%
\AgdaKeyword{where}\<%
\\
\>[0][@{}l@{\AgdaIndent{0}}]%
\>[2]\AgdaInductiveConstructor{∅}%
\>[7]\AgdaSymbol{:}\AgdaSpace{}%
\AgdaDatatype{TEnv}\AgdaSpace{}%
\AgdaInductiveConstructor{[]}\<%
\\
%
\>[2]\AgdaOperator{\AgdaInductiveConstructor{\AgdaUnderscore{}◁\AgdaUnderscore{}}}%
\>[7]\AgdaSymbol{:}\AgdaSpace{}%
\AgdaDatatype{Type}\AgdaSpace{}%
\AgdaGeneralizable{Δ}\AgdaSpace{}%
\AgdaGeneralizable{l}\AgdaSpace{}%
\AgdaSymbol{→}\AgdaSpace{}%
\AgdaDatatype{TEnv}\AgdaSpace{}%
\AgdaGeneralizable{Δ}\AgdaSpace{}%
\AgdaSymbol{→}\AgdaSpace{}%
\AgdaDatatype{TEnv}\AgdaSpace{}%
\AgdaGeneralizable{Δ}%
\>[37]\AgdaComment{--\ term\ variable}\<%
\\
%
\>[2]\AgdaOperator{\AgdaInductiveConstructor{\AgdaUnderscore{}◁*\AgdaUnderscore{}}}\AgdaSpace{}%
\AgdaSymbol{:}\AgdaSpace{}%
\AgdaSymbol{∀}\AgdaSpace{}%
\AgdaBound{l}\AgdaSpace{}%
\AgdaSymbol{→}\AgdaSpace{}%
\AgdaDatatype{TEnv}\AgdaSpace{}%
\AgdaGeneralizable{Δ}\AgdaSpace{}%
\AgdaSymbol{→}\AgdaSpace{}%
\AgdaDatatype{TEnv}\AgdaSpace{}%
\AgdaSymbol{(}\AgdaBound{l}\AgdaSpace{}%
\AgdaOperator{\AgdaInductiveConstructor{∷}}\AgdaSpace{}%
\AgdaGeneralizable{Δ}\AgdaSymbol{)}\AgdaSpace{}%
\AgdaComment{--\ type\ variable}\<%
\end{code}}
\newcommand\TFinn{%
\begin{code}%
\>[0]\AgdaKeyword{data}\AgdaSpace{}%
\AgdaDatatype{inn}\AgdaSpace{}%
\AgdaSymbol{:}\AgdaSpace{}%
\AgdaDatatype{Type}\AgdaSpace{}%
\AgdaGeneralizable{Δ}\AgdaSpace{}%
\AgdaGeneralizable{l}\AgdaSpace{}%
\AgdaSymbol{→}\AgdaSpace{}%
\AgdaDatatype{TEnv}\AgdaSpace{}%
\AgdaGeneralizable{Δ}\AgdaSpace{}%
\AgdaSymbol{→}\AgdaSpace{}%
\AgdaPrimitive{Set}\AgdaSpace{}%
\AgdaKeyword{where}\<%
\\
\>[0][@{}l@{\AgdaIndent{0}}]%
\>[2]\AgdaInductiveConstructor{here}%
\>[8]\AgdaSymbol{:}\AgdaSpace{}%
\AgdaSymbol{∀}\AgdaSpace{}%
\AgdaSymbol{\{}\AgdaBound{T}\AgdaSpace{}%
\AgdaSymbol{:}\AgdaSpace{}%
\AgdaDatatype{Type}\AgdaSpace{}%
\AgdaGeneralizable{Δ}\AgdaSpace{}%
\AgdaGeneralizable{l}\AgdaSymbol{\}\{}\AgdaBound{Γ}\AgdaSymbol{\}}\<%
\\
%
\>[8]\AgdaSymbol{→}\AgdaSpace{}%
\AgdaDatatype{inn}\AgdaSpace{}%
\AgdaBound{T}\AgdaSpace{}%
\AgdaSymbol{(}\AgdaBound{T}\AgdaSpace{}%
\AgdaOperator{\AgdaInductiveConstructor{◁}}\AgdaSpace{}%
\AgdaBound{Γ}\AgdaSymbol{)}\<%
\\
%
\>[2]\AgdaInductiveConstructor{there}%
\>[1120I]\AgdaSymbol{:}\AgdaSpace{}%
\AgdaSymbol{∀}\AgdaSpace{}%
\AgdaSymbol{\{}\AgdaBound{T}\AgdaSpace{}%
\AgdaSymbol{:}\AgdaSpace{}%
\AgdaDatatype{Type}\AgdaSpace{}%
\AgdaGeneralizable{Δ}\AgdaSpace{}%
\AgdaGeneralizable{l}\AgdaSymbol{\}\{}\AgdaBound{T′}\AgdaSpace{}%
\AgdaSymbol{:}\AgdaSpace{}%
\AgdaDatatype{Type}\AgdaSpace{}%
\AgdaGeneralizable{Δ}\AgdaSpace{}%
\AgdaGeneralizable{l′}\AgdaSymbol{\}\{}\AgdaBound{Γ}\AgdaSymbol{\}}\<%
\\
\>[.][@{}l@{}]\<[1120I]%
\>[8]\AgdaSymbol{→}\AgdaSpace{}%
\AgdaDatatype{inn}\AgdaSpace{}%
\AgdaBound{T}\AgdaSpace{}%
\AgdaBound{Γ}\AgdaSpace{}%
\AgdaSymbol{→}\AgdaSpace{}%
\AgdaDatatype{inn}\AgdaSpace{}%
\AgdaBound{T}\AgdaSpace{}%
\AgdaSymbol{(}\AgdaBound{T′}\AgdaSpace{}%
\AgdaOperator{\AgdaInductiveConstructor{◁}}\AgdaSpace{}%
\AgdaBound{Γ}\AgdaSymbol{)}\<%
\\
%
\>[2]\AgdaInductiveConstructor{tskip}%
\>[1140I]\AgdaSymbol{:}\AgdaSpace{}%
\AgdaSymbol{∀}\AgdaSpace{}%
\AgdaSymbol{\{}\AgdaBound{T}\AgdaSpace{}%
\AgdaSymbol{:}\AgdaSpace{}%
\AgdaDatatype{Type}\AgdaSpace{}%
\AgdaGeneralizable{Δ}\AgdaSpace{}%
\AgdaGeneralizable{l}\AgdaSymbol{\}\{}\AgdaBound{Γ}\AgdaSymbol{\}}\<%
\\
\>[.][@{}l@{}]\<[1140I]%
\>[8]\AgdaSymbol{→}\AgdaSpace{}%
\AgdaDatatype{inn}\AgdaSpace{}%
\AgdaBound{T}\AgdaSpace{}%
\AgdaBound{Γ}\AgdaSpace{}%
\AgdaSymbol{→}\AgdaSpace{}%
\AgdaDatatype{inn}\AgdaSpace{}%
\AgdaSymbol{(}\AgdaFunction{wkT}\AgdaSpace{}%
\AgdaBound{T}\AgdaSymbol{)}\AgdaSpace{}%
\AgdaSymbol{(}\AgdaGeneralizable{l′}\AgdaSpace{}%
\AgdaOperator{\AgdaInductiveConstructor{◁*}}\AgdaSpace{}%
\AgdaBound{Γ}\AgdaSymbol{)}\<%
\end{code}}
\newcommand\TFExpr{%
\begin{code}%
\>[0]\AgdaKeyword{data}\AgdaSpace{}%
\AgdaDatatype{Expr}\AgdaSpace{}%
\AgdaSymbol{(}\AgdaBound{Δ}\AgdaSpace{}%
\AgdaSymbol{:}\AgdaSpace{}%
\AgdaFunction{LEnv}\AgdaSymbol{)}\AgdaSpace{}%
\AgdaSymbol{(}\AgdaBound{Γ}\AgdaSpace{}%
\AgdaSymbol{:}\AgdaSpace{}%
\AgdaDatatype{TEnv}\AgdaSpace{}%
\AgdaBound{Δ}\AgdaSymbol{)}\AgdaSpace{}%
\AgdaSymbol{:}\AgdaSpace{}%
\AgdaDatatype{Type}\AgdaSpace{}%
\AgdaBound{Δ}\AgdaSpace{}%
\AgdaGeneralizable{l}\AgdaSpace{}%
\AgdaSymbol{→}\AgdaSpace{}%
\AgdaPrimitive{Set}\AgdaSpace{}%
\AgdaKeyword{where}\<%
\\
\>[0][@{}l@{\AgdaIndent{0}}]%
\>[2]\AgdaOperator{\AgdaInductiveConstructor{\#\AgdaUnderscore{}}}%
\>[7]\AgdaSymbol{:}\AgdaSpace{}%
\AgdaSymbol{∀}\AgdaSpace{}%
\AgdaSymbol{(}\AgdaBound{n}\AgdaSpace{}%
\AgdaSymbol{:}\AgdaSpace{}%
\AgdaDatatype{ℕ}\AgdaSymbol{)}\AgdaSpace{}%
\AgdaSymbol{→}\AgdaSpace{}%
\AgdaDatatype{Expr}\AgdaSpace{}%
\AgdaBound{Δ}\AgdaSpace{}%
\AgdaBound{Γ}\AgdaSpace{}%
\AgdaInductiveConstructor{nat}\<%
\\
%
\>[2]\AgdaOperator{\AgdaInductiveConstructor{`\AgdaUnderscore{}}}%
\>[7]\AgdaSymbol{:}\AgdaSpace{}%
\AgdaSymbol{∀}\AgdaSpace{}%
\AgdaSymbol{\{}\AgdaBound{T}\AgdaSpace{}%
\AgdaSymbol{:}\AgdaSpace{}%
\AgdaDatatype{Type}\AgdaSpace{}%
\AgdaBound{Δ}\AgdaSpace{}%
\AgdaGeneralizable{l}\AgdaSymbol{\}}\<%
\\
%
\>[7]\AgdaSymbol{→}\AgdaSpace{}%
\AgdaDatatype{inn}\AgdaSpace{}%
\AgdaBound{T}\AgdaSpace{}%
\AgdaBound{Γ}\AgdaSpace{}%
\AgdaSymbol{→}\AgdaSpace{}%
\AgdaDatatype{Expr}\AgdaSpace{}%
\AgdaBound{Δ}\AgdaSpace{}%
\AgdaBound{Γ}\AgdaSpace{}%
\AgdaBound{T}\<%
\\
%
\>[2]\AgdaOperator{\AgdaInductiveConstructor{ƛ\AgdaUnderscore{}}}%
\>[7]\AgdaSymbol{:}\AgdaSpace{}%
\AgdaSymbol{∀}\AgdaSpace{}%
\AgdaSymbol{\{}\AgdaBound{T}\AgdaSpace{}%
\AgdaSymbol{:}\AgdaSpace{}%
\AgdaDatatype{Type}\AgdaSpace{}%
\AgdaBound{Δ}\AgdaSpace{}%
\AgdaGeneralizable{l}\AgdaSymbol{\}\{}\AgdaBound{T′}\AgdaSpace{}%
\AgdaSymbol{:}\AgdaSpace{}%
\AgdaDatatype{Type}\AgdaSpace{}%
\AgdaBound{Δ}\AgdaSpace{}%
\AgdaGeneralizable{l′}\AgdaSymbol{\}}\<%
\\
%
\>[7]\AgdaSymbol{→}\AgdaSpace{}%
\AgdaDatatype{Expr}\AgdaSpace{}%
\AgdaBound{Δ}\AgdaSpace{}%
\AgdaSymbol{(}\AgdaBound{T}\AgdaSpace{}%
\AgdaOperator{\AgdaInductiveConstructor{◁}}\AgdaSpace{}%
\AgdaBound{Γ}\AgdaSymbol{)}\AgdaSpace{}%
\AgdaBound{T′}\AgdaSpace{}%
\AgdaSymbol{→}\AgdaSpace{}%
\AgdaDatatype{Expr}\AgdaSpace{}%
\AgdaBound{Δ}\AgdaSpace{}%
\AgdaBound{Γ}\AgdaSpace{}%
\AgdaSymbol{(}\AgdaBound{T}\AgdaSpace{}%
\AgdaOperator{\AgdaInductiveConstructor{⇒}}\AgdaSpace{}%
\AgdaBound{T′}\AgdaSymbol{)}\<%
\\
%
\>[2]\AgdaOperator{\AgdaInductiveConstructor{\AgdaUnderscore{}·\AgdaUnderscore{}}}%
\>[7]\AgdaSymbol{:}\AgdaSpace{}%
\AgdaSymbol{∀}\AgdaSpace{}%
\AgdaSymbol{\{}\AgdaBound{T}\AgdaSpace{}%
\AgdaSymbol{:}\AgdaSpace{}%
\AgdaDatatype{Type}\AgdaSpace{}%
\AgdaBound{Δ}\AgdaSpace{}%
\AgdaGeneralizable{l}\AgdaSymbol{\}\{}\AgdaBound{T′}\AgdaSpace{}%
\AgdaSymbol{:}\AgdaSpace{}%
\AgdaDatatype{Type}\AgdaSpace{}%
\AgdaBound{Δ}\AgdaSpace{}%
\AgdaGeneralizable{l′}\AgdaSymbol{\}}\<%
\\
%
\>[7]\AgdaSymbol{→}\AgdaSpace{}%
\AgdaDatatype{Expr}\AgdaSpace{}%
\AgdaBound{Δ}\AgdaSpace{}%
\AgdaBound{Γ}\AgdaSpace{}%
\AgdaSymbol{(}\AgdaBound{T}\AgdaSpace{}%
\AgdaOperator{\AgdaInductiveConstructor{⇒}}\AgdaSpace{}%
\AgdaBound{T′}\AgdaSymbol{)}\AgdaSpace{}%
\AgdaSymbol{→}\AgdaSpace{}%
\AgdaDatatype{Expr}\AgdaSpace{}%
\AgdaBound{Δ}\AgdaSpace{}%
\AgdaBound{Γ}\AgdaSpace{}%
\AgdaBound{T}\AgdaSpace{}%
\AgdaSymbol{→}\AgdaSpace{}%
\AgdaDatatype{Expr}\AgdaSpace{}%
\AgdaBound{Δ}\AgdaSpace{}%
\AgdaBound{Γ}\AgdaSpace{}%
\AgdaBound{T′}\<%
\\
%
\>[2]\AgdaInductiveConstructor{Λ}%
\>[7]\AgdaSymbol{:}\AgdaSpace{}%
\AgdaSymbol{∀}\AgdaSpace{}%
\AgdaSymbol{(}\AgdaBound{l}\AgdaSpace{}%
\AgdaSymbol{:}\AgdaSpace{}%
\AgdaPostulate{Level}\AgdaSymbol{)}\AgdaSpace{}%
\AgdaSymbol{→}\AgdaSpace{}%
\AgdaSymbol{\{}\AgdaBound{T}\AgdaSpace{}%
\AgdaSymbol{:}\AgdaSpace{}%
\AgdaDatatype{Type}\AgdaSpace{}%
\AgdaSymbol{(}\AgdaBound{l}\AgdaSpace{}%
\AgdaOperator{\AgdaInductiveConstructor{∷}}\AgdaSpace{}%
\AgdaBound{Δ}\AgdaSymbol{)}\AgdaSpace{}%
\AgdaGeneralizable{l′}\AgdaSymbol{\}}\<%
\\
%
\>[7]\AgdaSymbol{→}\AgdaSpace{}%
\AgdaDatatype{Expr}\AgdaSpace{}%
\AgdaSymbol{(}\AgdaBound{l}\AgdaSpace{}%
\AgdaOperator{\AgdaInductiveConstructor{∷}}\AgdaSpace{}%
\AgdaBound{Δ}\AgdaSymbol{)}\AgdaSpace{}%
\AgdaSymbol{(}\AgdaBound{l}\AgdaSpace{}%
\AgdaOperator{\AgdaInductiveConstructor{◁*}}\AgdaSpace{}%
\AgdaBound{Γ}\AgdaSymbol{)}\AgdaSpace{}%
\AgdaBound{T}\AgdaSpace{}%
\AgdaSymbol{→}\AgdaSpace{}%
\AgdaDatatype{Expr}\AgdaSpace{}%
\AgdaBound{Δ}\AgdaSpace{}%
\AgdaBound{Γ}\AgdaSpace{}%
\AgdaSymbol{(}\AgdaInductiveConstructor{`∀}\AgdaSpace{}%
\AgdaBound{l}\AgdaSpace{}%
\AgdaBound{T}\AgdaSymbol{)}\<%
\\
%
\>[2]\AgdaOperator{\AgdaInductiveConstructor{\AgdaUnderscore{}∙\AgdaUnderscore{}}}%
\>[7]\AgdaSymbol{:}\AgdaSpace{}%
\AgdaSymbol{∀}\AgdaSpace{}%
\AgdaSymbol{\{}\AgdaBound{T}\AgdaSpace{}%
\AgdaSymbol{:}\AgdaSpace{}%
\AgdaDatatype{Type}\AgdaSpace{}%
\AgdaSymbol{(}\AgdaGeneralizable{l}\AgdaSpace{}%
\AgdaOperator{\AgdaInductiveConstructor{∷}}\AgdaSpace{}%
\AgdaBound{Δ}\AgdaSymbol{)}\AgdaSpace{}%
\AgdaGeneralizable{l′}\AgdaSymbol{\}}\<%
\\
%
\>[7]\AgdaSymbol{→}\AgdaSpace{}%
\AgdaDatatype{Expr}\AgdaSpace{}%
\AgdaBound{Δ}\AgdaSpace{}%
\AgdaBound{Γ}\AgdaSpace{}%
\AgdaSymbol{(}\AgdaInductiveConstructor{`∀}\AgdaSpace{}%
\AgdaGeneralizable{l}\AgdaSpace{}%
\AgdaBound{T}\AgdaSymbol{)}\AgdaSpace{}%
\AgdaSymbol{→}\AgdaSpace{}%
\AgdaSymbol{(}\AgdaBound{T′}\AgdaSpace{}%
\AgdaSymbol{:}\AgdaSpace{}%
\AgdaDatatype{Type}\AgdaSpace{}%
\AgdaBound{Δ}\AgdaSpace{}%
\AgdaGeneralizable{l}\AgdaSymbol{)}\<%
\\
%
\>[7]\AgdaSymbol{→}\AgdaSpace{}%
\AgdaDatatype{Expr}\AgdaSpace{}%
\AgdaBound{Δ}\AgdaSpace{}%
\AgdaBound{Γ}\AgdaSpace{}%
\AgdaSymbol{(}\AgdaBound{T}\AgdaSpace{}%
\AgdaOperator{\AgdaFunction{[}}\AgdaSpace{}%
\AgdaBound{T′}\AgdaSpace{}%
\AgdaOperator{\AgdaFunction{]T}}\AgdaSymbol{)}\<%
\end{code}}
\begin{code}[hide]%
\>[0]\AgdaComment{--\ value\ environments}\<%
\end{code}
\newcommand\TFVEnv{%
\begin{code}%
\>[0]\AgdaFunction{Env}\AgdaSpace{}%
\AgdaSymbol{:}\AgdaSpace{}%
\AgdaSymbol{(}\AgdaBound{Δ}\AgdaSpace{}%
\AgdaSymbol{:}\AgdaSpace{}%
\AgdaFunction{LEnv}\AgdaSymbol{)}\AgdaSpace{}%
\AgdaSymbol{→}\AgdaSpace{}%
\AgdaDatatype{TEnv}\AgdaSpace{}%
\AgdaBound{Δ}\AgdaSpace{}%
\AgdaSymbol{→}\AgdaSpace{}%
\AgdaDatatype{DEnv}\AgdaSpace{}%
\AgdaBound{Δ}\AgdaSpace{}%
\AgdaSymbol{→}\AgdaSpace{}%
\AgdaPrimitive{Setω}\<%
\\
\>[0]\AgdaFunction{Env}\AgdaSpace{}%
\AgdaBound{Δ}\AgdaSpace{}%
\AgdaBound{Γ}\AgdaSpace{}%
\AgdaBound{η}\AgdaSpace{}%
\AgdaSymbol{=}\AgdaSpace{}%
\AgdaSymbol{∀}\AgdaSpace{}%
\AgdaSymbol{\{}\AgdaBound{l}\AgdaSymbol{\}\{}\AgdaBound{T}\AgdaSpace{}%
\AgdaSymbol{:}\AgdaSpace{}%
\AgdaDatatype{Type}\AgdaSpace{}%
\AgdaBound{Δ}\AgdaSpace{}%
\AgdaBound{l}\AgdaSymbol{\}}\AgdaSpace{}%
\AgdaSymbol{→}\AgdaSpace{}%
\AgdaDatatype{inn}\AgdaSpace{}%
\AgdaBound{T}\AgdaSpace{}%
\AgdaBound{Γ}\AgdaSpace{}%
\AgdaSymbol{→}\AgdaSpace{}%
\AgdaOperator{\AgdaFunction{𝓣⟦}}\AgdaSpace{}%
\AgdaBound{T}\AgdaSpace{}%
\AgdaOperator{\AgdaFunction{⟧}}\AgdaSpace{}%
\AgdaBound{η}\<%
\end{code}}
\begin{code}[hide]%
\>[0]\AgdaFunction{extend}\AgdaSpace{}%
\AgdaSymbol{:}\AgdaSpace{}%
\AgdaSymbol{∀}\AgdaSpace{}%
\AgdaSymbol{\{}\AgdaBound{T}\AgdaSpace{}%
\AgdaSymbol{:}\AgdaSpace{}%
\AgdaDatatype{Type}\AgdaSpace{}%
\AgdaGeneralizable{Δ}\AgdaSpace{}%
\AgdaGeneralizable{l}\AgdaSymbol{\}\{}\AgdaBound{Γ}\AgdaSpace{}%
\AgdaSymbol{:}\AgdaSpace{}%
\AgdaDatatype{TEnv}\AgdaSpace{}%
\AgdaGeneralizable{Δ}\AgdaSymbol{\}\{}\AgdaBound{η}\AgdaSpace{}%
\AgdaSymbol{:}\AgdaSpace{}%
\AgdaDatatype{DEnv}\AgdaSpace{}%
\AgdaGeneralizable{Δ}\AgdaSymbol{\}}\<%
\\
\>[0][@{}l@{\AgdaIndent{0}}]%
\>[2]\AgdaSymbol{→}\AgdaSpace{}%
\AgdaFunction{Env}\AgdaSpace{}%
\AgdaGeneralizable{Δ}\AgdaSpace{}%
\AgdaBound{Γ}\AgdaSpace{}%
\AgdaBound{η}\<%
\\
%
\>[2]\AgdaSymbol{→}\AgdaSpace{}%
\AgdaOperator{\AgdaFunction{𝓣⟦}}\AgdaSpace{}%
\AgdaBound{T}\AgdaSpace{}%
\AgdaOperator{\AgdaFunction{⟧}}\AgdaSpace{}%
\AgdaBound{η}\<%
\\
%
\>[2]\AgdaSymbol{→}\AgdaSpace{}%
\AgdaFunction{Env}\AgdaSpace{}%
\AgdaGeneralizable{Δ}\AgdaSpace{}%
\AgdaSymbol{(}\AgdaBound{T}\AgdaSpace{}%
\AgdaOperator{\AgdaInductiveConstructor{◁}}\AgdaSpace{}%
\AgdaBound{Γ}\AgdaSymbol{)}\AgdaSpace{}%
\AgdaBound{η}\<%
\\
\>[0]\AgdaFunction{extend}\AgdaSpace{}%
\AgdaBound{γ}\AgdaSpace{}%
\AgdaBound{v}\AgdaSpace{}%
\AgdaInductiveConstructor{here}\AgdaSpace{}%
\AgdaSymbol{=}\AgdaSpace{}%
\AgdaBound{v}\<%
\\
\>[0]\AgdaFunction{extend}\AgdaSpace{}%
\AgdaBound{γ}\AgdaSpace{}%
\AgdaBound{v}\AgdaSpace{}%
\AgdaSymbol{(}\AgdaInductiveConstructor{there}\AgdaSpace{}%
\AgdaBound{x}\AgdaSymbol{)}\AgdaSpace{}%
\AgdaSymbol{=}\AgdaSpace{}%
\AgdaBound{γ}\AgdaSpace{}%
\AgdaBound{x}\<%
\end{code}
\newcommand\TFExtendTskip{%
\begin{code}%
\>[0]\AgdaFunction{extend-tskip}\AgdaSpace{}%
\AgdaSymbol{:}\AgdaSpace{}%
\AgdaSymbol{∀}\AgdaSpace{}%
\AgdaSymbol{\{}\AgdaBound{Δ}\AgdaSpace{}%
\AgdaSymbol{:}\AgdaSpace{}%
\AgdaFunction{LEnv}\AgdaSymbol{\}\{}\AgdaBound{Γ}\AgdaSpace{}%
\AgdaSymbol{:}\AgdaSpace{}%
\AgdaDatatype{TEnv}\AgdaSpace{}%
\AgdaBound{Δ}\AgdaSymbol{\}\{}\AgdaBound{η}\AgdaSpace{}%
\AgdaSymbol{:}\AgdaSpace{}%
\AgdaDatatype{DEnv}\AgdaSpace{}%
\AgdaBound{Δ}\AgdaSymbol{\}\{}\AgdaBound{⟦α⟧}\AgdaSpace{}%
\AgdaSymbol{:}\AgdaSpace{}%
\AgdaPrimitive{Set}\AgdaSpace{}%
\AgdaGeneralizable{l}\AgdaSymbol{\}}\<%
\\
\>[0][@{}l@{\AgdaIndent{0}}]%
\>[2]\AgdaSymbol{→}\AgdaSpace{}%
\AgdaFunction{Env}\AgdaSpace{}%
\AgdaBound{Δ}\AgdaSpace{}%
\AgdaBound{Γ}\AgdaSpace{}%
\AgdaBound{η}\AgdaSpace{}%
\AgdaSymbol{→}\AgdaSpace{}%
\AgdaFunction{Env}\AgdaSpace{}%
\AgdaSymbol{(}\AgdaGeneralizable{l}\AgdaSpace{}%
\AgdaOperator{\AgdaInductiveConstructor{∷}}\AgdaSpace{}%
\AgdaBound{Δ}\AgdaSymbol{)}\AgdaSpace{}%
\AgdaSymbol{(}\AgdaGeneralizable{l}\AgdaSpace{}%
\AgdaOperator{\AgdaInductiveConstructor{◁*}}\AgdaSpace{}%
\AgdaBound{Γ}\AgdaSymbol{)}\AgdaSpace{}%
\AgdaSymbol{(}\AgdaBound{⟦α⟧}\AgdaSpace{}%
\AgdaOperator{\AgdaInductiveConstructor{∷}}\AgdaSpace{}%
\AgdaBound{η}\AgdaSymbol{)}\<%
\\
\>[0]\AgdaFunction{extend-tskip}\AgdaSpace{}%
\AgdaSymbol{\{}\AgdaArgument{η}\AgdaSpace{}%
\AgdaSymbol{=}\AgdaSpace{}%
\AgdaBound{η}\AgdaSymbol{\}}\AgdaSpace{}%
\AgdaSymbol{\{}\AgdaArgument{⟦α⟧}\AgdaSpace{}%
\AgdaSymbol{=}\AgdaSpace{}%
\AgdaBound{⟦α⟧}\AgdaSymbol{\}}\AgdaSpace{}%
\AgdaBound{γ}\AgdaSpace{}%
\AgdaSymbol{(}\AgdaInductiveConstructor{tskip}\AgdaSymbol{\{}\AgdaArgument{T}\AgdaSpace{}%
\AgdaSymbol{=}\AgdaSpace{}%
\AgdaBound{T}\AgdaSymbol{\}}\AgdaSpace{}%
\AgdaBound{x}\AgdaSymbol{)}\<%
\\
\>[0][@{}l@{\AgdaIndent{0}}]%
\>[2]\AgdaKeyword{rewrite}%
\>[1408I]\AgdaFunction{ren*-preserves-semantics}\AgdaSpace{}%
\AgdaSymbol{\{}\AgdaArgument{ρ}\AgdaSpace{}%
\AgdaSymbol{=}\AgdaSpace{}%
\AgdaFunction{wkᵣ}\AgdaSymbol{\}\{}\AgdaBound{η}\AgdaSymbol{\}\{}\AgdaBound{⟦α⟧}\AgdaSpace{}%
\AgdaOperator{\AgdaInductiveConstructor{∷}}\AgdaSpace{}%
\AgdaBound{η}\AgdaSymbol{\}}\<%
\\
\>[1408I][@{}l@{\AgdaIndent{0}}]%
\>[12]\AgdaSymbol{(}\AgdaFunction{wkᵣ∈Ren*}\AgdaSpace{}%
\AgdaBound{η}\AgdaSpace{}%
\AgdaBound{⟦α⟧}\AgdaSymbol{)}\AgdaSpace{}%
\AgdaBound{T}\<%
\\
%
\>[2]\AgdaSymbol{=}\AgdaSpace{}%
\AgdaBound{γ}\AgdaSpace{}%
\AgdaBound{x}\<%
\end{code}}
\begin{code}[hide]%
\>[0]\AgdaFunction{subst-to-env*}\AgdaSpace{}%
\AgdaSymbol{:}\AgdaSpace{}%
\AgdaSymbol{(}\AgdaBound{σ}\AgdaSpace{}%
\AgdaSymbol{:}\AgdaSpace{}%
\AgdaDatatype{Sub}\AgdaSpace{}%
\AgdaGeneralizable{Δ₁}\AgdaSpace{}%
\AgdaGeneralizable{Δ₂}\AgdaSymbol{)}\AgdaSpace{}%
\AgdaSymbol{→}\AgdaSpace{}%
\AgdaSymbol{(}\AgdaBound{η₂}\AgdaSpace{}%
\AgdaSymbol{:}\AgdaSpace{}%
\AgdaDatatype{DEnv}\AgdaSpace{}%
\AgdaGeneralizable{Δ₂}\AgdaSymbol{)}\AgdaSpace{}%
\AgdaSymbol{→}\AgdaSpace{}%
\AgdaDatatype{DEnv}\AgdaSpace{}%
\AgdaGeneralizable{Δ₁}\<%
\\
\>[0]\AgdaFunction{subst-to-env*}\AgdaSpace{}%
\AgdaInductiveConstructor{[]}\AgdaSpace{}%
\AgdaBound{η₂}\AgdaSpace{}%
\AgdaSymbol{=}\AgdaSpace{}%
\AgdaInductiveConstructor{[]}\<%
\\
\>[0]\AgdaFunction{subst-to-env*}\AgdaSpace{}%
\AgdaSymbol{(}\AgdaBound{T}\AgdaSpace{}%
\AgdaOperator{\AgdaInductiveConstructor{∷}}\AgdaSpace{}%
\AgdaBound{σ}\AgdaSymbol{)}\AgdaSpace{}%
\AgdaBound{η₂}\AgdaSpace{}%
\AgdaSymbol{=}\AgdaSpace{}%
\AgdaOperator{\AgdaFunction{𝓣⟦}}\AgdaSpace{}%
\AgdaBound{T}\AgdaSpace{}%
\AgdaOperator{\AgdaFunction{⟧}}\AgdaSpace{}%
\AgdaBound{η₂}\AgdaSpace{}%
\AgdaOperator{\AgdaInductiveConstructor{∷}}\AgdaSpace{}%
\AgdaFunction{subst-to-env*}\AgdaSpace{}%
\AgdaBound{σ}\AgdaSpace{}%
\AgdaBound{η₂}\<%
\\
%
\\[\AgdaEmptyExtraSkip]%
\>[0]\AgdaFunction{subst-var-preserves}\AgdaSpace{}%
\AgdaSymbol{:}\AgdaSpace{}%
\AgdaSymbol{(}\AgdaBound{α}\AgdaSpace{}%
\AgdaSymbol{:}\AgdaSpace{}%
\AgdaGeneralizable{l}\AgdaSpace{}%
\AgdaOperator{\AgdaDatatype{∈}}\AgdaSpace{}%
\AgdaGeneralizable{Δ₁}\AgdaSymbol{)}\AgdaSpace{}%
\AgdaSymbol{(}\AgdaBound{σ}\AgdaSpace{}%
\AgdaSymbol{:}\AgdaSpace{}%
\AgdaDatatype{Sub}\AgdaSpace{}%
\AgdaGeneralizable{Δ₁}\AgdaSpace{}%
\AgdaGeneralizable{Δ₂}\AgdaSymbol{)}\AgdaSpace{}%
\AgdaSymbol{(}\AgdaBound{η₂}\AgdaSpace{}%
\AgdaSymbol{:}\AgdaSpace{}%
\AgdaDatatype{DEnv}\AgdaSpace{}%
\AgdaGeneralizable{Δ₂}\AgdaSymbol{)}\<%
\\
\>[0][@{}l@{\AgdaIndent{0}}]%
\>[2]\AgdaSymbol{→}\AgdaSpace{}%
\AgdaOperator{\AgdaFunction{𝓣⟦}}\AgdaSpace{}%
\AgdaFunction{apply-sub}\AgdaSpace{}%
\AgdaBound{σ}\AgdaSpace{}%
\AgdaBound{α}\AgdaSpace{}%
\AgdaOperator{\AgdaFunction{⟧}}\AgdaSpace{}%
\AgdaBound{η₂}\AgdaSpace{}%
\AgdaOperator{\AgdaDatatype{≡}}\AgdaSpace{}%
\AgdaFunction{apply-env}\AgdaSpace{}%
\AgdaSymbol{(}\AgdaFunction{subst-to-env*}\AgdaSpace{}%
\AgdaBound{σ}\AgdaSpace{}%
\AgdaBound{η₂}\AgdaSymbol{)}\AgdaSpace{}%
\AgdaBound{α}\<%
\\
\>[0]\AgdaFunction{subst-var-preserves}\AgdaSpace{}%
\AgdaInductiveConstructor{here}\AgdaSpace{}%
\AgdaSymbol{(}\AgdaBound{T}\AgdaSpace{}%
\AgdaOperator{\AgdaInductiveConstructor{∷}}\AgdaSpace{}%
\AgdaBound{σ}\AgdaSymbol{)}\AgdaSpace{}%
\AgdaBound{η₂}\AgdaSpace{}%
\AgdaSymbol{=}\AgdaSpace{}%
\AgdaInductiveConstructor{refl}\<%
\\
\>[0]\AgdaFunction{subst-var-preserves}\AgdaSpace{}%
\AgdaSymbol{(}\AgdaInductiveConstructor{there}\AgdaSpace{}%
\AgdaBound{α}\AgdaSymbol{)}\AgdaSpace{}%
\AgdaSymbol{(\AgdaUnderscore{}}\AgdaSpace{}%
\AgdaOperator{\AgdaInductiveConstructor{∷}}\AgdaSpace{}%
\AgdaBound{σ}\AgdaSymbol{)}\AgdaSpace{}%
\AgdaBound{η₂}\AgdaSpace{}%
\AgdaSymbol{=}\AgdaSpace{}%
\AgdaFunction{subst-var-preserves}\AgdaSpace{}%
\AgdaBound{α}\AgdaSpace{}%
\AgdaBound{σ}\AgdaSpace{}%
\AgdaBound{η₂}\<%
\\
%
\\[\AgdaEmptyExtraSkip]%
\>[0]\AgdaFunction{subst-to-env*-wk}\AgdaSpace{}%
\AgdaSymbol{:}\AgdaSpace{}%
\AgdaSymbol{(}\AgdaBound{σ}\AgdaSpace{}%
\AgdaSymbol{:}\AgdaSpace{}%
\AgdaDatatype{Sub}\AgdaSpace{}%
\AgdaGeneralizable{Δ₁}\AgdaSpace{}%
\AgdaGeneralizable{Δ₂}\AgdaSymbol{)}\AgdaSpace{}%
\AgdaSymbol{(}\AgdaBound{⟦α⟧}\AgdaSpace{}%
\AgdaSymbol{:}\AgdaSpace{}%
\AgdaPrimitive{Set}\AgdaSpace{}%
\AgdaGeneralizable{l}\AgdaSymbol{)}\AgdaSpace{}%
\AgdaSymbol{(}\AgdaBound{η₂}\AgdaSpace{}%
\AgdaSymbol{:}\AgdaSpace{}%
\AgdaDatatype{DEnv}\AgdaSpace{}%
\AgdaGeneralizable{Δ₂}\AgdaSymbol{)}\<%
\\
\>[0][@{}l@{\AgdaIndent{0}}]%
\>[2]\AgdaSymbol{→}\AgdaSpace{}%
\AgdaFunction{subst-to-env*}\AgdaSpace{}%
\AgdaSymbol{(}\AgdaFunction{wkₛ}\AgdaSpace{}%
\AgdaBound{σ}\AgdaSymbol{)}\AgdaSpace{}%
\AgdaSymbol{(}\AgdaBound{⟦α⟧}\AgdaSpace{}%
\AgdaOperator{\AgdaInductiveConstructor{∷}}\AgdaSpace{}%
\AgdaBound{η₂}\AgdaSymbol{)}\AgdaSpace{}%
\AgdaOperator{\AgdaDatatype{≡ω}}\AgdaSpace{}%
\AgdaFunction{subst-to-env*}\AgdaSpace{}%
\AgdaBound{σ}\AgdaSpace{}%
\AgdaBound{η₂}\<%
\\
\>[0]\AgdaFunction{subst-to-env*-wk}\AgdaSpace{}%
\AgdaInductiveConstructor{[]}\AgdaSpace{}%
\AgdaBound{⟦α⟧}\AgdaSpace{}%
\AgdaBound{η₂}\AgdaSpace{}%
\AgdaSymbol{=}\AgdaSpace{}%
\AgdaInductiveConstructor{refl}\<%
\\
\>[0]\AgdaFunction{subst-to-env*-wk}\AgdaSpace{}%
\AgdaSymbol{(}\AgdaBound{T}\AgdaSpace{}%
\AgdaOperator{\AgdaInductiveConstructor{∷}}\AgdaSpace{}%
\AgdaBound{σ}\AgdaSymbol{)}\AgdaSpace{}%
\AgdaBound{⟦α⟧}\AgdaSpace{}%
\AgdaBound{η₂}\<%
\\
\>[0][@{}l@{\AgdaIndent{0}}]%
\>[2]\AgdaKeyword{rewrite}\AgdaSpace{}%
\AgdaFunction{ren*-preserves-semantics}\AgdaSpace{}%
\AgdaSymbol{\{}\AgdaArgument{ρ}\AgdaSpace{}%
\AgdaSymbol{=}\AgdaSpace{}%
\AgdaFunction{wkᵣ}\AgdaSymbol{\}\{}\AgdaBound{η₂}\AgdaSymbol{\}\{}\AgdaBound{⟦α⟧}\AgdaSpace{}%
\AgdaOperator{\AgdaInductiveConstructor{∷}}\AgdaSpace{}%
\AgdaBound{η₂}\AgdaSymbol{\}}\AgdaSpace{}%
\AgdaSymbol{(}\AgdaFunction{wkᵣ∈Ren*}\AgdaSpace{}%
\AgdaBound{η₂}\AgdaSpace{}%
\AgdaBound{⟦α⟧}\AgdaSymbol{)}\AgdaSpace{}%
\AgdaBound{T}\<%
\\
%
\>[2]\AgdaSymbol{=}\AgdaSpace{}%
\AgdaFunction{congωω}\AgdaSpace{}%
\AgdaSymbol{(}\AgdaOperator{\AgdaFunction{𝓣⟦}}\AgdaSpace{}%
\AgdaBound{T}\AgdaSpace{}%
\AgdaOperator{\AgdaFunction{⟧}}\AgdaSpace{}%
\AgdaBound{η₂}\AgdaSpace{}%
\AgdaOperator{\AgdaInductiveConstructor{∷\AgdaUnderscore{}}}\AgdaSymbol{)}\AgdaSpace{}%
\AgdaSymbol{(}\AgdaFunction{subst-to-env*-wk}\AgdaSpace{}%
\AgdaBound{σ}\AgdaSpace{}%
\AgdaBound{⟦α⟧}\AgdaSpace{}%
\AgdaBound{η₂}\AgdaSymbol{)}\<%
\\
%
\\[\AgdaEmptyExtraSkip]%
\>[0]\AgdaFunction{subst-to-env*-build}\AgdaSpace{}%
\AgdaSymbol{:}\AgdaSpace{}%
\AgdaSymbol{∀}\AgdaSpace{}%
\AgdaSymbol{(}\AgdaBound{ρ}\AgdaSpace{}%
\AgdaSymbol{:}\AgdaSpace{}%
\AgdaFunction{Ren}\AgdaSpace{}%
\AgdaGeneralizable{Δ₁}\AgdaSpace{}%
\AgdaGeneralizable{Δ₂}\AgdaSymbol{)}\AgdaSpace{}%
\AgdaSymbol{(}\AgdaBound{η₁}\AgdaSpace{}%
\AgdaSymbol{:}\AgdaSpace{}%
\AgdaDatatype{DEnv}\AgdaSpace{}%
\AgdaGeneralizable{Δ₁}\AgdaSymbol{)}\AgdaSpace{}%
\AgdaSymbol{(}\AgdaBound{η₂}\AgdaSpace{}%
\AgdaSymbol{:}\AgdaSpace{}%
\AgdaDatatype{DEnv}\AgdaSpace{}%
\AgdaGeneralizable{Δ₂}\AgdaSymbol{)}\<%
\\
\>[0][@{}l@{\AgdaIndent{0}}]%
\>[2]\AgdaSymbol{→}\AgdaSpace{}%
\AgdaFunction{Ren*}\AgdaSpace{}%
\AgdaBound{ρ}\AgdaSpace{}%
\AgdaBound{η₁}\AgdaSpace{}%
\AgdaBound{η₂}\<%
\\
%
\>[2]\AgdaSymbol{→}\AgdaSpace{}%
\AgdaFunction{subst-to-env*}\AgdaSpace{}%
\AgdaSymbol{(}\AgdaFunction{build-id}\AgdaSpace{}%
\AgdaGeneralizable{Δ₁}\AgdaSpace{}%
\AgdaBound{ρ}\AgdaSymbol{)}\AgdaSpace{}%
\AgdaBound{η₂}\AgdaSpace{}%
\AgdaOperator{\AgdaDatatype{≡ω}}\AgdaSpace{}%
\AgdaBound{η₁}\<%
\\
\>[0]\AgdaFunction{subst-to-env*-build}\AgdaSpace{}%
\AgdaBound{ρ}\AgdaSpace{}%
\AgdaInductiveConstructor{[]}\AgdaSpace{}%
\AgdaBound{η₂}\AgdaSpace{}%
\AgdaBound{ren*}\AgdaSpace{}%
\AgdaSymbol{=}\AgdaSpace{}%
\AgdaInductiveConstructor{refl}\<%
\\
\>[0]\AgdaFunction{subst-to-env*-build}\AgdaSpace{}%
\AgdaSymbol{\{}\AgdaArgument{Δ₁}\AgdaSpace{}%
\AgdaSymbol{=}\AgdaSpace{}%
\AgdaSymbol{\AgdaUnderscore{}}\AgdaSpace{}%
\AgdaOperator{\AgdaInductiveConstructor{∷}}\AgdaSpace{}%
\AgdaBound{Δ₁}\AgdaSymbol{\}}\AgdaSpace{}%
\AgdaBound{ρ}\AgdaSpace{}%
\AgdaSymbol{(}\AgdaBound{⟦α⟧}\AgdaSpace{}%
\AgdaOperator{\AgdaInductiveConstructor{∷}}\AgdaSpace{}%
\AgdaBound{η₁}\AgdaSymbol{)}\AgdaSpace{}%
\AgdaBound{η₂}\AgdaSpace{}%
\AgdaBound{ren*}\AgdaSpace{}%
\AgdaSymbol{=}\<%
\\
\>[0][@{}l@{\AgdaIndent{0}}]%
\>[2]\AgdaFunction{transω}%
\>[1593I]\AgdaSymbol{(}\AgdaFunction{congωω}\AgdaSpace{}%
\AgdaSymbol{(}\AgdaFunction{apply-env}\AgdaSpace{}%
\AgdaBound{η₂}\AgdaSpace{}%
\AgdaSymbol{(}\AgdaBound{ρ}\AgdaSpace{}%
\AgdaInductiveConstructor{here}\AgdaSymbol{)}\AgdaSpace{}%
\AgdaOperator{\AgdaInductiveConstructor{∷\AgdaUnderscore{}}}\AgdaSymbol{)}\AgdaSpace{}%
\AgdaSymbol{(}\AgdaFunction{subst-to-env*-build}\AgdaSpace{}%
\AgdaSymbol{(}\AgdaBound{ρ}\AgdaSpace{}%
\AgdaOperator{\AgdaFunction{∘}}\AgdaSpace{}%
\AgdaInductiveConstructor{there}\AgdaSymbol{)}\AgdaSpace{}%
\AgdaBound{η₁}\AgdaSpace{}%
\AgdaBound{η₂}\AgdaSpace{}%
\AgdaSymbol{(}\AgdaFunction{ren*-pop}\AgdaSpace{}%
\AgdaBound{ρ}\AgdaSpace{}%
\AgdaBound{⟦α⟧}\AgdaSpace{}%
\AgdaBound{η₁}\AgdaSpace{}%
\AgdaBound{η₂}\AgdaSpace{}%
\AgdaBound{ren*}\AgdaSymbol{)))}\<%
\\
\>[.][@{}l@{}]\<[1593I]%
\>[9]\AgdaSymbol{(}\AgdaFunction{conglω}\AgdaSpace{}%
\AgdaSymbol{(}\AgdaOperator{\AgdaInductiveConstructor{\AgdaUnderscore{}∷}}\AgdaSpace{}%
\AgdaBound{η₁}\AgdaSymbol{)}\AgdaSpace{}%
\AgdaSymbol{(}\AgdaBound{ren*}\AgdaSpace{}%
\AgdaInductiveConstructor{here}\AgdaSymbol{))}\<%
\\
%
\\[\AgdaEmptyExtraSkip]%
\>[0]\AgdaFunction{subst-to-env*-id}\AgdaSpace{}%
\AgdaSymbol{:}\AgdaSpace{}%
\AgdaSymbol{(}\AgdaBound{η}\AgdaSpace{}%
\AgdaSymbol{:}\AgdaSpace{}%
\AgdaDatatype{DEnv}\AgdaSpace{}%
\AgdaGeneralizable{Δ}\AgdaSymbol{)}\AgdaSpace{}%
\AgdaSymbol{→}\AgdaSpace{}%
\AgdaFunction{subst-to-env*}\AgdaSpace{}%
\AgdaFunction{idₛ}\AgdaSpace{}%
\AgdaBound{η}\AgdaSpace{}%
\AgdaOperator{\AgdaDatatype{≡ω}}\AgdaSpace{}%
\AgdaBound{η}\<%
\\
\>[0]\AgdaFunction{subst-to-env*-id}\AgdaSpace{}%
\AgdaSymbol{\{}\AgdaBound{Δ}\AgdaSymbol{\}}\AgdaSpace{}%
\AgdaBound{η}\AgdaSpace{}%
\AgdaSymbol{=}\AgdaSpace{}%
\AgdaFunction{subst-to-env*-build}\AgdaSpace{}%
\AgdaSymbol{\{}\AgdaArgument{Δ₁}\AgdaSpace{}%
\AgdaSymbol{=}\AgdaSpace{}%
\AgdaBound{Δ}\AgdaSymbol{\}}\AgdaSpace{}%
\AgdaFunction{id}\AgdaSpace{}%
\AgdaBound{η}\AgdaSpace{}%
\AgdaBound{η}\AgdaSpace{}%
\AgdaSymbol{(}\AgdaFunction{ren*-id}\AgdaSpace{}%
\AgdaBound{η}\AgdaSymbol{)}\<%
\\
%
\\[\AgdaEmptyExtraSkip]%
\>[0]\AgdaFunction{subst-preserves}\AgdaSpace{}%
\AgdaSymbol{:}\<%
\\
\>[0][@{}l@{\AgdaIndent{0}}]%
\>[2]\AgdaSymbol{∀}\AgdaSpace{}%
\AgdaSymbol{\{}\AgdaBound{Δ₁}\AgdaSpace{}%
\AgdaBound{Δ₂}\AgdaSymbol{\}\{}\AgdaBound{l}\AgdaSymbol{\}\{}\AgdaBound{η₂}\AgdaSpace{}%
\AgdaSymbol{:}\AgdaSpace{}%
\AgdaDatatype{DEnv}\AgdaSpace{}%
\AgdaBound{Δ₂}\AgdaSymbol{\}}\<%
\\
%
\>[2]\AgdaSymbol{→}\AgdaSpace{}%
\AgdaSymbol{(}\AgdaBound{σ}\AgdaSpace{}%
\AgdaSymbol{:}\AgdaSpace{}%
\AgdaDatatype{Sub}\AgdaSpace{}%
\AgdaBound{Δ₁}\AgdaSpace{}%
\AgdaBound{Δ₂}\AgdaSymbol{)}\AgdaSpace{}%
\AgdaSymbol{(}\AgdaBound{T}\AgdaSpace{}%
\AgdaSymbol{:}\AgdaSpace{}%
\AgdaDatatype{Type}\AgdaSpace{}%
\AgdaBound{Δ₁}\AgdaSpace{}%
\AgdaBound{l}\AgdaSymbol{)}\<%
\\
%
\>[2]\AgdaSymbol{→}\AgdaSpace{}%
\AgdaOperator{\AgdaFunction{𝓣⟦}}\AgdaSpace{}%
\AgdaFunction{subT}\AgdaSpace{}%
\AgdaBound{σ}\AgdaSpace{}%
\AgdaBound{T}\AgdaSpace{}%
\AgdaOperator{\AgdaFunction{⟧}}\AgdaSpace{}%
\AgdaBound{η₂}\AgdaSpace{}%
\AgdaOperator{\AgdaDatatype{≡}}\AgdaSpace{}%
\AgdaOperator{\AgdaFunction{𝓣⟦}}\AgdaSpace{}%
\AgdaBound{T}\AgdaSpace{}%
\AgdaOperator{\AgdaFunction{⟧}}\AgdaSpace{}%
\AgdaSymbol{(}\AgdaFunction{subst-to-env*}\AgdaSpace{}%
\AgdaBound{σ}\AgdaSpace{}%
\AgdaBound{η₂}\AgdaSymbol{)}\<%
\\
\>[0]\AgdaFunction{subst-preserves}\AgdaSpace{}%
\AgdaBound{σ}\AgdaSpace{}%
\AgdaInductiveConstructor{nat}\AgdaSpace{}%
\AgdaSymbol{=}\AgdaSpace{}%
\AgdaInductiveConstructor{refl}\<%
\\
\>[0]\AgdaFunction{subst-preserves}\AgdaSpace{}%
\AgdaSymbol{\{}\AgdaArgument{η₂}\AgdaSpace{}%
\AgdaSymbol{=}\AgdaSpace{}%
\AgdaBound{η₂}\AgdaSymbol{\}}\AgdaSpace{}%
\AgdaBound{σ}\AgdaSpace{}%
\AgdaSymbol{(}\AgdaOperator{\AgdaInductiveConstructor{`}}\AgdaSpace{}%
\AgdaBound{α}\AgdaSymbol{)}\AgdaSpace{}%
\AgdaSymbol{=}\AgdaSpace{}%
\AgdaFunction{subst-var-preserves}\AgdaSpace{}%
\AgdaBound{α}\AgdaSpace{}%
\AgdaBound{σ}\AgdaSpace{}%
\AgdaBound{η₂}\<%
\\
\>[0]\AgdaFunction{subst-preserves}\AgdaSymbol{\{}\AgdaArgument{η₂}\AgdaSpace{}%
\AgdaSymbol{=}\AgdaSpace{}%
\AgdaBound{η₂}\AgdaSymbol{\}}\AgdaSpace{}%
\AgdaBound{σ}\AgdaSpace{}%
\AgdaSymbol{(}\AgdaBound{T₁}\AgdaSpace{}%
\AgdaOperator{\AgdaInductiveConstructor{⇒}}\AgdaSpace{}%
\AgdaBound{T₂}\AgdaSymbol{)}\<%
\\
\>[0][@{}l@{\AgdaIndent{0}}]%
\>[2]\AgdaKeyword{rewrite}\AgdaSpace{}%
\AgdaFunction{subst-preserves}\AgdaSymbol{\{}\AgdaArgument{η₂}\AgdaSpace{}%
\AgdaSymbol{=}\AgdaSpace{}%
\AgdaBound{η₂}\AgdaSymbol{\}}\AgdaSpace{}%
\AgdaBound{σ}\AgdaSpace{}%
\AgdaBound{T₁}\<%
\\
%
\>[2]\AgdaSymbol{|}%
\>[5]\AgdaFunction{subst-preserves}\AgdaSymbol{\{}\AgdaArgument{η₂}\AgdaSpace{}%
\AgdaSymbol{=}\AgdaSpace{}%
\AgdaBound{η₂}\AgdaSymbol{\}}\AgdaSpace{}%
\AgdaBound{σ}\AgdaSpace{}%
\AgdaBound{T₂}\AgdaSpace{}%
\AgdaSymbol{=}\AgdaSpace{}%
\AgdaInductiveConstructor{refl}\<%
\\
\>[0]\AgdaFunction{subst-preserves}\AgdaSpace{}%
\AgdaSymbol{\{}\AgdaArgument{η₂}\AgdaSpace{}%
\AgdaSymbol{=}\AgdaSpace{}%
\AgdaBound{η₂}\AgdaSymbol{\}}\AgdaSpace{}%
\AgdaBound{σ}\AgdaSpace{}%
\AgdaSymbol{(}\AgdaInductiveConstructor{`∀}\AgdaSpace{}%
\AgdaBound{l}\AgdaSpace{}%
\AgdaBound{T}\AgdaSymbol{)}\AgdaSpace{}%
\AgdaSymbol{=}\<%
\\
\>[0][@{}l@{\AgdaIndent{0}}]%
\>[2]\AgdaPostulate{∀-extensionality}\AgdaSpace{}%
\AgdaSymbol{(λ}\AgdaSpace{}%
\AgdaBound{⟦α⟧}\AgdaSpace{}%
\AgdaSymbol{→}\<%
\\
\>[2][@{}l@{\AgdaIndent{0}}]%
\>[4]\AgdaFunction{trans}%
\>[1710I]\AgdaSymbol{(}\AgdaFunction{subst-preserves}\AgdaSpace{}%
\AgdaSymbol{\{}\AgdaArgument{η₂}\AgdaSpace{}%
\AgdaSymbol{=}\AgdaSpace{}%
\AgdaBound{⟦α⟧}\AgdaSpace{}%
\AgdaOperator{\AgdaInductiveConstructor{∷}}\AgdaSpace{}%
\AgdaBound{η₂}\AgdaSymbol{\}}\AgdaSpace{}%
\AgdaSymbol{(}\AgdaFunction{extₛ}\AgdaSpace{}%
\AgdaBound{σ}\AgdaSymbol{)}\AgdaSpace{}%
\AgdaBound{T}\AgdaSymbol{)}\<%
\\
\>[.][@{}l@{}]\<[1710I]%
\>[10]\AgdaSymbol{(}\AgdaFunction{congωl}\AgdaSpace{}%
\AgdaSymbol{(λ}\AgdaSpace{}%
\AgdaBound{H}\AgdaSpace{}%
\AgdaSymbol{→}\AgdaSpace{}%
\AgdaOperator{\AgdaFunction{𝓣⟦}}\AgdaSpace{}%
\AgdaBound{T}\AgdaSpace{}%
\AgdaOperator{\AgdaFunction{⟧}}\AgdaSpace{}%
\AgdaSymbol{(}\AgdaBound{⟦α⟧}\AgdaSpace{}%
\AgdaOperator{\AgdaInductiveConstructor{∷}}\AgdaSpace{}%
\AgdaBound{H}\AgdaSymbol{))}\AgdaSpace{}%
\AgdaSymbol{(}\AgdaFunction{subst-to-env*-wk}\AgdaSpace{}%
\AgdaBound{σ}\AgdaSpace{}%
\AgdaBound{⟦α⟧}\AgdaSpace{}%
\AgdaBound{η₂}\AgdaSymbol{)))}\<%
\end{code}
\newcommand\TFSingleSubstPreserves{%
\begin{code}%
\>[0]\AgdaFunction{single-subst-preserves}\AgdaSpace{}%
\AgdaSymbol{:}\<%
\\
\>[0][@{}l@{\AgdaIndent{0}}]%
\>[2]\AgdaSymbol{∀}\AgdaSpace{}%
\AgdaSymbol{(}\AgdaBound{η}\AgdaSpace{}%
\AgdaSymbol{:}\AgdaSpace{}%
\AgdaDatatype{DEnv}\AgdaSpace{}%
\AgdaGeneralizable{Δ}\AgdaSymbol{)}\AgdaSpace{}%
\AgdaSymbol{(}\AgdaBound{T′}\AgdaSpace{}%
\AgdaSymbol{:}\AgdaSpace{}%
\AgdaDatatype{Type}\AgdaSpace{}%
\AgdaGeneralizable{Δ}\AgdaSpace{}%
\AgdaGeneralizable{l}\AgdaSymbol{)}\AgdaSpace{}%
\AgdaSymbol{(}\AgdaBound{T}\AgdaSpace{}%
\AgdaSymbol{:}\AgdaSpace{}%
\AgdaDatatype{Type}\AgdaSpace{}%
\AgdaSymbol{(}\AgdaGeneralizable{l}\AgdaSpace{}%
\AgdaOperator{\AgdaInductiveConstructor{∷}}\AgdaSpace{}%
\AgdaGeneralizable{Δ}\AgdaSymbol{)}\AgdaSpace{}%
\AgdaGeneralizable{l′}\AgdaSymbol{)}\<%
\\
%
\>[2]\AgdaSymbol{→}\AgdaSpace{}%
\AgdaOperator{\AgdaFunction{𝓣⟦}}\AgdaSpace{}%
\AgdaBound{T}\AgdaSpace{}%
\AgdaOperator{\AgdaFunction{[}}\AgdaSpace{}%
\AgdaBound{T′}\AgdaSpace{}%
\AgdaOperator{\AgdaFunction{]T}}\AgdaSpace{}%
\AgdaOperator{\AgdaFunction{⟧}}\AgdaSpace{}%
\AgdaBound{η}\AgdaSpace{}%
\AgdaOperator{\AgdaDatatype{≡}}\AgdaSpace{}%
\AgdaOperator{\AgdaFunction{𝓣⟦}}\AgdaSpace{}%
\AgdaBound{T}\AgdaSpace{}%
\AgdaOperator{\AgdaFunction{⟧}}\AgdaSpace{}%
\AgdaSymbol{(}\AgdaOperator{\AgdaFunction{𝓣⟦}}\AgdaSpace{}%
\AgdaBound{T′}\AgdaSpace{}%
\AgdaOperator{\AgdaFunction{⟧}}\AgdaSpace{}%
\AgdaBound{η}\AgdaSpace{}%
\AgdaOperator{\AgdaInductiveConstructor{∷}}\AgdaSpace{}%
\AgdaBound{η}\AgdaSymbol{)}\<%
\end{code}}
\begin{code}[hide]%
\>[0]\AgdaFunction{single-subst-preserves}\AgdaSpace{}%
\AgdaSymbol{\{}\AgdaArgument{Δ}\AgdaSpace{}%
\AgdaSymbol{=}\AgdaSpace{}%
\AgdaBound{Δ}\AgdaSymbol{\}}\AgdaSpace{}%
\AgdaSymbol{\{}\AgdaArgument{l}\AgdaSpace{}%
\AgdaSymbol{=}\AgdaSpace{}%
\AgdaBound{l}\AgdaSymbol{\}\{}\AgdaArgument{l′}\AgdaSpace{}%
\AgdaSymbol{=}\AgdaSpace{}%
\AgdaBound{l′}\AgdaSymbol{\}}\AgdaSpace{}%
\AgdaBound{η}\AgdaSpace{}%
\AgdaBound{T′}\AgdaSpace{}%
\AgdaBound{T}\AgdaSpace{}%
\AgdaSymbol{=}\<%
\\
\>[0][@{}l@{\AgdaIndent{0}}]%
\>[2]\AgdaFunction{trans}%
\>[1778I]\AgdaSymbol{(}\AgdaFunction{subst-preserves}\AgdaSpace{}%
\AgdaSymbol{(}\AgdaFunction{singleₛ}\AgdaSpace{}%
\AgdaFunction{idₛ}\AgdaSpace{}%
\AgdaBound{T′}\AgdaSymbol{)}\AgdaSpace{}%
\AgdaBound{T}\AgdaSymbol{)}\<%
\\
\>[.][@{}l@{}]\<[1778I]%
\>[8]\AgdaSymbol{(}\AgdaFunction{congωl}\AgdaSpace{}%
\AgdaSymbol{(λ}\AgdaSpace{}%
\AgdaBound{H}\AgdaSpace{}%
\AgdaSymbol{→}\AgdaSpace{}%
\AgdaOperator{\AgdaFunction{𝓣⟦}}\AgdaSpace{}%
\AgdaBound{T}\AgdaSpace{}%
\AgdaOperator{\AgdaFunction{⟧}}\AgdaSpace{}%
\AgdaSymbol{(}\AgdaOperator{\AgdaFunction{𝓣⟦}}\AgdaSpace{}%
\AgdaBound{T′}\AgdaSpace{}%
\AgdaOperator{\AgdaFunction{⟧}}\AgdaSpace{}%
\AgdaBound{η}\AgdaSpace{}%
\AgdaOperator{\AgdaInductiveConstructor{∷}}\AgdaSpace{}%
\AgdaBound{H}\AgdaSymbol{))}\AgdaSpace{}%
\AgdaSymbol{(}\AgdaFunction{subst-to-env*-id}\AgdaSpace{}%
\AgdaBound{η}\AgdaSymbol{))}\<%
\end{code}
\newcommand\TFExprSem{%
\begin{code}%
\>[0]\AgdaOperator{\AgdaFunction{𝓔⟦\AgdaUnderscore{}⟧}}\AgdaSpace{}%
\AgdaSymbol{:}\AgdaSpace{}%
\AgdaSymbol{∀}\AgdaSpace{}%
\AgdaSymbol{\{}\AgdaBound{T}\AgdaSpace{}%
\AgdaSymbol{:}\AgdaSpace{}%
\AgdaDatatype{Type}\AgdaSpace{}%
\AgdaGeneralizable{Δ}\AgdaSpace{}%
\AgdaGeneralizable{l}\AgdaSymbol{\}\{}\AgdaBound{Γ}\AgdaSpace{}%
\AgdaSymbol{:}\AgdaSpace{}%
\AgdaDatatype{TEnv}\AgdaSpace{}%
\AgdaGeneralizable{Δ}\AgdaSymbol{\}}\<%
\\
\>[0][@{}l@{\AgdaIndent{0}}]%
\>[2]\AgdaSymbol{→}\AgdaSpace{}%
\AgdaDatatype{Expr}\AgdaSpace{}%
\AgdaGeneralizable{Δ}\AgdaSpace{}%
\AgdaBound{Γ}\AgdaSpace{}%
\AgdaBound{T}\AgdaSpace{}%
\AgdaSymbol{→}\AgdaSpace{}%
\AgdaSymbol{(}\AgdaBound{η}\AgdaSpace{}%
\AgdaSymbol{:}\AgdaSpace{}%
\AgdaDatatype{DEnv}\AgdaSpace{}%
\AgdaGeneralizable{Δ}\AgdaSymbol{)}\AgdaSpace{}%
\AgdaSymbol{→}\AgdaSpace{}%
\AgdaFunction{Env}\AgdaSpace{}%
\AgdaGeneralizable{Δ}\AgdaSpace{}%
\AgdaBound{Γ}\AgdaSpace{}%
\AgdaBound{η}\AgdaSpace{}%
\AgdaSymbol{→}\AgdaSpace{}%
\AgdaOperator{\AgdaFunction{𝓣⟦}}\AgdaSpace{}%
\AgdaBound{T}\AgdaSpace{}%
\AgdaOperator{\AgdaFunction{⟧}}\AgdaSpace{}%
\AgdaBound{η}\<%
\\
\>[0]\AgdaOperator{\AgdaFunction{𝓔⟦}}\AgdaSpace{}%
\AgdaOperator{\AgdaInductiveConstructor{\#}}\AgdaSpace{}%
\AgdaBound{n}\AgdaSpace{}%
\AgdaOperator{\AgdaFunction{⟧}}\AgdaSpace{}%
\AgdaBound{η}\AgdaSpace{}%
\AgdaBound{γ}\AgdaSpace{}%
\AgdaSymbol{=}\AgdaSpace{}%
\AgdaBound{n}\<%
\\
\>[0]\AgdaOperator{\AgdaFunction{𝓔⟦}}\AgdaSpace{}%
\AgdaOperator{\AgdaInductiveConstructor{`}}\AgdaSpace{}%
\AgdaBound{x}\AgdaSpace{}%
\AgdaOperator{\AgdaFunction{⟧}}\AgdaSpace{}%
\AgdaBound{η}\AgdaSpace{}%
\AgdaBound{γ}\AgdaSpace{}%
\AgdaSymbol{=}\AgdaSpace{}%
\AgdaBound{γ}\AgdaSpace{}%
\AgdaBound{x}\<%
\\
\>[0]\AgdaOperator{\AgdaFunction{𝓔⟦}}\AgdaSpace{}%
\AgdaOperator{\AgdaInductiveConstructor{ƛ\AgdaUnderscore{}}}\AgdaSpace{}%
\AgdaBound{e}\AgdaSpace{}%
\AgdaOperator{\AgdaFunction{⟧}}\AgdaSpace{}%
\AgdaBound{η}\AgdaSpace{}%
\AgdaBound{γ}\AgdaSpace{}%
\AgdaSymbol{=}\AgdaSpace{}%
\AgdaSymbol{λ}\AgdaSpace{}%
\AgdaBound{v}\AgdaSpace{}%
\AgdaSymbol{→}\AgdaSpace{}%
\AgdaOperator{\AgdaFunction{𝓔⟦}}\AgdaSpace{}%
\AgdaBound{e}\AgdaSpace{}%
\AgdaOperator{\AgdaFunction{⟧}}\AgdaSpace{}%
\AgdaBound{η}\AgdaSpace{}%
\AgdaSymbol{(}\AgdaFunction{extend}\AgdaSpace{}%
\AgdaBound{γ}\AgdaSpace{}%
\AgdaBound{v}\AgdaSymbol{)}\<%
\\
\>[0]\AgdaOperator{\AgdaFunction{𝓔⟦}}\AgdaSpace{}%
\AgdaBound{e₁}\AgdaSpace{}%
\AgdaOperator{\AgdaInductiveConstructor{·}}\AgdaSpace{}%
\AgdaBound{e₂}\AgdaSpace{}%
\AgdaOperator{\AgdaFunction{⟧}}\AgdaSpace{}%
\AgdaBound{η}\AgdaSpace{}%
\AgdaBound{γ}\AgdaSpace{}%
\AgdaSymbol{=}\AgdaSpace{}%
\AgdaOperator{\AgdaFunction{𝓔⟦}}\AgdaSpace{}%
\AgdaBound{e₁}\AgdaSpace{}%
\AgdaOperator{\AgdaFunction{⟧}}\AgdaSpace{}%
\AgdaBound{η}\AgdaSpace{}%
\AgdaBound{γ}\AgdaSpace{}%
\AgdaSymbol{(}\AgdaOperator{\AgdaFunction{𝓔⟦}}\AgdaSpace{}%
\AgdaBound{e₂}\AgdaSpace{}%
\AgdaOperator{\AgdaFunction{⟧}}\AgdaSpace{}%
\AgdaBound{η}\AgdaSpace{}%
\AgdaBound{γ}\AgdaSymbol{)}\<%
\\
\>[0]\AgdaOperator{\AgdaFunction{𝓔⟦}}\AgdaSpace{}%
\AgdaInductiveConstructor{Λ}\AgdaSpace{}%
\AgdaBound{l}\AgdaSpace{}%
\AgdaBound{e}\AgdaSpace{}%
\AgdaOperator{\AgdaFunction{⟧}}\AgdaSpace{}%
\AgdaBound{η}\AgdaSpace{}%
\AgdaBound{γ}\AgdaSpace{}%
\AgdaSymbol{=}\AgdaSpace{}%
\AgdaSymbol{λ}\AgdaSpace{}%
\AgdaBound{⟦α⟧}\AgdaSpace{}%
\AgdaSymbol{→}\AgdaSpace{}%
\AgdaOperator{\AgdaFunction{𝓔⟦}}\AgdaSpace{}%
\AgdaBound{e}\AgdaSpace{}%
\AgdaOperator{\AgdaFunction{⟧}}\AgdaSpace{}%
\AgdaSymbol{(}\AgdaBound{⟦α⟧}\AgdaSpace{}%
\AgdaOperator{\AgdaInductiveConstructor{∷}}\AgdaSpace{}%
\AgdaBound{η}\AgdaSymbol{)}\AgdaSpace{}%
\AgdaSymbol{(}\AgdaFunction{extend-tskip}\AgdaSpace{}%
\AgdaBound{γ}\AgdaSymbol{)}\<%
\\
\>[0]\AgdaOperator{\AgdaFunction{𝓔⟦}}\AgdaSpace{}%
\AgdaOperator{\AgdaInductiveConstructor{\AgdaUnderscore{}∙\AgdaUnderscore{}}}\AgdaSpace{}%
\AgdaSymbol{\{}\AgdaArgument{T}\AgdaSpace{}%
\AgdaSymbol{=}\AgdaSpace{}%
\AgdaBound{T}\AgdaSymbol{\}}\AgdaSpace{}%
\AgdaBound{e}\AgdaSpace{}%
\AgdaBound{T′}\AgdaSpace{}%
\AgdaOperator{\AgdaFunction{⟧}}\AgdaSpace{}%
\AgdaBound{η}\AgdaSpace{}%
\AgdaBound{γ}\AgdaSpace{}%
\AgdaSymbol{=}\<%
\\
\>[0][@{}l@{\AgdaIndent{0}}]%
\>[2]\AgdaFunction{subst}\AgdaSpace{}%
\AgdaFunction{id}\AgdaSpace{}%
\AgdaSymbol{(}\AgdaFunction{sym}\AgdaSpace{}%
\AgdaSymbol{(}\AgdaFunction{single-subst-preserves}\AgdaSpace{}%
\AgdaBound{η}\AgdaSpace{}%
\AgdaBound{T′}\AgdaSpace{}%
\AgdaBound{T}\AgdaSymbol{))}\<%
\\
\>[2][@{}l@{\AgdaIndent{0}}]%
\>[4]\AgdaSymbol{(}\AgdaOperator{\AgdaFunction{𝓔⟦}}\AgdaSpace{}%
\AgdaBound{e}\AgdaSpace{}%
\AgdaOperator{\AgdaFunction{⟧}}\AgdaSpace{}%
\AgdaBound{η}\AgdaSpace{}%
\AgdaBound{γ}\AgdaSpace{}%
\AgdaSymbol{(}\AgdaOperator{\AgdaFunction{𝓣⟦}}\AgdaSpace{}%
\AgdaBound{T′}\AgdaSpace{}%
\AgdaOperator{\AgdaFunction{⟧}}\AgdaSpace{}%
\AgdaBound{η}\AgdaSymbol{))}\<%
\end{code}}

\section{Types}
\label{sec:types}

The definition of the type language is taken literally from Leivant's
paper. It is defined as an inductive type parameterized over a level
environment (that assigns levels to free type variables) and indexed
over the level of the type. 
\TFType
The unit type lives at level $0$. Type variables live at their
declared level.
The level of a function type $S \Rightarrow T$ is the maximum of the
levels of $S$ and $T$. The level of a universal quantification at
level $l$ is the maximum of $l + 1$ and the level of the body.

As for the simply-typed lambda calculus, we can define a compositional
mapping from type syntax to Agda types.
\TFTSem
Given a type at level $l$, this function returns an Agda type in
$\ASet~l$. To do so it needs a domain environment to interpret type
variables. This environment gets extended in the last clause that maps
universal quantification to a dependent function that takes an element
of $\ASet~l$ and pushes it on the environment.

The type of the domain environment is interesting because its range type is
unusual.
\TFTEnv
As a value in the environment (a type, colloquially speaking) can live
in a $\ASet~l$, for any $l$, we 
cannot assign the type any fixed level. Agda's solution is to provide
a limit type $\ASeto$, which we use in this definition. 

\section{Expressions}
\label{sec:expressions}

Inspired by the encoding of System $F\omega$ by Chapman and coworkers
\cite{DBLP:conf/mpc/ChapmanKNW19}, we define a 
unified environment for type variables and term variables. Type
environments grow to the left.
\TFTVEnv
Membership of a term variable in a type environment is defined by the
inn relation.
\TFinn
In the last alternative, we skip over a type binding. Hence, the type
$T$ we find under the binding must be weakened to account for the
extra type variables. Weakening is a special case of renaming, which
is implemented as advocated by Benton and coworkers \cite{DBLP:journals/jar/BentonHKM12}.

The type of expressions is now given as follows.
\TFExpr
Variables, lambda abstractions, and application are encoded just like
for the simply-typed lambda calculus.
Type abstraction takes a level $l$ and a body where the new type
variable is bound to $l$. 
Type application takes an expression with universal quantification at
level $l$ and a type $T'$ of level $l$. It constructs an expression
where type $T'$ has been substituted in the body $T$ of the
quantified type. Substitution is defined as in PLFA
\cite{DBLP:journals/jar/BentonHKM12,DBLP:journals/scp/KokkeSW20}. 

\section{Semantics}
\label{sec:semantics}

It remains to define a compositional function from the expression
syntax to the semantic domain that we already prepared in
Section~\ref{sec:types}. We start with value environments.
\TFVEnv
Value environments are represented as functions---we could have done
that in the simply-typed interpreter, too. They are indexed by a
domain environment to be able to calculate the correct return type.

The definition of the interpretation function follows.
\TFExprSem
The cases for term variables, lambda abstraction, and application are
similar to the  simply-typed lambda calculus.

The first issue arises in the case for type abstraction. We interpret
a type abstraction at level $l$ as a function with argument type
$\ASet~l$. This argument as to be pushed onto the domain environment
$\eta$ and we have to account at the value level for the additional
type variable in the type environment. The following function
adapts the types.
\TFExtendTskip
The lemma we need in the rewrite clause proves that interpreting a
weakened type in an extended domain environment gives the same result
as interpreting the type in the orginal domain environment.
The statement of this lemma is more general as it applies to arbitrary
renamings:
\TFRenPreserverSemanticsType
The argument $ren*$ roughly states that $\eta_1$ and $\eta_2$ are domain
environments related by renaming $\rho$ as follows  $\eta_1 \equiv \eta_2 \circ \rho$.
The proof of the lemma is by induction on $T$, but the case for
universal quantification comes with a surprising goal:
\TFProblematicGoal
We can show that the ranges of the function are equal
with the inductive hypothesis. But the usual extensionality principle does
not let us expose this equation. However, it can be used to prove a
dependent extensionality principle (from the standard library), which enables us to complete the proof.
\TFDependentExt

The final case for type application opens two different cans of
worms. First, the type of the right hand side does not match the
expected type. Essentially, we have to prove that the composition of
the meaning function for types commutes with substitution. Here
$T[T']$ substitutes $T'$ for the outermost variable of $T$.
\TFSingleSubstPreserves
Second, some steps in the proof involve equalities over entities of
$\ASeto$. These cannot be handled with the standard definition of
propositional equality which works parametrically for entities of
$\ASet~l$, for any $l$, but not for $\ASeto$. While it is easy to
define these equalities, it is somewhat tedious to re-establish
standard lemmas for transforming equality proofs
like cong, subst, and trans to deal with $\ASeto$.

\section{Future Work}
\label{sec:future-work}

We are currently formalizing the small-step semantics of the language
with the goal of proving an adequacy theorem for reduction with
respect to the denotational semantics.


%%
%% The acknowledgments section is defined using the "acks" environment
%% (and NOT an unnumbered section). This ensures the proper
%% identification of the section in the article metadata, and the
%% consistent spelling of the heading.
% \begin{acks}
% To Robert, for the bagels and explaining CMYK and color spaces.
% \end{acks}

%%
%% The next two lines define the bibliography style to be used, and
%% the bibliography file.
\bibliographystyle{ACM-Reference-Format}
\bibliography{references}


%%
%% If your work has an appendix, this is the place to put it.
% \appendix


\end{document}
\endinput
%%
%% End of file `sample-acmsmall-submission.tex'.
